\documentclass[a4paper,pdftex,10pt]{article}
\usepackage[margin=2.5cm,nohead]{geometry}
\usepackage[utf8]{inputenc}
\usepackage[T1]{fontenc} 
\usepackage[english,slovene]{babel} 
\usepackage{amsmath,amsfonts,amsthm,amssymb,mathrsfs,empheq} % Math packages
\usepackage{mathtools}
\usepackage{dsfont}
\usepackage{wrapfig}
\usepackage[pdftex]{graphicx}
%\usepackage{makeidx}
\usepackage{url}
\usepackage{caption}
\usepackage{subcaption}
\usepackage{tabularx}
\usepackage{float}

\usepackage[version=3]{mhchem} %kemija

\DeclarePairedDelimiter{\evdel}{\langle}{\rangle}   %pricakovana vrednost


\renewcommand{\vec}[1]{\boldsymbol{\mathbf{#1}}}                                        
\newcommand{\ihat}[0]{\boldsymbol{\mathbf{\oldhat{\textbf{\i}}}}} % pokončna j in i (j i n i
\newcommand{\iu}{{i\mkern1mu}}	    %imaginarno število
\DeclarePairedDelimiterX{\norm}[1]{\lVert}{\rVert}{#1} %norma

\usepackage{fancyhdr} % Custom headers and footers
\pagestyle{fancyplain} % Makes all pages in the document conform to the custom headers and footers
\fancyhead{} % No page header - if you want one, create it in the same way as the footers below
\fancyfoot[L]{} % Empty left footer
\fancyfoot[C]{} % Empty center footer
\fancyfoot[R]{\thepage} % Page numbering for right footer
\renewcommand{\headrulewidth}{0pt} % Remove header underlines
\renewcommand{\footrulewidth}{0pt} % Remove footer underlines
\setlength{\headheight}{13.6pt} % Customize the height of the header

%\numberwithin{equation}{section} % Number equations within sections (i.e. 1.1, 1.2, 2.1, 2.2 instead of 1, 2, 3, 4)
\numberwithin{figure}{section} % Number figures within sections (i.e. 1.1, 1.2, 2.1, 2.2 instead of 1, 2, 3, 4)
%\numberwithin{table}{section} % Number tables within sections (i.e. 1.1, 1.2, 2.1, 2.2 instead of 1, 2, 3, 4)

\setlength\parindent{0pt} % Removes all indentation from paragraphs - comment this line for an assignment with lots of text

%----------------------------------------------------------------------------------------
%	TITLE SECTION
%----------------------------------------------------------------------------------------

\newcommand{\horrule}[1]{\rule{\linewidth}{#1}} % Create horizontal rule command with 1 argument of height

\title{	
\normalfont \normalsize 
\textsc{Modelska analiza 1} \\ [25pt] % Your university, school and/or department name(s)
%\horrule{0.2pt} \\[0.4cm] % Thin top horizontal rule
\huge 12. naloga - Metoda maksimalne entropije in linearna napoved\\ % The assignment title
%\horrule{0.2pt} \\[0.5cm] % Thick bottom horizontal rule
}

\author{Tina Klobas} % Your name

\date{\normalsize\today} % Today's date or a custom date

\begin{document}

\maketitle % Print the title

\section{Opis problema}
Pri tej nalogi si bomo pogledali metodo maksimalne entropije za določanje frekvenčnih 
spektrov signalov.

%----------------------------------------------------------------------------------------
%	PROBLEM 1
%----------------------------------------------------------------------------------------
\section{Določanje frekvenčnega spektra}
Določi frekvenčni spekter signalov iz datotek \path{val2.dat}, \path{val3.dat} in 
\path{co2.dat} z~metodo maksimalne entropije. V~slednji datoteki s~koncentracijo CO2 
v~zraku upoštevaj splošen (linearen) letni trend. Preizkusi delovanje metode v~odvisnosti 
od števila polov in od gostote prikaza. Pri CO2 si oglej še njihovo lego. Primerjaj 
natančnost metode z rezultati, ki jih da FFT ob uporabi filtrov. Sestavi tudi testni 
signal iz vsote sinusnih členov z bližnjimi frekvencami ter razišči ločljivost metode.

\subsection{Reševanje in rezultati}
Za metodo maksimalne entropije bomo potrebovali avtokorelacijsko funkcijo signala:
\begin{equation}
    R(i) = \frac{1}{N-i} \sum_{n=0}^{N-1-i} s_n s_{n+i},
\end{equation}
kjer so $s_n$ izhodni podatki iz sistema. Pri tem problemu želimo minimizirati kvadrat
napake napovedi:
\begin{equation}
    \text{min} = E[s_n^2] + \sum_{k=1}^p a_k E[s_n s_{n-k} ].
\end{equation}
Koeficiente $a_k$ dobimo z reševanje Toeplitzovega sistema, ki jih nato uporabimo za izračun
gostote spektra:
\begin{equation}
    P(\omega) = \frac{1}{| 1+ \sum_{k=1}^p a_k \mathrm{e}^{-i\omega k}|^2}.
\end{equation}
\pagebreak

%----------------------------------------------------------------------------------------
%	PROBLEM 2
%----------------------------------------------------------------------------------------
\section{Wienerjev filter}
\end{document}
