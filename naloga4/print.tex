\documentclass[a4paper,pdftex,10pt]{article}
\usepackage[a4paper]{geometry}
\usepackage[utf8]{inputenc}
\usepackage[T1]{fontenc} 
\usepackage[english,slovene]{babel} 
\usepackage{amsmath,amsfonts,amsthm,amssymb,mathrsfs,empheq} % Math packages
\usepackage{wasysym} %not je procent in promil
\usepackage{mathtools}
\usepackage{dsfont}
\usepackage[pdftex]{graphicx}
%\usepackage{makeidx}


\renewcommand{\vec}[1]{\boldsymbol{\mathbf{#1}}}                                        
\newcommand{\ihat}[0]{\boldsymbol{\mathbf{\oldhat{\textbf{\i}}}}} % pokončna j in i (j i n i
\newcommand{\iu}{{i\mkern1mu}}	    %imaginarno število

\usepackage{fancyhdr} % Custom headers and footers
\pagestyle{fancyplain} % Makes all pages in the document conform to the custom headers and footers
\fancyhead{} % No page header - if you want one, create it in the same way as the footers below
\fancyfoot[L]{} % Empty left footer
\fancyfoot[C]{} % Empty center footer
\fancyfoot[R]{\thepage} % Page numbering for right footer
\renewcommand{\headrulewidth}{0pt} % Remove header underlines
\renewcommand{\footrulewidth}{0pt} % Remove footer underlines
\setlength{\headheight}{13.6pt} % Customize the height of the header

\numberwithin{equation}{section} % Number equations within sections (i.e. 1.1, 1.2, 2.1, 2.2 instead of 1, 2, 3, 4)
\numberwithin{figure}{section} % Number figures within sections (i.e. 1.1, 1.2, 2.1, 2.2 instead of 1, 2, 3, 4)
\numberwithin{table}{section} % Number tables within sections (i.e. 1.1, 1.2, 2.1, 2.2 instead of 1, 2, 3, 4)

\setlength\parindent{0pt} % Removes all indentation from paragraphs - comment this line for an assignment with lots of text

%----------------------------------------------------------------------------------------
%	TITLE SECTION
%----------------------------------------------------------------------------------------

\newcommand{\horrule}[1]{\rule{\linewidth}{#1}} % Create horizontal rule command with 1 argument of height

\title{	
\normalfont \normalsize 
\textsc{Modelska analiza 1} \\ [25pt] % Your university, school and/or department name(s)
%\horrule{0.2pt} \\[0.4cm] % Thin top horizontal rule
\huge 4. naloga\\ % The assignment title
%\horrule{0.2pt} \\[0.5cm] % Thick bottom horizontal rule
}

\author{Tina Klobas} % Your name

\date{\normalsize\today} % Today's date or a custom date

\begin{document}

\maketitle % Print the title

%----------------------------------------------------------------------------------------
%	PROBLEM 1
%----------------------------------------------------------------------------------------

\section{Model Lotka-Volterra}
Opazujemo življenje lisic ter zajcev na določenem ozemlju. Privzamemo, da je za lisico 
zajec edini vir hrane, za zajca pa je lisica edini naravni plenilec. Ob izolaciji vrst 
število lisic \emph{L}, zaradi pomanjkanja hrane, eksponentno pada z~naravno 
konstanto umrljivosti $\gamma$, število zajcev \emph{Z}, zaradi proste rasti brez 
naravnega plenilca, pa eksponentno narašča z~naravno konstanto rodnosti $\alpha$. Ob 
sobivanju vrst na istem področju tako dobimo sklopljen sistem enačb, ki opisuje ta pojav:
\begin{align}\label{lotka-volterra}
    \frac{\mathrm{d}Z}{\mathrm{d}t} &= \alpha L - \beta Z L \\
    \frac{\mathrm{d}L}{\mathrm{d}t} &= -\gamma L + \delta Z L,
\end{align}
kjer sta $\beta$ umrljivost plena in $\delta$ rodnost plenilcev. Sistem enačb lahko spravimo
v~brezdimenzijsko obliko, da si zmanjšamo število parametrov s~katerimi se moramo ukvarjati.
Tako z~definiranima količinama $l=\beta/\alpha L$ in $z=\delta/\gamma Z$ ter 
brezdimenzijskim časom $\tau = t\sqrt{\alpha \gamma}$ prepišemo sistem~\ref{lotka-volterra}
\begin{align}\label{LV}
    \frac{\mathrm{d}z}{\mathrm{d}\tau} &= p z (1-l) \\
    \frac{\mathrm{d}l}{\mathrm{d}\tau} &= \frac{l}{p} (z-1).
\end{align}
$p = \sqrt{\alpha/\gamma}$ je tako edini preostali prosti parameter. 

\begin{figure}    
    \centering
    \resizebox{0.8\linewidth}{!}{\input{graf1.tex}}
    \caption{Potek spreminjanja populacij (z~modro zajci in z~oranžno lisice) pri začetnih 
    vrednostih $z_0=1$ in $l_0=0,25$.}
    \label{slika1}
\end{figure}
\begin{figure}    
    \centering
    \resizebox{0.8\linewidth}{!}{\input{graf2.tex}}
    \caption{Potek spreminjanja populacij (z~modro zajci in z~oranžno lisice) pri začetnih
    vrednostih $z_0=2$ in $l_0=1$.}
    \label{slika2}
\end{figure}
\begin{figure}    
    \centering
    \resizebox{0.8\linewidth}{!}{\input{graf3.tex}}
    \caption{Fazni potek populacij kjer spreminjamo $p$ in imamo konstantne začetne 
    vrednosti.}
    \label{slika3}
\end{figure}

Pri obeh grafih~\ref{slika1},~\ref{slika2} vidimo podobno obnašanje s~spreminjanjem 
parametra $p$;
pri visokih $p$ je amplituda nihanja lisic večja pri, majhnih $p$ pa amplituda zajcev.
Tako obnašanje je vidno tudi na faznem diagramu~\ref{slika3} kjer so temnejši elipsoidi 
daljši v~smeri lisic, svetlejši pa v~smeri zajcev.\\
Na grafu~\ref{slika5} je narisano spreminjanje populacij plena in lovca pri paroma enakih
začetnih vrednostih $z_0=l_0$. Vidimo, da z~nižanjem le-teh premikamo prvi maksimum proti
kasnejšim časom, hkrati pa se viša tudi njegova vrednost. Slednje je posledica tega, da~ko
smo spravili sistem v~brezdimenzijsko obliko, smo vzeli $l = L \frac{\beta}{\alpha}$,
se pravi smo število lisic pomnožili z~razmerjem med njeno smrtnostjo in rodnostjo ; nižja
kot bo začetna vrednost $l_0$ večjo rodnost bodo imele lisice. (Enako velja tudi za 
$z_0$.)

\begin{figure}    
    \centering
    \resizebox{0.8\linewidth}{!}{% GNUPLOT: LaTeX picture with Postscript
\begingroup
  \makeatletter
  \providecommand\color[2][]{%
    \GenericError{(gnuplot) \space\space\space\@spaces}{%
      Package color not loaded in conjunction with
      terminal option `colourtext'%
    }{See the gnuplot documentation for explanation.%
    }{Either use 'blacktext' in gnuplot or load the package
      color.sty in LaTeX.}%
    \renewcommand\color[2][]{}%
  }%
  \providecommand\includegraphics[2][]{%
    \GenericError{(gnuplot) \space\space\space\@spaces}{%
      Package graphicx or graphics not loaded%
    }{See the gnuplot documentation for explanation.%
    }{The gnuplot epslatex terminal needs graphicx.sty or graphics.sty.}%
    \renewcommand\includegraphics[2][]{}%
  }%
  \providecommand\rotatebox[2]{#2}%
  \@ifundefined{ifGPcolor}{%
    \newif\ifGPcolor
    \GPcolortrue
  }{}%
  \@ifundefined{ifGPblacktext}{%
    \newif\ifGPblacktext
    \GPblacktexttrue
  }{}%
  % define a \g@addto@macro without @ in the name:
  \let\gplgaddtomacro\g@addto@macro
  % define empty templates for all commands taking text:
  \gdef\gplbacktext{}%
  \gdef\gplfronttext{}%
  \makeatother
  \ifGPblacktext
    % no textcolor at all
    \def\colorrgb#1{}%
    \def\colorgray#1{}%
  \else
    % gray or color?
    \ifGPcolor
      \def\colorrgb#1{\color[rgb]{#1}}%
      \def\colorgray#1{\color[gray]{#1}}%
      \expandafter\def\csname LTw\endcsname{\color{white}}%
      \expandafter\def\csname LTb\endcsname{\color{black}}%
      \expandafter\def\csname LTa\endcsname{\color{black}}%
      \expandafter\def\csname LT0\endcsname{\color[rgb]{1,0,0}}%
      \expandafter\def\csname LT1\endcsname{\color[rgb]{0,1,0}}%
      \expandafter\def\csname LT2\endcsname{\color[rgb]{0,0,1}}%
      \expandafter\def\csname LT3\endcsname{\color[rgb]{1,0,1}}%
      \expandafter\def\csname LT4\endcsname{\color[rgb]{0,1,1}}%
      \expandafter\def\csname LT5\endcsname{\color[rgb]{1,1,0}}%
      \expandafter\def\csname LT6\endcsname{\color[rgb]{0,0,0}}%
      \expandafter\def\csname LT7\endcsname{\color[rgb]{1,0.3,0}}%
      \expandafter\def\csname LT8\endcsname{\color[rgb]{0.5,0.5,0.5}}%
    \else
      % gray
      \def\colorrgb#1{\color{black}}%
      \def\colorgray#1{\color[gray]{#1}}%
      \expandafter\def\csname LTw\endcsname{\color{white}}%
      \expandafter\def\csname LTb\endcsname{\color{black}}%
      \expandafter\def\csname LTa\endcsname{\color{black}}%
      \expandafter\def\csname LT0\endcsname{\color{black}}%
      \expandafter\def\csname LT1\endcsname{\color{black}}%
      \expandafter\def\csname LT2\endcsname{\color{black}}%
      \expandafter\def\csname LT3\endcsname{\color{black}}%
      \expandafter\def\csname LT4\endcsname{\color{black}}%
      \expandafter\def\csname LT5\endcsname{\color{black}}%
      \expandafter\def\csname LT6\endcsname{\color{black}}%
      \expandafter\def\csname LT7\endcsname{\color{black}}%
      \expandafter\def\csname LT8\endcsname{\color{black}}%
    \fi
  \fi
    \setlength{\unitlength}{0.0500bp}%
    \ifx\gptboxheight\undefined%
      \newlength{\gptboxheight}%
      \newlength{\gptboxwidth}%
      \newsavebox{\gptboxtext}%
    \fi%
    \setlength{\fboxrule}{0.5pt}%
    \setlength{\fboxsep}{1pt}%
\begin{picture}(7180.00,4300.00)%
    \gplgaddtomacro\gplbacktext{%
      \csname LTb\endcsname%%
      \put(504,408){\makebox(0,0)[r]{\strut{}$0$}}%
      \csname LTb\endcsname%%
      \put(504,1023){\makebox(0,0)[r]{\strut{}$20$}}%
      \csname LTb\endcsname%%
      \put(504,1637){\makebox(0,0)[r]{\strut{}$40$}}%
      \csname LTb\endcsname%%
      \put(504,2252){\makebox(0,0)[r]{\strut{}$60$}}%
      \csname LTb\endcsname%%
      \put(504,2866){\makebox(0,0)[r]{\strut{}$80$}}%
      \csname LTb\endcsname%%
      \put(504,3481){\makebox(0,0)[r]{\strut{}$100$}}%
      \csname LTb\endcsname%%
      \put(504,4095){\makebox(0,0)[r]{\strut{}$120$}}%
      \csname LTb\endcsname%%
      \put(616,204){\makebox(0,0){\strut{}$0$}}%
      \csname LTb\endcsname%%
      \put(1861,204){\makebox(0,0){\strut{}$500$}}%
      \csname LTb\endcsname%%
      \put(3107,204){\makebox(0,0){\strut{}$1000$}}%
      \csname LTb\endcsname%%
      \put(4352,204){\makebox(0,0){\strut{}$1500$}}%
      \csname LTb\endcsname%%
      \put(5598,204){\makebox(0,0){\strut{}$2000$}}%
      \csname LTb\endcsname%%
      \put(6843,204){\makebox(0,0){\strut{}$2500$}}%
    }%
    \gplgaddtomacro\gplfronttext{%
      \csname LTb\endcsname%%
      \put(3931,3688){\makebox(0,0)[l]{\strut{}$B=$ vsaka $5./10.$ meritev}}%
      \csname LTb\endcsname%%
      \put(3931,3892){\makebox(0,0)[l]{\strut{}$A=$ vsaka meritev}}%
      \csname LTb\endcsname%%
      \put(3730,1637){\makebox(0,0)[l]{\strut{}$\mu_A$}}%
      \csname LTb\endcsname%%
      \put(3730,1483){\makebox(0,0)[l]{\strut{}$\mu_B$}}%
    }%
    \gplbacktext
    \put(0,0){\includegraphics{graf5}}%
    \gplfronttext
  \end{picture}%
\endgroup
}
    \caption{Na levem grafu je prikazano obnašanje populacije zajcev, na desni pa lisic
    pri različnih začetnih vrednostih ($l_0=z_0$) in enaki vrednosti $p=1$.}
    \label{slika5}
\end{figure}

\subsection{Nepremični točki}
Pogledamo kje ima sistem enačb~\ref{LV} ničelna odvoda:
\begin{align*}
    0 &= p z (1-l) \\
    0 &= \frac{l}{p} (z-1).
\end{align*}
Rešitvi sta točki $T_1=(0,0)$ in $T_2=(1,1)$. 

\subsubsection*{Okolica točke $l=0, z=0$:}
Za majhne odmike od točke $T_1$ lahko zanemarimo člena produktni člen $zl$ 
v~enačbah~\ref{LV} in dobimo diferencialni enačbi in rešitvi
\begin{align*}
    \frac{\mathrm{d}z}{\mathrm{d}\tau} = p z &\Rightarrow z = z_0 \mathrm{e}^{p\tau} \\
    \frac{\mathrm{d}l}{\mathrm{d}\tau} = -\frac{l}{p} &\Rightarrow l = l_0 
    \mathrm{e}^{-\tau/p},
\end{align*}
kar opisuje izumiranje populacije lisic in eksponentno rast populacije zajcev v~začetnem
opisu izoliranih vrst. To območje je prikazano na grafih~\ref{slika6}, kjer se tudi vidi
eksponentno padanje/naraščanje populacij.
Točka $T_1$ je sedlo; funkcija v~eni smeri narašča. v~drugi pa pada. To lahko preverimo
tudi z~Jakobijevo matriko odvodov sistema~\ref{LV}:
\begin{equation}\label{Jacobi}
    J = \begin{bmatrix}
	\frac{\partial \dot{z}}{\partial z} 
	& \frac{\partial \dot{z}}{\partial l} \\
	\frac{\partial \dot{l}}{\partial z} 
	& \frac{\partial \dot{l}}{\partial l} 
    \end{bmatrix} = 
    \begin{bmatrix}
	p(1-l) & -pz \\
	\frac{l}{p} & \frac{z-1}{p}
    \end{bmatrix},
\end{equation}
saj, ko vstavimo točko $(0,0)$ dobimo matriko
\begin{equation}
    \begin{bmatrix}
	p & 0 \\
	0 & -\frac{1}{p}
    \end{bmatrix},
\end{equation}
ki bo imela vedno nasprotno predznačeni lastni vrednosti, se pravi bo zmeraj nedefinitna
(kar je lastnost sedla).

\begin{figure}    
    \centering
    \resizebox{0.8\textwidth}{!}{% GNUPLOT: LaTeX picture with Postscript
\begingroup
  \makeatletter
  \providecommand\color[2][]{%
    \GenericError{(gnuplot) \space\space\space\@spaces}{%
      Package color not loaded in conjunction with
      terminal option `colourtext'%
    }{See the gnuplot documentation for explanation.%
    }{Either use 'blacktext' in gnuplot or load the package
      color.sty in LaTeX.}%
    \renewcommand\color[2][]{}%
  }%
  \providecommand\includegraphics[2][]{%
    \GenericError{(gnuplot) \space\space\space\@spaces}{%
      Package graphicx or graphics not loaded%
    }{See the gnuplot documentation for explanation.%
    }{The gnuplot epslatex terminal needs graphicx.sty or graphics.sty.}%
    \renewcommand\includegraphics[2][]{}%
  }%
  \providecommand\rotatebox[2]{#2}%
  \@ifundefined{ifGPcolor}{%
    \newif\ifGPcolor
    \GPcolortrue
  }{}%
  \@ifundefined{ifGPblacktext}{%
    \newif\ifGPblacktext
    \GPblacktexttrue
  }{}%
  % define a \g@addto@macro without @ in the name:
  \let\gplgaddtomacro\g@addto@macro
  % define empty templates for all commands taking text:
  \gdef\gplbacktext{}%
  \gdef\gplfronttext{}%
  \makeatother
  \ifGPblacktext
    % no textcolor at all
    \def\colorrgb#1{}%
    \def\colorgray#1{}%
  \else
    % gray or color?
    \ifGPcolor
      \def\colorrgb#1{\color[rgb]{#1}}%
      \def\colorgray#1{\color[gray]{#1}}%
      \expandafter\def\csname LTw\endcsname{\color{white}}%
      \expandafter\def\csname LTb\endcsname{\color{black}}%
      \expandafter\def\csname LTa\endcsname{\color{black}}%
      \expandafter\def\csname LT0\endcsname{\color[rgb]{1,0,0}}%
      \expandafter\def\csname LT1\endcsname{\color[rgb]{0,1,0}}%
      \expandafter\def\csname LT2\endcsname{\color[rgb]{0,0,1}}%
      \expandafter\def\csname LT3\endcsname{\color[rgb]{1,0,1}}%
      \expandafter\def\csname LT4\endcsname{\color[rgb]{0,1,1}}%
      \expandafter\def\csname LT5\endcsname{\color[rgb]{1,1,0}}%
      \expandafter\def\csname LT6\endcsname{\color[rgb]{0,0,0}}%
      \expandafter\def\csname LT7\endcsname{\color[rgb]{1,0.3,0}}%
      \expandafter\def\csname LT8\endcsname{\color[rgb]{0.5,0.5,0.5}}%
    \else
      % gray
      \def\colorrgb#1{\color{black}}%
      \def\colorgray#1{\color[gray]{#1}}%
      \expandafter\def\csname LTw\endcsname{\color{white}}%
      \expandafter\def\csname LTb\endcsname{\color{black}}%
      \expandafter\def\csname LTa\endcsname{\color{black}}%
      \expandafter\def\csname LT0\endcsname{\color{black}}%
      \expandafter\def\csname LT1\endcsname{\color{black}}%
      \expandafter\def\csname LT2\endcsname{\color{black}}%
      \expandafter\def\csname LT3\endcsname{\color{black}}%
      \expandafter\def\csname LT4\endcsname{\color{black}}%
      \expandafter\def\csname LT5\endcsname{\color{black}}%
      \expandafter\def\csname LT6\endcsname{\color{black}}%
      \expandafter\def\csname LT7\endcsname{\color{black}}%
      \expandafter\def\csname LT8\endcsname{\color{black}}%
    \fi
  \fi
    \setlength{\unitlength}{0.0500bp}%
    \ifx\gptboxheight\undefined%
      \newlength{\gptboxheight}%
      \newlength{\gptboxwidth}%
      \newsavebox{\gptboxtext}%
    \fi%
    \setlength{\fboxrule}{0.5pt}%
    \setlength{\fboxsep}{1pt}%
\begin{picture}(7180.00,4300.00)%
    \gplgaddtomacro\gplbacktext{%
      \csname LTb\endcsname%%
      \put(728,408){\makebox(0,0)[r]{\strut{}$-4000$}}%
      \csname LTb\endcsname%%
      \put(728,1023){\makebox(0,0)[r]{\strut{}$-2000$}}%
      \csname LTb\endcsname%%
      \put(728,1637){\makebox(0,0)[r]{\strut{}$0$}}%
      \csname LTb\endcsname%%
      \put(728,2252){\makebox(0,0)[r]{\strut{}$2000$}}%
      \csname LTb\endcsname%%
      \put(728,2866){\makebox(0,0)[r]{\strut{}$4000$}}%
      \csname LTb\endcsname%%
      \put(728,3481){\makebox(0,0)[r]{\strut{}$6000$}}%
      \csname LTb\endcsname%%
      \put(728,4095){\makebox(0,0)[r]{\strut{}$8000$}}%
      \csname LTb\endcsname%%
      \put(840,204){\makebox(0,0){\strut{}$-5000$}}%
      \csname LTb\endcsname%%
      \put(1698,204){\makebox(0,0){\strut{}$0$}}%
      \csname LTb\endcsname%%
      \put(2555,204){\makebox(0,0){\strut{}$5000$}}%
      \csname LTb\endcsname%%
      \put(3413,204){\makebox(0,0){\strut{}$10000$}}%
      \csname LTb\endcsname%%
      \put(4270,204){\makebox(0,0){\strut{}$15000$}}%
      \csname LTb\endcsname%%
      \put(5128,204){\makebox(0,0){\strut{}$20000$}}%
      \csname LTb\endcsname%%
      \put(5985,204){\makebox(0,0){\strut{}$25000$}}%
      \csname LTb\endcsname%%
      \put(6843,204){\makebox(0,0){\strut{}$30000$}}%
    }%
    \gplgaddtomacro\gplfronttext{%
      \csname LTb\endcsname%%
      \put(1369,3484){\makebox(0,0)[l]{\strut{}brez hitrosti}}%
      \csname LTb\endcsname%%
      \put(1369,3688){\makebox(0,0)[l]{\strut{}brez lokacija}}%
      \csname LTb\endcsname%%
      \put(1369,3892){\makebox(0,0)[l]{\strut{}kontrola}}%
    }%
    \gplbacktext
    \put(0,0){\includegraphics{graf6}}%
    \gplfronttext
  \end{picture}%
\endgroup
}
    \caption{Na levem grafu je prikazano obnašanje populacije zajcev, na desni pa lisic
    pri začetnih pogojih $l_0=0,099, \, z_0=0,001$ in različnih vrednostih $p$.}
    \label{slika6}
\end{figure}


\subsubsection*{Okolica točke $l=1, \; z=1$:}
Podobno naredimo okoli druge točke: $l=1+\lambda$ in $z=1+\zeta$ in preoblikujemo sistem:
\begin{align}
    \frac{\mathrm{d}z}{\mathrm{d}\tau} = \frac{\mathrm{d}\zeta}{\mathrm{d}\tau} &= - p 
    (1+\zeta)\lambda \approx -p\lambda \\
    \frac{\mathrm{d}l}{\mathrm{d}\tau} = \frac{\mathrm{d}\lambda}{\mathrm{d}\tau}&= 
    \frac{\zeta}{p} (1 + \lambda) \approx \frac{\zeta}{p} 
\end{align}
in če naredimo še en odvod po času dobimo rešitev: 
\begin{align}
    \ddot{\zeta} = -\zeta \; \; &\Rightarrow \; \; \zeta=A\sin{\tau} + B\cos{\tau}\\
    \ddot{\lambda} = -\lambda \; \; &\Rightarrow \; \; \lambda=C\sin{\tau} + d\cos{\tau}.
\end{align}
Preverimo kakšna stacionarna točka je $T_2$ in s~pomočjo odvodov~\ref{Jacobi} dobimo
matriko
\begin{equation}
    \begin{bmatrix}
	0 & -p \\
	\frac{1}{p} & 0
    \end{bmatrix},
\end{equation}
z~lastnima vrednostma $\eta_1=-\iu$ in $\eta_2 =\iu$. Ker sta lastni vrednosti imaginarni
in nasprotno predznačeni s~tem potrdimo, že prej dobljeno periodično obnašanje populacije
lisic in zajcev v~okolici točke $(1,1)$. Perioda takega nihanja je ravno $\omega = 
2\pi/\sqrt{\eta_1\eta_2}=2\pi$. \\

\begin{figure}    
    \centering
    \resizebox{0.8\linewidth}{!}{% GNUPLOT: LaTeX picture with Postscript
\begingroup
  \makeatletter
  \providecommand\color[2][]{%
    \GenericError{(gnuplot) \space\space\space\@spaces}{%
      Package color not loaded in conjunction with
      terminal option `colourtext'%
    }{See the gnuplot documentation for explanation.%
    }{Either use 'blacktext' in gnuplot or load the package
      color.sty in LaTeX.}%
    \renewcommand\color[2][]{}%
  }%
  \providecommand\includegraphics[2][]{%
    \GenericError{(gnuplot) \space\space\space\@spaces}{%
      Package graphicx or graphics not loaded%
    }{See the gnuplot documentation for explanation.%
    }{The gnuplot epslatex terminal needs graphicx.sty or graphics.sty.}%
    \renewcommand\includegraphics[2][]{}%
  }%
  \providecommand\rotatebox[2]{#2}%
  \@ifundefined{ifGPcolor}{%
    \newif\ifGPcolor
    \GPcolortrue
  }{}%
  \@ifundefined{ifGPblacktext}{%
    \newif\ifGPblacktext
    \GPblacktexttrue
  }{}%
  % define a \g@addto@macro without @ in the name:
  \let\gplgaddtomacro\g@addto@macro
  % define empty templates for all commands taking text:
  \gdef\gplbacktext{}%
  \gdef\gplfronttext{}%
  \makeatother
  \ifGPblacktext
    % no textcolor at all
    \def\colorrgb#1{}%
    \def\colorgray#1{}%
  \else
    % gray or color?
    \ifGPcolor
      \def\colorrgb#1{\color[rgb]{#1}}%
      \def\colorgray#1{\color[gray]{#1}}%
      \expandafter\def\csname LTw\endcsname{\color{white}}%
      \expandafter\def\csname LTb\endcsname{\color{black}}%
      \expandafter\def\csname LTa\endcsname{\color{black}}%
      \expandafter\def\csname LT0\endcsname{\color[rgb]{1,0,0}}%
      \expandafter\def\csname LT1\endcsname{\color[rgb]{0,1,0}}%
      \expandafter\def\csname LT2\endcsname{\color[rgb]{0,0,1}}%
      \expandafter\def\csname LT3\endcsname{\color[rgb]{1,0,1}}%
      \expandafter\def\csname LT4\endcsname{\color[rgb]{0,1,1}}%
      \expandafter\def\csname LT5\endcsname{\color[rgb]{1,1,0}}%
      \expandafter\def\csname LT6\endcsname{\color[rgb]{0,0,0}}%
      \expandafter\def\csname LT7\endcsname{\color[rgb]{1,0.3,0}}%
      \expandafter\def\csname LT8\endcsname{\color[rgb]{0.5,0.5,0.5}}%
    \else
      % gray
      \def\colorrgb#1{\color{black}}%
      \def\colorgray#1{\color[gray]{#1}}%
      \expandafter\def\csname LTw\endcsname{\color{white}}%
      \expandafter\def\csname LTb\endcsname{\color{black}}%
      \expandafter\def\csname LTa\endcsname{\color{black}}%
      \expandafter\def\csname LT0\endcsname{\color{black}}%
      \expandafter\def\csname LT1\endcsname{\color{black}}%
      \expandafter\def\csname LT2\endcsname{\color{black}}%
      \expandafter\def\csname LT3\endcsname{\color{black}}%
      \expandafter\def\csname LT4\endcsname{\color{black}}%
      \expandafter\def\csname LT5\endcsname{\color{black}}%
      \expandafter\def\csname LT6\endcsname{\color{black}}%
      \expandafter\def\csname LT7\endcsname{\color{black}}%
      \expandafter\def\csname LT8\endcsname{\color{black}}%
    \fi
  \fi
    \setlength{\unitlength}{0.0500bp}%
    \ifx\gptboxheight\undefined%
      \newlength{\gptboxheight}%
      \newlength{\gptboxwidth}%
      \newsavebox{\gptboxtext}%
    \fi%
    \setlength{\fboxrule}{0.5pt}%
    \setlength{\fboxsep}{1pt}%
\begin{picture}(7200.00,4320.00)%
    \gplgaddtomacro\gplbacktext{%
      \csname LTb\endcsname%%
      \put(708,652){\makebox(0,0)[r]{\strut{}$0$}}%
      \csname LTb\endcsname%%
      \put(708,996){\makebox(0,0)[r]{\strut{}$0.1$}}%
      \csname LTb\endcsname%%
      \put(708,1341){\makebox(0,0)[r]{\strut{}$0.2$}}%
      \csname LTb\endcsname%%
      \put(708,1685){\makebox(0,0)[r]{\strut{}$0.3$}}%
      \csname LTb\endcsname%%
      \put(708,2029){\makebox(0,0)[r]{\strut{}$0.4$}}%
      \csname LTb\endcsname%%
      \put(708,2374){\makebox(0,0)[r]{\strut{}$0.5$}}%
      \csname LTb\endcsname%%
      \put(708,2718){\makebox(0,0)[r]{\strut{}$0.6$}}%
      \csname LTb\endcsname%%
      \put(708,3062){\makebox(0,0)[r]{\strut{}$0.7$}}%
      \csname LTb\endcsname%%
      \put(708,3406){\makebox(0,0)[r]{\strut{}$0.8$}}%
      \csname LTb\endcsname%%
      \put(708,3751){\makebox(0,0)[r]{\strut{}$0.9$}}%
      \csname LTb\endcsname%%
      \put(708,4095){\makebox(0,0)[r]{\strut{}$1$}}%
      \csname LTb\endcsname%%
      \put(820,448){\makebox(0,0){\strut{}$0$}}%
      \csname LTb\endcsname%%
      \put(1996,448){\makebox(0,0){\strut{}$100$}}%
      \csname LTb\endcsname%%
      \put(3173,448){\makebox(0,0){\strut{}$200$}}%
      \csname LTb\endcsname%%
      \put(4349,448){\makebox(0,0){\strut{}$300$}}%
      \csname LTb\endcsname%%
      \put(5525,448){\makebox(0,0){\strut{}$400$}}%
      \csname LTb\endcsname%%
      \put(6702,448){\makebox(0,0){\strut{}$500$}}%
    }%
    \gplgaddtomacro\gplfronttext{%
      \csname LTb\endcsname%%
      \put(186,2373){\rotatebox{-270}{\makebox(0,0){\strut{}amplituda}}}%
      \csname LTb\endcsname%%
      \put(3831,142){\makebox(0,0){\strut{}$x$}}%
      \csname LTb\endcsname%%
      \put(1349,3280){\makebox(0,0)[l]{\strut{}Welch}}%
      \csname LTb\endcsname%%
      \put(1349,3484){\makebox(0,0)[l]{\strut{}Hann}}%
      \csname LTb\endcsname%%
      \put(1349,3688){\makebox(0,0)[l]{\strut{}Bartlett}}%
      \csname LTb\endcsname%%
      \put(1349,3892){\makebox(0,0)[l]{\strut{}eksponent}}%
    }%
    \gplbacktext
    \put(0,0){\includegraphics[width={360.00bp},height={216.00bp}]{graf4}}%
    \gplfronttext
  \end{picture}%
\endgroup
}
    \caption{Na levem grafu je prikazano obnašanje populacij pri $p=1$ za začetni vrednosti
    $z_0=1,\; l_0=1.5$ (z~modro barvo) ter $z_0=1,\; l_0=0.95$ (z~oranžno). 
    Na desnem grafu je prikazan fazni diagram
    za $p=1$ in spreminjanje začetnih vrednosti $l_0$ od $1,05$ (najtemnejše barve) do 
    $0,95$ (najsvetlejše barve). }
    \label{slika4}
\end{figure}
Tako obnašanje je razvidno tudi z grafa~\ref{slika4}, kjer se vidi, da je 
zakasnitev med nihanjema ravno pol periode. Z~grafom~\ref{slika7} potrdimo, da se 
z~bližanjem točki $T_2$ vrednost periode približuje $\omega=2\pi$. 

\begin{figure}    
    \centering
    \resizebox{0.8\textwidth}{!}{\input{graf7.tex}}
    \caption{Na grafu je prikazano spreminjanje frekvence od začetne vrednosti $z_0$.}
    \label{slika7}
\end{figure}

\newpage

%----------------------------------------------------------------------------------------
%	PROBLEM 2
%----------------------------------------------------------------------------------------

\section{Model laserja}
Prejšnjemu problemu je podoben model delovanja laserja, ki ima dodaten člen $R$ zaradi 
konstantnega dovajanja novih atomov v~sistem:
\begin{align}\label{af}
    \dot{f} &= -\alpha f + Baf \\
    \dot{a} &= -\beta a - Baf + R.
\end{align}
Z~uvedbo novih količin 
\begin{align*}
    A &= \frac{B}{\alpha} a \\
    F &= \frac{B}{\beta} f \\
    \tau  &= t\sqrt{\alpha\beta}
\end{align*}
ter parametra izgub $p$ in črpanja $r$
\begin{align*}
    p &= \sqrt{\frac{\beta}{\alpha}} \\
    r &= \frac{BR}{\sqrt{\alpha^3\beta}}
\end{align*}
dobimo brezdimenzijski sistem spreminjanja količin atomov $A$ in fotonov $F$:
\begin{align}\label{laser}
    \dot{F} &= \frac{F}{p} \left(A - 1\right) \\
    \dot{A} &= r - pA\left(F + 1\right).
\end{align}
Zapišimo še Jakobijevo matriko sistema enačb, ki jo bomo kasneje uporabili:
\begin{equation}\label{Jacobi}
    J = \begin{bmatrix}
	\frac{\partial \dot{A}}{\partial A} 
	& \frac{\partial \dot{A}}{\partial F} \\
	\frac{\partial \dot{F}}{\partial A} 
	& \frac{\partial \dot{F}}{\partial F} 
    \end{bmatrix} = 
    \begin{bmatrix}
	-p(F+1) & -pA \\
	\frac{F}{p} & \frac{A-1}{p}
    \end{bmatrix}.
\end{equation}
Na grafu~\ref{slika8} vidimo kaj se dogaja v~primerih, ko nimamo črpanja ($r=0$), imamo pa 
izgube, torej $p>r$. Sistem zmeraj zamre, kako hitro, pa je odvisno od velikosti izgub
$p$, in sicer večje kot so, krajši je relaksacijski čas atomov, ter daljši fotonov; in 
obratno pri manjših izgubah. To je vidno tudi iz konveksnosti/konkavnosti desnega grafa.\\

Naslednji graf~\ref{slika9} prikazuje tri različne primere; ko je črpanje manjše, enako
in večje kot izgube. Edino pri primeru, ko je $r>p$ vidimo, da se je število fotonov 
povečalo, in preden se je ustalilo, je nekaj časa osciliralo. Slednje se je zgodilo tudi
pri atomih, kar je na desnem grafu vidno kot spirala. S~tema grafoma potrdimo, da morajo 
biti, za delovanje laserja, izgube manjše od črpanja.
\begin{figure}    
    \centering
    \resizebox{.48\linewidth}{!}{\input{graf8.tex}}
    \resizebox{0.48\linewidth}{!}{\input{graf8b.tex}}
    \caption{Na levem grafu narisano spreminjanje količine fotonov in atomov za $r<p$ in 
    $r=0$. Na desnem grafu je narisan isti časovni potek v~odvisnosti ene količine od druge}
    \label{slika8}
\end{figure}

\begin{figure}    
    \centering
    \resizebox{.49\linewidth}{!}{\input{graf9a.tex}}
    \resizebox{.49\linewidth}{!}{\input{graf9b.tex}}
    \caption{Na levem grafu narisano spreminjanje količine fotonov in atomov za različne
    $r$ in $p$. Na desnem grafu je narisan isti časovni potek v~odvisnosti ene količine od 
    druge.}
    \label{slika9}
\end{figure}

\subsection{Zastojni točki}
Poglejmo kje ima sistem~\ref{laser} ničelna odvoda:
\begin{align*}
    0 &= \frac{F}{p} \left(A - 1\right) \\
    0 &= r - pA\left(F + 1\right)
\end{align*}
in dobimo dve točki $T_1 = (r / p, 0) $ ter $T_2 = (1, r/p - 1)$, obe odvisni od razmerja
med črpanjem in izgubami.
\subsubsection*{Točka $A=\frac{r}{p},\; F=0$:}
Zapišemo Jakobijevo matriko sistema v tej točki: 
\begin{equation}
    \begin{bmatrix}
	-p & -r \\
	0 & \frac{r-p}{p^2}
    \end{bmatrix}.
\end{equation}
Ena lastna vrednost matrike je $\lambda_1=-p$ in je zmeraj negativna, zato je vse odvisno
od druge lastne vrednosti $\lambda_2= (r-p)/p^2$:
\begin{enumerate}
    \item $r>p \Rightarrow \operatorname{Re}(\lambda_2)>0$: sistem ima v~taki točko 
	nasprotno predznačeni lastni vrednosti, zato je ta ekstrem sedlo, 
    \item $r=p \Rightarrow \operatorname{Re}(\lambda_2)=0$ taka točka je stabilna,
    \item $r<p \Rightarrow \operatorname{Re}(\lambda_2)<0$: obe lastni vrednosti sta 
	negativni, zato je taka točka asimptotsko stabilna.
\end{enumerate}
Za različne primere obnašanja lahko vidimo na grafih~\ref{slika8},~\ref{slika9}, ko se 
količine sesedejo v~točko $(r/p,0)$, primera, ko sta lastni vrednosti nasprotno 
predznačeni niti ne opazimo in se sistem sprosti v~drugo točko (ki si jo bomo ogledali 
v~naslednjem razdelku). To tudi pričakujemo, saj se bo za primere, ko so izgube večje (ali)
enake črpanju, sistem ugasne. 

\subsubsection{Točka $A = 1, \; F=\frac{r}{p} - 1$:}
Jakobijeva matrika sistema v~tej točki:
\begin{equation}
    \begin{bmatrix}
	-r & -p \\
	\frac{r-p}{p^2} & 0
    \end{bmatrix},
\end{equation}
ima dve lastni vrednosti:
\begin{equation}
    \lambda_{1,2} = -\frac{r}{2} \pm \frac{1}{2} \sqrt{r^2 - 4\frac{r-p}{p^2}}. 
\end{equation}
Obnašanje je tukaj odvisno od vrednosti, ki stoji pod korenom $D=r^2 - 4\frac{r-p}{p^2}$:
\begin{equation}
    D=0 \quad \Rightarrow \quad r_{1,2} = \frac{2}{p} (1 \pm \sqrt{1-p^2}),
\end{equation}
seveda hočemo, da je koeficient črpanja realen in tako imamo zahtevo $p^2<1$. Poglejmo
si kako se obnaša sistem za različne vrednosti $D$:
\begin{enumerate}
    \item $D>0 \Rightarrow r>r_1 \, \vee \, 0<r<r_2$: 
	\begin{enumerate}
	    \item za $r > \sqrt{D}$ bosta obe lastni vrednosti negativni in točka bo 
		asimptotsko stabilna,
	    \item za $r < \sqrt{D}$ bosta lastni vrednosti nasprotno predznačeni
		in v~taki točki bo sedlo.
	\end{enumerate}
    \item $D=0 \Rightarrow p=1 \text{ in } r=2$: lastni vrednosti $\lambda_{1,2} = -1$
	sta negativni in je tako tudi ta točka asimptotsko stabilna.
    \item $D<0 \Rightarrow r \in (\frac{2}{p} (1-\sqrt{1-p^2}), \frac{2}{p} 
	(1-\sqrt{1+p^2}))$: lastni vrednosti sta kompleksno konjugirani zato je ta točka
	center spiralnih orbit (v~prvem razdelku smo imeli koncentrične, ker ni bilo 
	realnega dela). 
	Ta fokus bo stabilen, saj je vrednost $\operatorname{Re}(\lambda)=-2r$ negativna.
\end{enumerate}
Omenjeno lahko opazimo na grafu~\ref{slika11}. Graf je bil narisan za $p=0,5$ in take
vrednosti $r$, da smo zavzeli vse tri prej omenjene primere. Vidimo, da sistem za 
$r=0.5<\sqrt{D}$ nismo prešli v~točko $T_2$, saj je ta nestabilna za te parametre, in se je
raje sesedel v~točko $T_1$. Za vrednost $r=16>2/p (1+\sqrt{1-p^2})$ je diskriminanta $D$
pozitivna in sistem ne oscilira, ampak se eksponentno približa limitni vrednosti $T_2$.\\
Na grafu~\ref{slika12} so prikazane oscilacije pri $r=1$ in $p=0.5$ za različne začetne
vrednosti $A_0$. Z~večanjem slednjih, višamo maksimum vrednosti fotonov, kar ni 
presenetljivo; več atomov kot imamo na voljo, več fotonov bo nastalo.

\begin{figure}    
    \resizebox{0.48\linewidth}{!}{\input{graf11a.tex}}
    \resizebox{0.48\linewidth}{!}{\input{graf11b.tex}}
    \caption{Na levem grafu je narisan potek populacije fotonov in atomov za $p=0.5$ in 
    za različne $r$, iz začetne vrednosti $A_0=0,5$, $F_0=2$. Na desnem je ustrezen fazni 
    diagram.}
    \label{slika11}
\end{figure}
\begin{figure}    
    \resizebox{0.48\linewidth}{!}{\input{graf12a.tex}}
    \resizebox{0.48\linewidth}{!}{\input{graf12b.tex}}
    \caption{Na levem grafu je narisan potek populacije fotonov in atomov za $F_0=2$ in 
    za različne $A_0$, pri čemer je $r=1$ in $p=0.5$. Na desnem je ustrezen fazni diagram.}
    \label{slika12}
\end{figure}

\subsection{Dušeno nihanje}
\begin{figure}    
    \resizebox{0.48\linewidth}{!}{\input{graf13a.tex}}
    \resizebox{0.48\linewidth}{!}{\input{graf13b.tex}}
    \caption{Na grafih je narisan potek populacije fotonov in atomov za $F_0=1.5$ in 
    za $A_0=2$ za različne $r,p$, pri čemer je razmerje $r/p=2$. Na desnem je ustrezen 
    fazni diagram.}
    \label{slika13}
\end{figure}

Na grafu~\ref{slika13} so primeri za konstantno razmerje $r/p$, le spreminjamo njihove 
vrednosti. Vidimo, da z~manjšanjem obeh parametrov, postajajo spirale vedno večje -- sistem
potrebuje več časa da se sesede v~ravnovesje. To je posledica tega, da je realni del
lastnih vrednosti čedalje manjši; manjšamo dušenje sistema. Hkrati se manjša tudi imaginarni
del in zgleda, da se s~tem krajšajo periode oscilacij. \\ %Preverimo to še računsko: 
Da dobimo relaksacijski čas $\tau$ lahko maksimume nihajev interpoliramo z~eksponentno
funkcijo -- prikazano na levem grafu~\ref{slika10}. To ponovimo za več različnih 
koeficientov črpanja in dobimo eksponentno odvisnost, na desnem grafu ~\ref{slika10}.
To odvisnost lahko tudi opišemo z~eksponentno funkcijo, za katero dobimo rešitev 
$$\tau(r) = 9,3 \mathrm{e}^{-3 r} + 0,9.$$

\begin{figure}    
    \centering
    \resizebox{0.48\linewidth}{!}{\input{graf10a.tex}}
    \resizebox{0.48\linewidth}{!}{\input{graf10b.tex}}
    \caption{Za primer $p=0.125$ in $r=0.25$ je narisan $F(t)$, s~\emph{fittom} maksimumov.}
    \label{slika10}
\end{figure}


%\begin{equation}
%    \omega = \frac{2\pi}{\sqrt{\lambda_1 \lambda_2}} = 2\pi (\frac{r-p}{p^2})^{-1}
%\end{equation}

\pagebreak

%----------------------------------------------------------------------------------------
%	PROBLEM 3
%----------------------------------------------------------------------------------------
\section{Model epidemije}
Model širjenja bolezni opišemo s~sistemom
\begin{align}\label{epidemija}
    \frac{\mathrm{d}D}{\mathrm{d}t} &= -\alpha D B, \\
    \frac{\mathrm{d}B}{\mathrm{d}t} &= \alpha D B - \beta B. \\
    \frac{\mathrm{d}I}{\mathrm{d}t} &= \beta B,
\end{align}
pri čemer gledamo neko konstantno populacijo $D$ predstavlja delež zdravih in dovzetnih na
okužbo, $B$ delež obolelih, $I$ pa delež imunih v~populaciji, torej tistih ki se ne morejo
(več) okužiti. Parameter $\alpha$ opisuje hitrost širjenja bolezni, $\beta$ pa zdravljenje
obolelih. Omejimo se še na konstantno populacijo: $D + B + I = N$.\\
Na zgornjih grafih~\ref{slika14} je prikazano obnašanje, ko nimamo zdravljenja za dve 
različni hitrosti obolevanja. Na spodnjih imamo še nek delež populacije, imune na bolezen
in vidimo, da to ne vpliva na hitrost okužbe.\\
\begin{figure}    
    \resizebox{0.48\linewidth}{!}{\input{graf14a.tex}}
    \resizebox{0.48\linewidth}{!}{\input{graf14b.tex}}\\
    \resizebox{0.48\linewidth}{!}{\input{graf14c.tex}}
    \resizebox{0.48\linewidth}{!}{\input{graf14d.tex}}
    \caption{Grafa sta narisana za primera, ko nimamo zdravljenja $\beta=0$ in za dve 
    različni hitrosti zbolevanja $\alpha$. Začetna vrednost je bila za zgornja primera
    $B=1\permil, \, I=0$ in $D=N-B$, ter $B=1\permil, \, I=1\%$ in $D=N-B-I$ za spodnja.}
    \label{slika14}
\end{figure}
Na grafih~\ref{slika5} vidimo kakšna je razlika, če na začetku že cepimo delež populacije.
Če delež cepljenja dovolj povečamo, lahko dosežemo da bolezen izumre, brez, da bi vsi 
dovzetni bolezen preživeli, kar vidimo na grafu~\ref{slika16}. \\
\begin{figure}    
    \resizebox{0.48\linewidth}{!}{\input{graf15a.tex}}
    \resizebox{0.48\linewidth}{!}{\input{graf15b.tex}}\\
    \caption{Grafa sta narisana za primera z~različnima začetnima vrednostma imunih na 
    okužbo; na levi je $I_0=0$ na desni pa $I_0=1\%$.}
    \label{slika15}
\end{figure}
\begin{figure}    
    \resizebox{0.48\linewidth}{!}{\input{graf16b.tex}}
    \resizebox{0.48\linewidth}{!}{\input{graf16a.tex}}\\
    \caption{Grafa sta narisana za primera z~$I_0=1\%$ in za dva različna $\beta$.}
    \label{slika16}
\end{figure}

Poglejmo si še kakšen vpliv ima hitrost zdravljenja $\beta$ na število 
obolelih~\ref{slika17}.  Vidimo, da se z~večanjem parametra $\beta$ (pri enakih zaletnih 
pogojih in $\alpha=0.5$) maksimalno število obolelih zmanjšuje in se premika proti daljšim
časom. 
Ko spreminjamo začetno število imunih, vidimo enak trend; z~večanjem $I_0$ se 
maksimumi nižajo in premikajo h~kasnejšim časom.
\begin{figure}    
    \resizebox{0.48\linewidth}{!}{\input{graf17a.tex}}
    \resizebox{0.48\linewidth}{!}{\input{graf17b.tex}}\\
    \caption{Na levi je prikazano spreminjanje $D$ za različne vrednosti parametra $\beta$, 
    ko je $\alpha=0.5$ in začetne vrednosti $I_0=1\%$ in $D_0=1\permil$. Na desni je 
    graf $D(I_0)$, kjer sta parametra $\alpha=0.5, \,\beta=0.05$, spreminjamo pa
    začetno vrednost $I_0$.}
    \label{slika17}
\end{figure}

\subsection{Stopnje bolezni}
Bolezen lahko razdelimo v~več stadijev, tako povečamo sistem diferencialnih enačb:
\begin{align}\label{epidemija}
    \dot{D} &= -\alpha D B, \\
    \dot{B_1} &= \alpha D B - \beta_1 B_1. \\
    \dot{B_2} &= \beta_1 B_1 - \beta_2 B_2. \\
    &\vdots \\
    \dot{I} &= \beta_n B_n,
\end{align}
kjer so $B=\sum_i B_i$ vsota vseh obolelih, $\beta_i$ pa hitrosti zdravljenja v~posameznem 
stadiju. Z~grafov~\ref{slika18} lahko vidimo, da se vrh epidemije pomika h~krajšim časom,
oblika funkcije pa postaja vedno bolj simetrična (vedno bolj Gaussova).
\begin{figure}    
    \centering
    \resizebox{0.48\linewidth}{!}{\input{graf18a.tex}}
    \resizebox{0.48\linewidth}{!}{\input{graf18b.tex}}
    \resizebox{0.48\linewidth}{!}{\input{graf18c.tex}}
    \caption{Na grafih je prikazano spreminjanje števila obolelih, če jih razdelimo v~več
    razredov.}
    \label{slika18}
\end{figure}
\end{document}
