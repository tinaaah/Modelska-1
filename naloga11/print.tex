\documentclass[a4paper,pdftex,10pt]{article}
\usepackage[a4paper]{geometry}
\usepackage[utf8]{inputenc}
\usepackage[T1]{fontenc} 
\usepackage[english,slovene]{babel} 
\usepackage{amsmath,amsfonts,amsthm,amssymb,mathrsfs,empheq} % Math packages
\usepackage{mathtools}
\usepackage{bbold}
\usepackage{dsfont}
\usepackage{wrapfig}
\usepackage[pdftex]{graphicx}
%\usepackage{makeidx}
\usepackage{url}
\usepackage{caption}
\usepackage{subcaption}
\usepackage{tabularx}
\usepackage{float}
\usepackage{siunitx}

\usepackage{standalone} %Da uporabi standalone tikze
\usepackage{tikz}
\usepackage{pgf}  
\usepackage[version=3]{mhchem} %kemija
\usepackage{enumitem} % Da lahko spremeniš način oštevilčevanja

\usetikzlibrary{arrows,automata}
\usetikzlibrary{positioning}

\renewcommand{\vec}[1]{\boldsymbol{\mathbf{#1}}}                                        
\newcommand{\ihat}[0]{\boldsymbol{\mathbf{\oldhat{\textbf{\i}}}}} % pokončna j in i (j i n i
\newcommand{\iu}{{i\mkern1mu}}	    %imaginarno število
\DeclarePairedDelimiterX{\norm}[1]{\lVert}{\rVert}{#1} %norma

\usepackage{fancyhdr} % Custom headers and footers
\pagestyle{fancyplain} % Makes all pages in the document conform to the custom headers and footers
\fancyhead{} % No page header - if you want one, create it in the same way as the footers below
\fancyfoot[L]{} % Empty left footer
\fancyfoot[C]{} % Empty center footer
\fancyfoot[R]{\thepage} % Page numbering for right footer
\renewcommand{\headrulewidth}{0pt} % Remove header underlines
\renewcommand{\footrulewidth}{0pt} % Remove footer underlines
\setlength{\headheight}{13.6pt} % Customize the height of the header

%\numberwithin{equation}{section} % Number equations within sections (i.e. 1.1, 1.2, 2.1, 2.2 instead of 1, 2, 3, 4)
\numberwithin{figure}{section} % Number figures within sections (i.e. 1.1, 1.2, 2.1, 2.2 instead of 1, 2, 3, 4)
%\numberwithin{table}{section} % Number tables within sections (i.e. 1.1, 1.2, 2.1, 2.2 instead of 1, 2, 3, 4)

\setlength\parindent{0pt} % Removes all indentation from paragraphs - comment this line for an assignment with lots of text

%----------------------------------------------------------------------------------------
%	TITLE SECTION
%----------------------------------------------------------------------------------------

\newcommand{\horrule}[1]{\rule{\linewidth}{#1}} % Create horizontal rule command with 1 argument of height

\title{	
\normalfont \normalsize 
\textsc{Modelska analiza 1} \\ [25pt] % Your university, school and/or department name(s)
%\horrule{0.2pt} \\[0.4cm] % Thin top horizontal rule
\huge 11. naloga - Optimalno filtriranje\\ % The assignment title
%\horrule{0.2pt} \\[0.5cm] % Thick bottom horizontal rule
}

\author{Tina Klobas} % Your name

\date{\normalsize\today} % Today's date or a custom date

\begin{document}

\maketitle % Print the title

\section{Opis problema}
Naj bo vektor $\vec{x}_t$ vektor stanja ob času $t$, $P_t$ pa pripadajoča kovariančna
matrika, ki opisuje njegovo statistično negotovost. Potrebujemo še začetno stanje 
$\vec{x}_0^+$ ter kovarianco $P_0^+$, ki ju dobimo iz meritve. \\
Potek Kalmanovega filtra je predstavljen na sledečem diagramu:
\begin{figure}[H]
  \centering
  \includestandalone[width=0.8\linewidth]{flowchart} % without the `.tex` extension
\end{figure}

Vidimo, da je sestavljen iz treh pomembnih korakov:
\begin{enumerate}
    \item Napovemo novo stanje ob času $t$; $\vec{x}_{t}^-$ in $P_{t}^-$. Matrika $F_t$ je
	prehodna matrika sistema, ki jo dobimo s~poznavanjem dinamike sistema. 
	Iz gibalnih enačb dobimo tudi kontrolni vektor $\vec{c}_t = B_t \vec{u}_t$. 
	Kovariančna matrika $Q_t$ pa je posledica zašumljenosti podatkov.
    \item Izračunamo \emph{Kalmanov ojačevalni faktor}, ki določa s~kolikšno utežjo
	v~naslednjem koraku nova meritev prispeva k~popravku. V~enačbi nastopata matriki
	$H_t$ -- senzorsko okno (velikost odvisna od števila senzorjev in dimenzije 
	$\vec{x}$ in $R_t$, ki je kovariančna matrika šuma meritve.
    \item Izboljšamo napoved stanja. $\vec{z}_t$, ki nastopa v~enačbah je vektor meritev
	-- podatkov, ki jih želimo obdelati.
\end{enumerate}
%----------------------------------------------------------------------------------------
%	PROBLEM 1
%----------------------------------------------------------------------------------------
\section{Rekonstrukcija lokacije z GPS}
Iz datoteke \path{kalman_cartesian_data.dat} dobimo podatke ob času $t$, zašumljene meritve 
položaja $\vec{r}_t = (x_t, y_t)$ hitrosti $\vec{v} = (v_{x,t},v_{y,t})$ ter pospeškov 
$\vec{a} = (a_{x,t}, a_{y,t})$. \\
V~našem primeru je vektor stanja $\vec{x} = (x,y,v_x,v_y)$, kontrolni vektor $\vec{c}=
(0,0,a_x \Delta t, a_y \Delta t)$, prehodna matrika je konstantna in je oblike:
$$ F = 
\begin{bmatrix}
    \mathbb{1}_{2\times 2} & \mathbb{1}_{2\times 2} \Delta t \\
    \mathbb{0}_{2\times 2} & \mathbb{1}_{2\times 2} 
\end{bmatrix}, $$
šum časovne evolucije sledi iz napake pospeška, $Q_n = \mathrm{diag}(0,0,\sigma_a^2 
\Delta t, \sigma_a^2 \Delta t)$, šum meritev pa iz napak GPS podatkov ter hitrosti,
$R_n = \mathrm{diag}(\sigma_{xy}^2, \sigma_{xy}^2, \sigma_v^2, \sigma_v^2)$. \\
Podatki so vzorčeni vsakih $\Delta t =1,783 \, \si{s}$. Za pospeške in GPS podatke sta znani
absolutni napaki $\sigma_{xy} = \SI{25}{m}$ in $\sigma_a = \SI{0.05}{m.s^{-2}}$, za 
hitrost pa poznamo relativno napako: $\sigma_v = 0.01 \|\vec{v}\|$, pri čemer napako vseeno
navzdol omejimo na $\SI{1}{km.h^{-1}}$. \\
Napovedati želimo trajektorijo vožnje in časovno odvisnost hitrosti, v~primeru, da
zmanjšamo gostoto vzorčenja -- imamo na voljo le vsako peto meritev hitrosti in vsako
deseto meritev lokacije. \\
Zaradi zadnjega napotka, da ne dobimo na vsakem koraku nove meritve, moramo uporabiti
tri različne okenske matrike $H$:
\begin{enumerate}
    \item ne dobimo ne meritve $\vec{v}$ ne meritve $\vec{r}$: $H_1 = 
	\mathbb{0}_{4\times4},$ 
    \item dobimo le meritev $\vec{v}$: $H_2 = 
	\begin{bmatrix} 
	    \mathbb{0}_{2\times2} & \mathbb{0}_{2\times2}\\ 
	    \mathbb{0}_{2\times2} & \mathbb{1}_{2\times2} 
	\end{bmatrix},$ 
    \item dobimo obe meritvi: $ H_3 = \mathbb{1}_{4\times4}.$
\end{enumerate}

\subsection{Rezultati}
Rezultate uporabe Kalmanovega filtra na naših podatkih, lahko primerjamo s~priloženo
datoteko točnih vrednosti \path{kalman_cartesian_kontrola.dat}.
\subsubsection{Točnost rezultatov}
Na grafu~\ref{slika1} je prikazana dejanska pot in tista, ki smo jo dobili s~Kalmanovim 
filtrom. Na približani sliki vidimo, da s~Kalmanovim filtrom dobimo neko odstopanje, ki 
si ga oglejmo še na naslednjih grafih. Na~sliki~\ref{slika2} so prikazani odmiki $x$ 
koordinate pozicije od dejanskih vrednosti. Če primerjamo črn del z~barvnima vidimo, da se 
s~Kalmanovim filtrom res zelo izboljša natančnost meritve. S~tem ko določamo katere meritve 
bomo na posamezne koraku vzeli (spreminjamo okensko funkcijo) tudi spreminjamo natančnost 
določanja lokacije, kar je tudi na tem grafu dobro opazno. \\
\begin{figure}[H] 
    \centering
    \resizebox{.49\linewidth}{!}{\input{graf1a.tex}}
    \resizebox{.49\linewidth}{!}{\input{graf1b.tex}}
    \caption{Dejanska pot vozila in njena rekonstrukcija s~pomočjo Kalmanovega filtra. Na desni je prikazan izsek poti, kjer se bolje vidi odstopanje.}
    \label{slika1}
\end{figure}
\begin{figure}[H]
    \centering
    \resizebox{.49\linewidth}{!}{\input{graf2a.tex}}
    \resizebox{.49\linewidth}{!}{\input{graf2b.tex}}
    \caption{Na grafu je narisano odstopanje pozicije (levo $x$, desno $y$) od dejanskih 
    podatkov in sicer; s~črno so narisani odmiki zašumljene meritve, z~rdečo je narisan 
    Kalmanov filter, ki smo ga naredili na vseh meritvah, z~zeleno pa Kalmanov filter, ko 
    smo meritve poti jemali na vsakem $10.$ koraku, meritve hitrosti pa na $5.$}
    \label{slika2}
\end{figure}
Naslednja grafa~\ref{slika2b} prikazujeta odstopanje hitrosti pri obeh variantah 
Kalmanovega filtra. Vidimo, da je relativna napaka pri obeh komponentah zelo majhna, večji
vrhovi pa nastanejo pri naglih zavojih na poti.
\begin{figure}[H]
    \centering
    \resizebox{.49\linewidth}{!}{\input{graf2c.tex}}
    \resizebox{.49\linewidth}{!}{\input{graf2d.tex}}
    \caption{Na grafu je narisano odstopanje hitrosti (levo $v_x$, desno $v_y$) od 
    dejanskih podatkov in sicer; z~rdečo Kalmanov filter, ki smo ga naredili na vseh 
    meritvah, z~zeleno pa Kalmanov filter, kjer smo meritve poti jemali na vsakem $10.$ 
    koraku, meritve hitrosti pa na $5.$}
    \label{slika2b}
\end{figure}

Z~matriko $P$:
\begin{equation}
    P = 
    \begin{bmatrix}
	\sigma_x^2 &  \sigma_{x,y} & \sigma_{x, \dot{x}} & \sigma_{x, \dot{y}}\\
	\sigma_{y,x} & \sigma_y^2 & \sigma_{y, \dot{x}} & \sigma_{y, \dot{y}}\\
	\sigma_{\dot{x},x} & \sigma_{\dot{x},y} 
			   & \sigma_{\dot{x}}^2 & \sigma_{\dot{x}, \dot{y}}\\
	\sigma_{\dot{y},x} & \sigma_{\dot{y},y} 
			   & \sigma_{\dot{y}, \dot{x}} & \sigma_{\dot{y}}^2
    \end{bmatrix}
\end{equation}
lahko določimo intervale zaupanja, oziroma elipsoide zaupanja. Izračunamo
njene lastne vrednosti $\lambda$ in ustrezne lastne vektorje $\vec{q}$, kar nam pomaga
pri izračunu polosi: $a_i=\sqrt{s \lambda_i}$, in kot nagnjenosti elipse $\alpha_{ij} = 
\arctan \frac{\| \vec{q_i} \|}{\| \vec{q_j} \|}$.\\
Na grafu~\ref{slika3} sta narisana oba primera -- na desni, kjer je prikazana rešitev za 
primer, ko imamo na voljo le vsako $10.$ meritev koordinate $x$ vidimo, kako je potek 
radija intervalov nazobčan in če malo bolje pogledamo, lahko opazimo da je ta žagasta 
oblika dvojna; radij zaupanja se veča, dokler ne dobimo zopet nove meritve hitrosti in 
nato še večji preskok, ko dobimo novo meritev koordinate. Nekje pri $t=600$ je viden malo 
večji preskok -- kar je na grafu~\ref{slika1} ravno točka zavoja v~bližini $(1000,-2000)$. 
Na tem mestu pade natančnost meritve, s~čimer se poveča interval zaupanja. Podobna 
preskoka sta tudi okoli $t=1400$ in $t=2100$, obakrat smo v~bližini večjih zavojev.\\
\begin{figure}[H]
    \centering
    \resizebox{.45\linewidth}{!}{\input{graf3a.tex}}
    \resizebox{.45\linewidth}{!}{\input{graf3b.tex}}
    \caption{Odmik Kalmanove rešitve od prave vrednosti lokacije $x$, narisan med intervali 
    zaupanja.  Na levi strani smo vzeli vse meritve, na desni pa vsako $5.$ meritev 
    hitrosti in vsako $10.$ meritev lokacije.}
    \label{slika3}
\end{figure}
Podobno lahko naredimo tudi za meritev hitrosti $v_x$ in dobimo graf~\ref{slika4}, kjer 
opazimo, da pa naši rezultati s~filtriranjem niso nikoli padli z~izbranih intervalov 
zaupanja. Tudi tukaj vidimo enake preskoke, kot smo jih prej na grafih za koordinate.
\begin{figure}[H]
    \centering
    \resizebox{.45\linewidth}{!}{\input{graf4a.tex}}
    \resizebox{.45\linewidth}{!}{\input{graf4b.tex}}
    \caption{Odmik Kalmanove rešitve od prave vrednosti hitrosti $v_x$, narisan med 
    intervali zaupanja.  Na levi strani smo vzeli vse meritve, na desni pa vsako $5.$ 
    meritev hitrosti in vsako $10.$ meritev lokacije.}
    \label{slika4}
\end{figure}
Naslednja stvar, ki jo lahko tudi opazujemo so residuali $\| \vec{z}_t - H_t \vec{x}_t^-\|$.
Njihovo spreminjanje s~časom je prikazano na grafu~\ref{slika5}, vendar nam k~razumevanju
sistema nič ne doprinese; rezultati so precej razpršeni, povprečno vrednost lahko 
izračunamo in jo tudi narišemo, vendar tega ob taki razpršeni rešitvi, ni koristno početi.
\begin{figure}[H]
    \centering
    \resizebox{.45\linewidth}{!}{% GNUPLOT: LaTeX picture with Postscript
\begingroup
  \makeatletter
  \providecommand\color[2][]{%
    \GenericError{(gnuplot) \space\space\space\@spaces}{%
      Package color not loaded in conjunction with
      terminal option `colourtext'%
    }{See the gnuplot documentation for explanation.%
    }{Either use 'blacktext' in gnuplot or load the package
      color.sty in LaTeX.}%
    \renewcommand\color[2][]{}%
  }%
  \providecommand\includegraphics[2][]{%
    \GenericError{(gnuplot) \space\space\space\@spaces}{%
      Package graphicx or graphics not loaded%
    }{See the gnuplot documentation for explanation.%
    }{The gnuplot epslatex terminal needs graphicx.sty or graphics.sty.}%
    \renewcommand\includegraphics[2][]{}%
  }%
  \providecommand\rotatebox[2]{#2}%
  \@ifundefined{ifGPcolor}{%
    \newif\ifGPcolor
    \GPcolortrue
  }{}%
  \@ifundefined{ifGPblacktext}{%
    \newif\ifGPblacktext
    \GPblacktexttrue
  }{}%
  % define a \g@addto@macro without @ in the name:
  \let\gplgaddtomacro\g@addto@macro
  % define empty templates for all commands taking text:
  \gdef\gplbacktext{}%
  \gdef\gplfronttext{}%
  \makeatother
  \ifGPblacktext
    % no textcolor at all
    \def\colorrgb#1{}%
    \def\colorgray#1{}%
  \else
    % gray or color?
    \ifGPcolor
      \def\colorrgb#1{\color[rgb]{#1}}%
      \def\colorgray#1{\color[gray]{#1}}%
      \expandafter\def\csname LTw\endcsname{\color{white}}%
      \expandafter\def\csname LTb\endcsname{\color{black}}%
      \expandafter\def\csname LTa\endcsname{\color{black}}%
      \expandafter\def\csname LT0\endcsname{\color[rgb]{1,0,0}}%
      \expandafter\def\csname LT1\endcsname{\color[rgb]{0,1,0}}%
      \expandafter\def\csname LT2\endcsname{\color[rgb]{0,0,1}}%
      \expandafter\def\csname LT3\endcsname{\color[rgb]{1,0,1}}%
      \expandafter\def\csname LT4\endcsname{\color[rgb]{0,1,1}}%
      \expandafter\def\csname LT5\endcsname{\color[rgb]{1,1,0}}%
      \expandafter\def\csname LT6\endcsname{\color[rgb]{0,0,0}}%
      \expandafter\def\csname LT7\endcsname{\color[rgb]{1,0.3,0}}%
      \expandafter\def\csname LT8\endcsname{\color[rgb]{0.5,0.5,0.5}}%
    \else
      % gray
      \def\colorrgb#1{\color{black}}%
      \def\colorgray#1{\color[gray]{#1}}%
      \expandafter\def\csname LTw\endcsname{\color{white}}%
      \expandafter\def\csname LTb\endcsname{\color{black}}%
      \expandafter\def\csname LTa\endcsname{\color{black}}%
      \expandafter\def\csname LT0\endcsname{\color{black}}%
      \expandafter\def\csname LT1\endcsname{\color{black}}%
      \expandafter\def\csname LT2\endcsname{\color{black}}%
      \expandafter\def\csname LT3\endcsname{\color{black}}%
      \expandafter\def\csname LT4\endcsname{\color{black}}%
      \expandafter\def\csname LT5\endcsname{\color{black}}%
      \expandafter\def\csname LT6\endcsname{\color{black}}%
      \expandafter\def\csname LT7\endcsname{\color{black}}%
      \expandafter\def\csname LT8\endcsname{\color{black}}%
    \fi
  \fi
    \setlength{\unitlength}{0.0500bp}%
    \ifx\gptboxheight\undefined%
      \newlength{\gptboxheight}%
      \newlength{\gptboxwidth}%
      \newsavebox{\gptboxtext}%
    \fi%
    \setlength{\fboxrule}{0.5pt}%
    \setlength{\fboxsep}{1pt}%
\begin{picture}(7180.00,4300.00)%
    \gplgaddtomacro\gplbacktext{%
      \csname LTb\endcsname%%
      \put(504,408){\makebox(0,0)[r]{\strut{}$0$}}%
      \csname LTb\endcsname%%
      \put(504,1023){\makebox(0,0)[r]{\strut{}$20$}}%
      \csname LTb\endcsname%%
      \put(504,1637){\makebox(0,0)[r]{\strut{}$40$}}%
      \csname LTb\endcsname%%
      \put(504,2252){\makebox(0,0)[r]{\strut{}$60$}}%
      \csname LTb\endcsname%%
      \put(504,2866){\makebox(0,0)[r]{\strut{}$80$}}%
      \csname LTb\endcsname%%
      \put(504,3481){\makebox(0,0)[r]{\strut{}$100$}}%
      \csname LTb\endcsname%%
      \put(504,4095){\makebox(0,0)[r]{\strut{}$120$}}%
      \csname LTb\endcsname%%
      \put(616,204){\makebox(0,0){\strut{}$0$}}%
      \csname LTb\endcsname%%
      \put(1861,204){\makebox(0,0){\strut{}$500$}}%
      \csname LTb\endcsname%%
      \put(3107,204){\makebox(0,0){\strut{}$1000$}}%
      \csname LTb\endcsname%%
      \put(4352,204){\makebox(0,0){\strut{}$1500$}}%
      \csname LTb\endcsname%%
      \put(5598,204){\makebox(0,0){\strut{}$2000$}}%
      \csname LTb\endcsname%%
      \put(6843,204){\makebox(0,0){\strut{}$2500$}}%
    }%
    \gplgaddtomacro\gplfronttext{%
      \csname LTb\endcsname%%
      \put(3931,3688){\makebox(0,0)[l]{\strut{}$B=$ vsaka $5./10.$ meritev}}%
      \csname LTb\endcsname%%
      \put(3931,3892){\makebox(0,0)[l]{\strut{}$A=$ vsaka meritev}}%
      \csname LTb\endcsname%%
      \put(3730,1637){\makebox(0,0)[l]{\strut{}$\mu_A$}}%
      \csname LTb\endcsname%%
      \put(3730,1483){\makebox(0,0)[l]{\strut{}$\mu_B$}}%
    }%
    \gplbacktext
    \put(0,0){\includegraphics{graf5}}%
    \gplfronttext
  \end{picture}%
\endgroup
}
    \caption{Potek $2.$ norme residuala v~primeru, ko imamo na voljo vse meritve in,
    ko imamo samo vsako $5.$ meritev hitrosti in vsako $10.$ meritev lokacije.}
    \label{slika5}
\end{figure}
Na naslednjem grafu~\ref{slika13} imamo prikazano normo kovariančne matrike $P$ Vidimo, da 
ko zmanjšamo pogostost vzorčenja povečamo negotovost stanja, poleg tega pa tudi tukaj 
opazimo to žagasto obliko -- $||P||$ se zmanjša ob novi meritvi. 
\begin{figure}[H]
    \centering
    \resizebox{.49\linewidth}{!}{\input{graf13a.tex}}
    \caption{Norma kovariančne matrike $P$, kjer je z~rdečo narisan filter pri katerem smo 
    dobivali meritve na vsakem koraku, z~rdečo pa smo vzeli lokacijo na vsakem $5.$, 
    hitrost pa na vsakem $5.$}
    \label{slika13}
\end{figure}


%----------------------------------------------------------------------------------------
%	PROBLEM 2
%----------------------------------------------------------------------------------------
\section{Merjenje lokacije s~telefonom}
Problem je enak kot prej, le da imamo zdaj ob vsakem koraku dve možnosti:
\begin{enumerate}[label=(\alph*)]
    \item lahko merimo le lokacijo in pospeške,
    \item ali pa lahko merimo le hitrosti in pospeške.
\end{enumerate}
Tako kot v~prvem delu naloge, lahko tudi zdaj narišemo trajektorije. Če dobljeni 
graf~\ref{slika6} primerjamo s~tistim iz prvega dela~\ref{slika1}, na prvi pogled ocenimo,
da je metoda, kjer smo dobivali vse podatke, seveda bolj natančna kot ti dve in da je 
Kalmanov filter samo s~podatki o~legah bolj skače okoli prave vrednosti kot pa tisti, ki 
ima na voljo le hitrosti. \begin{figure}[H]
    \centering
    \resizebox{.49\linewidth}{!}{\input{graf6a.tex}}
    \resizebox{.49\linewidth}{!}{\input{graf6b.tex}}
    \caption{Primerjava poteka poti med~točnim rezultatom, Kalmanovim filtrom, ko imamo
    na voljo samo podatke o~pozicijah (z~rdečo) in ko imamo na voljo samo meritve hitrosti
    (s~svetlo modro). Na desni je približano območje okoli prvega krožnega zavoja, kjer se
    razlike med tremi potmi bolje razločijo.}
    \label{slika6}
\end{figure}
Preverimo bolj natančno na grafih napak -- na~\ref{slika7a} vidimo, da filter z~lokacijami 
res več skače okoli prave vrednosti, vendar je bližje pravi vrednosti, kar je veliko bolj 
očitno pri $y$ koordinato. Tukaj zgleda, da je filter v~začetku \emph{vrglo} s~tira in ker 
potem ni dobil nobenega podatka o~točni lokaciji, popravke pa delal le na podlagi hitrosti, 
imel približno konstantno odstopanje od točne lege.
\begin{figure}[H]
    \centering
    \resizebox{.49\linewidth}{!}{\input{graf7a.tex}}
    \resizebox{.49\linewidth}{!}{\input{graf7b.tex}}
    \caption{Odstopanje pozicije ($x$ na levi in $y$ na desni) od točnih podatkov. S~črno
    je narisana zašumljena meritev, z~rdečo meritev samo s~pospeški in lokacijo in z~modro
    samo s~pospeški in hitrostmi.}
    \label{slika7a}
\end{figure}
Grafi odstopanj hitrosti~\ref{slika7b} so podobni tistim iz prvega dela. Napaki za obe
varianti filtra imata enako obliko -- pri istih časih imamo maksimalne odstopanja hitrosti.
To se zgodi v~točkah ko imamo nagle spremembe hitrosti -- filter potrebuje korak da ujame
nazaj točno vrednost. Da se to zgodi v~istih točkah niti ni presenetljivo, saj je ta nagla
sprememba hitrosti v~smeri posledica zavojev poti.
\begin{figure}[H]
    \centering
    \resizebox{.49\linewidth}{!}{\input{graf7c.tex}}
    \resizebox{.49\linewidth}{!}{\input{graf7d.tex}}
    \caption{Odstopanje hitrosti ($v_x$ na levi in $v_y$ na desni) od točnih podatkov. 
    Z~rdečo je Kalmanov filter kjer dobivamo samo podatke o~pospeških in lokacijah in 
    z~modro samo o~pospeških in hitrostih.}
    \label{slika7b}
\end{figure}
Na grafu~\ref{slika12} norme kovariančne matrike pri prvi metodi kjer imamo na voljo 
meritvami leg, konstantna -- popravek iz matrike šuma $R$ je ves čas konstanten, saj 
nimamo člena z~relativno napako $\sigma_v$. Pri drugi metodi, pa je ta člen z~relativno
napako razlog za to, da negotovost stanja vedno večja.
\begin{figure}[H]
    \centering
    \resizebox{.49\linewidth}{!}{\input{graf12a.tex}}
    \caption{Norma matrike $P$ -- z~rdečo je narisan Kalmanov filter kjer dobivamo samo 
    podatke o~pospeških in lokacijah in z~modro samo o~pospeških in hitrostih.}
    \label{slika12}
\end{figure}

%----------------------------------------------------------------------------------------
%	PROBLEM 3
%----------------------------------------------------------------------------------------
\section{Orientacija vozila}
V~resnici nam akcelerometer podaja pospeške $\vec{a} = (a_t, a_r)$ glede na trenutno 
orientacijo vozila. Med kontrolnim vektorjem in nehomogenim delom dinamičnega modela
zato stoji še ena linearna preslikava $B_n:$ $\vec{c}_n = B_n \vec{u}_n.$ V~tem primeru
gre za ortogonalno transformacijo, definirano s~trenutno oceno hitrosti:
\begin{equation}
    \vec{u}_n = (\vec{0}, \vec{a_n} \Delta t), \quad
    B_n = 
    \begin{bmatrix}
	\mathbb{1}_{2\times 2} & \mathbb{0}_{2\times 2} \\
	\mathbb{0}_{2\times 2} & B_n^{vv}
    \end{bmatrix}, \quad
    B_n^{vv} = \frac{1}{|| \vec{v}_n ||}
    \begin{bmatrix}
	v_x & -v_y \\
	v_y & v_x
    \end{bmatrix}.
\end{equation}
Upoštevati moramo še negotovost trenutne ocene hitrosti:
\begin{equation}
    Q_n^{vv} = \Delta t^2 \left\{ \sigma_a^2 \mathbb{1}_{2\times 2} + 
	\frac{\vec{v}_n^{\perp} P_n^{vv} \vec{v}_n^{\perp}}{ ||\vec{v}_b||^4}
	\left[ (B_n^{vv} \vec{a}_n^{\perp}) \otimes (B_n^{vv} \vec{a}_n^{\perp}) 
	\right] \right\},
\end{equation}
kjer sta $Q_n^{vv}$ in $P_n^{vv}$ hitrostno-hitrostna bloka $4\times4$ kovariančnih matrik.
Vektorja $\vec{v}_n^{\perp}$ in $\vec{a}_n^{\perp}$ sta hitrost in pospešek, zavrtena 
v~pozitivni smeri za $\frac{\pi}{2}$. Pot bomo rekonstruirali iz danih podatkov 
\path{kalman_relative_data.dat}, katere stolpci so čas $t$, GPS položaj $(x,y)$ ter pospeška
$(a_t,a_r)$.\\
Na grafu~\ref{slika8a} je narisana dejanska pot in ta, ki smo jo dobili s~Kalmanovim 
filtrom. Na prvi pogled nas tudi tukaj najbolj vrže s~tira prvi zavoj. Na grafih 
hitrosti~\ref{slika8b} vidimo, da tudi hitrost v~tej točki najbolj zgrešimo.
\begin{figure}[H]
    \centering
    \resizebox{.49\linewidth}{!}{\input{graf8a.tex}}
    \resizebox{.49\linewidth}{!}{\input{graf8b.tex}}
    \caption{Pot vozila in njena rekonstrukcija s~Kalmanovim filtrom, če dobivamo podatke
    o~orientaciji vozila.}
    \label{slika8a}
\end{figure}
\begin{figure}[H]
    \centering
    \resizebox{.49\linewidth}{!}{% GNUPLOT: LaTeX picture with Postscript
\begingroup
  \makeatletter
  \providecommand\color[2][]{%
    \GenericError{(gnuplot) \space\space\space\@spaces}{%
      Package color not loaded in conjunction with
      terminal option `colourtext'%
    }{See the gnuplot documentation for explanation.%
    }{Either use 'blacktext' in gnuplot or load the package
      color.sty in LaTeX.}%
    \renewcommand\color[2][]{}%
  }%
  \providecommand\includegraphics[2][]{%
    \GenericError{(gnuplot) \space\space\space\@spaces}{%
      Package graphicx or graphics not loaded%
    }{See the gnuplot documentation for explanation.%
    }{The gnuplot epslatex terminal needs graphicx.sty or graphics.sty.}%
    \renewcommand\includegraphics[2][]{}%
  }%
  \providecommand\rotatebox[2]{#2}%
  \@ifundefined{ifGPcolor}{%
    \newif\ifGPcolor
    \GPcolortrue
  }{}%
  \@ifundefined{ifGPblacktext}{%
    \newif\ifGPblacktext
    \GPblacktexttrue
  }{}%
  % define a \g@addto@macro without @ in the name:
  \let\gplgaddtomacro\g@addto@macro
  % define empty templates for all commands taking text:
  \gdef\gplbacktext{}%
  \gdef\gplfronttext{}%
  \makeatother
  \ifGPblacktext
    % no textcolor at all
    \def\colorrgb#1{}%
    \def\colorgray#1{}%
  \else
    % gray or color?
    \ifGPcolor
      \def\colorrgb#1{\color[rgb]{#1}}%
      \def\colorgray#1{\color[gray]{#1}}%
      \expandafter\def\csname LTw\endcsname{\color{white}}%
      \expandafter\def\csname LTb\endcsname{\color{black}}%
      \expandafter\def\csname LTa\endcsname{\color{black}}%
      \expandafter\def\csname LT0\endcsname{\color[rgb]{1,0,0}}%
      \expandafter\def\csname LT1\endcsname{\color[rgb]{0,1,0}}%
      \expandafter\def\csname LT2\endcsname{\color[rgb]{0,0,1}}%
      \expandafter\def\csname LT3\endcsname{\color[rgb]{1,0,1}}%
      \expandafter\def\csname LT4\endcsname{\color[rgb]{0,1,1}}%
      \expandafter\def\csname LT5\endcsname{\color[rgb]{1,1,0}}%
      \expandafter\def\csname LT6\endcsname{\color[rgb]{0,0,0}}%
      \expandafter\def\csname LT7\endcsname{\color[rgb]{1,0.3,0}}%
      \expandafter\def\csname LT8\endcsname{\color[rgb]{0.5,0.5,0.5}}%
    \else
      % gray
      \def\colorrgb#1{\color{black}}%
      \def\colorgray#1{\color[gray]{#1}}%
      \expandafter\def\csname LTw\endcsname{\color{white}}%
      \expandafter\def\csname LTb\endcsname{\color{black}}%
      \expandafter\def\csname LTa\endcsname{\color{black}}%
      \expandafter\def\csname LT0\endcsname{\color{black}}%
      \expandafter\def\csname LT1\endcsname{\color{black}}%
      \expandafter\def\csname LT2\endcsname{\color{black}}%
      \expandafter\def\csname LT3\endcsname{\color{black}}%
      \expandafter\def\csname LT4\endcsname{\color{black}}%
      \expandafter\def\csname LT5\endcsname{\color{black}}%
      \expandafter\def\csname LT6\endcsname{\color{black}}%
      \expandafter\def\csname LT7\endcsname{\color{black}}%
      \expandafter\def\csname LT8\endcsname{\color{black}}%
    \fi
  \fi
    \setlength{\unitlength}{0.0500bp}%
    \ifx\gptboxheight\undefined%
      \newlength{\gptboxheight}%
      \newlength{\gptboxwidth}%
      \newsavebox{\gptboxtext}%
    \fi%
    \setlength{\fboxrule}{0.5pt}%
    \setlength{\fboxsep}{1pt}%
\begin{picture}(7200.00,4320.00)%
    \gplgaddtomacro\gplbacktext{%
      \csname LTb\endcsname%%
      \put(708,652){\makebox(0,0)[r]{\strut{}$-20$}}%
      \csname LTb\endcsname%%
      \put(708,1226){\makebox(0,0)[r]{\strut{}$-10$}}%
      \csname LTb\endcsname%%
      \put(708,1800){\makebox(0,0)[r]{\strut{}$0$}}%
      \csname LTb\endcsname%%
      \put(708,2374){\makebox(0,0)[r]{\strut{}$10$}}%
      \csname LTb\endcsname%%
      \put(708,2947){\makebox(0,0)[r]{\strut{}$20$}}%
      \csname LTb\endcsname%%
      \put(708,3521){\makebox(0,0)[r]{\strut{}$30$}}%
      \csname LTb\endcsname%%
      \put(708,4095){\makebox(0,0)[r]{\strut{}$40$}}%
      \csname LTb\endcsname%%
      \put(820,448){\makebox(0,0){\strut{}$0$}}%
      \csname LTb\endcsname%%
      \put(2025,448){\makebox(0,0){\strut{}$500$}}%
      \csname LTb\endcsname%%
      \put(3229,448){\makebox(0,0){\strut{}$1000$}}%
      \csname LTb\endcsname%%
      \put(4434,448){\makebox(0,0){\strut{}$1500$}}%
      \csname LTb\endcsname%%
      \put(5638,448){\makebox(0,0){\strut{}$2000$}}%
      \csname LTb\endcsname%%
      \put(6843,448){\makebox(0,0){\strut{}$2500$}}%
    }%
    \gplgaddtomacro\gplfronttext{%
      \csname LTb\endcsname%%
      \put(186,2373){\rotatebox{-270}{\makebox(0,0){\strut{}$v_x$}}}%
      \csname LTb\endcsname%%
      \put(3831,142){\makebox(0,0){\strut{}$t$}}%
      \csname LTb\endcsname%%
      \put(5051,3688){\makebox(0,0)[l]{\strut{}Realno}}%
      \csname LTb\endcsname%%
      \put(5051,3892){\makebox(0,0)[l]{\strut{}Akcelerometer}}%
    }%
    \gplbacktext
    \put(0,0){\includegraphics[width={360.00bp},height={216.00bp}]{graf8c}}%
    \gplfronttext
  \end{picture}%
\endgroup
}
    \resizebox{.49\linewidth}{!}{\input{graf8d.tex}}
    \caption{Hitrosti, ki jih dobimo s~Kalmanovim filtrom ob upoštevanju orientacije vozila
    in prave vrednosti.}
    \label{slika8b}
\end{figure}
Grafi odstopanja lege~\ref{slika9a} potrdijo, to kar smo opazilo -- največji odmik od prave
vrednosti je v~prvem krožnem zavoju. Enako je tudi s~hitrostmi~\ref{slika9b}.
\begin{figure}[H]
    \centering
    \resizebox{.49\linewidth}{!}{\input{graf9a.tex}}
    \resizebox{.49\linewidth}{!}{\input{graf9b.tex}}
    \caption{Pot vozila in njena rekonstrukcija s~Kalmanovim filtrom, če dobivamo podatke
    o~orientaciji vozila.}
    \label{slika9a}
\end{figure}
\begin{figure}[H]
    \centering
    \resizebox{.49\linewidth}{!}{\input{graf9c.tex}}
    \resizebox{.49\linewidth}{!}{\input{graf9d.tex}}
    \caption{Pot vozila in njena rekonstrukcija s~Kalmanovim filtrom, če dobivamo podatke
    o~orientaciji vozila.}
    \label{slika9b}
\end{figure}
Tako kot v~prvem delu naloge lahko pogledamo kako pogostost vzorčenja vpliva na natančnost
Kalmanovega filtra. Na spodnjih grafih~\ref{slika10a} in~\ref{slika10b} vidimo podobno kot
v~prvem delu naloge -- če zmanjšamo pogostost vzorčenja, povečamo odstopanje rešitve. 
Zanimivo je, da smo pri $y$ koordinati v~točki krožnega zavoja ($t\approx 450$) z~manj 
pogostim vzorčenjem bližje rešitvi. Na grafu odmika hitrosti~\ref{slika10b} opazimo, da
smo tudi hitrost bolje zadeli z~manjšim vzorčenjem. To, da smo v~eni točki izboljšali
natančnost, pa seveda ne odtehta tega, da smo jo drugje poslabšali. Lahko bi naredili 
kombinacijo in vzorčili na vsakem koraku, le na enem območju redkeje.
\begin{figure}[H]
    \centering
    \resizebox{.49\linewidth}{!}{\input{graf10a.tex}}
    \resizebox{.49\linewidth}{!}{\input{graf10b.tex}}
    \caption{Odstopanje pozicije ($x$ na levi in $y$ na desni) od točnih podatkov. S~črno
    je narisana zašumljena meritev, z~oranžno smo meritve lokacije dobili na vsakem koraku,
    z~modro na vsakem $5.$ in z~rdečo na vsakem $10.$}
    \label{slika10a}
\end{figure}
\begin{figure}[H]
    \centering
    \resizebox{.49\linewidth}{!}{\input{graf10c.tex}}
    \resizebox{.49\linewidth}{!}{\input{graf10d.tex}}
    \caption{Odstopanje hitrosti (desno $v_x$, levo $v_y$) za Kalmanov filter, kjer 
    dobivamo podatke o~orientaciji. Z~oranžno je narisan filter kjer smo meritve lokacije
    dobili na vsakem koraku, z~modro na vsakem $5.$ in z~rdečo na vsakem $10.$}
    \label{slika10b}
\end{figure}
Poglejmo si še grafe~\ref{slika11}, kjer so prikazane norme kovariančne matrike vektorja 
stanja $P$, kovariančne matrika vektorja šuma $Q$, in faktorja ojačanja $K$. Pri matriki
$P$ vidimo žagasto obliko, ki je posledica vzorčenja -- po $5$ih oziroma $10$ih korakih
se napaka spet izboljša. Velik $Q$ dobimo v~tistih točkah, ko je hitrost slab približek
prave, še posebej v~prvem delu meritve -- do krožnega zavoja. Graf norme Kalmanovega filtra
ima enako obliko kot graf norme $P$ (saj sta linearno odvisna), poleg tega pa je redkejši 
pri manjšem vzorčenju, saj je povezan tudi z~okensko funkcijo $H$, ki je $\mathbb{0}$ na
korakih, ko ne dobivamo podatkov o~meritvah.
\begin{figure}[H]
    \centering
    \resizebox{.49\linewidth}{!}{\input{graf11a.tex}}
    \resizebox{.49\linewidth}{!}{\input{graf11c.tex}}
    \resizebox{.49\linewidth}{!}{\input{graf11b.tex}}
    \caption{Norme matrik $P$, $Q$ in $K$.}
    \label{slika11}
\end{figure}

%\begin{figure}
    %\centering 
    %\includegraphics[width=.60\linewidth]{sipanje} 
    %\caption{Pot žarka nevtronov skozi ploščo debeline $d$.}
    %\label{slika7} 
%\end{figure}

\end{document}
