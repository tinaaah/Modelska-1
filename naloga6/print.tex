\documentclass[a4paper,pdftex,12pt]{article}
\usepackage[utf8]{inputenc}
\usepackage[T1]{fontenc} 
\usepackage[english,slovene]{babel} 
\usepackage{amsmath,amsfonts,amsthm,amssymb,mathrsfs,empheq} % Math packages
\usepackage{mathtools}
\usepackage{dsfont}
\usepackage[pdftex]{graphicx}
%\usepackage{makeidx}

\usepackage[version=3]{mhchem} %kemija


\renewcommand{\vec}[1]{\boldsymbol{\mathbf{#1}}}                                        
\newcommand{\ihat}[0]{\boldsymbol{\mathbf{\oldhat{\textbf{\i}}}}} % pokončna j in i (j i n i
\newcommand{\iu}{{i\mkern1mu}}	    %imaginarno število
\DeclarePairedDelimiterX{\norm}[1]{\lVert}{\rVert}{#1} %norma

\usepackage{fancyhdr} % Custom headers and footers
\pagestyle{fancyplain} % Makes all pages in the document conform to the custom headers and footers
\fancyhead{} % No page header - if you want one, create it in the same way as the footers below
\fancyfoot[L]{} % Empty left footer
\fancyfoot[C]{} % Empty center footer
\fancyfoot[R]{\thepage} % Page numbering for right footer
\renewcommand{\headrulewidth}{0pt} % Remove header underlines
\renewcommand{\footrulewidth}{0pt} % Remove footer underlines
\setlength{\headheight}{13.6pt} % Customize the height of the header

%\numberwithin{equation}{section} % Number equations within sections (i.e. 1.1, 1.2, 2.1, 2.2 instead of 1, 2, 3, 4)
\numberwithin{figure}{section} % Number figures within sections (i.e. 1.1, 1.2, 2.1, 2.2 instead of 1, 2, 3, 4)
%\numberwithin{table}{section} % Number tables within sections (i.e. 1.1, 1.2, 2.1, 2.2 instead of 1, 2, 3, 4)

\setlength\parindent{0pt} % Removes all indentation from paragraphs - comment this line for an assignment with lots of text

%----------------------------------------------------------------------------------------
%	TITLE SECTION
%----------------------------------------------------------------------------------------

\newcommand{\horrule}[1]{\rule{\linewidth}{#1}} % Create horizontal rule command with 1 argument of height

\title{	
\normalfont \normalsize 
\textsc{Modelska analiza 1} \\ [25pt] % Your university, school and/or department name(s)
%\horrule{0.2pt} \\[0.4cm] % Thin top horizontal rule
\huge 6. naloga\\ % The assignment title
%\horrule{0.2pt} \\[0.5cm] % Thick bottom horizontal rule
}

\author{Tina Klobas} % Your name

\date{\normalsize\today} % Today's date or a custom date

\begin{document}

\maketitle % Print the title

\section{Opis problema}
Pri tej nalogi želimo danim nizom podatkov prilagoditi linearno funkcijo (\emph{fit}).
Imamo začeten linearni sistem
\begin{equation}
    A \vec{x} = \vec{b}
\end{equation}
kjer je $A$ podana matrika parametrov (funkcij) sistema, $\vec{b}$ vektor rešitev,
vektor $\vec{x}$ pa želimo poiskati tako, da bo napaka
\begin{equation}\label{minimum}
    \norm{A\vec{x}-\vec{b}}=\mathrm{min}.
\end{equation}

\subsubsection*{Singularni razcep}
Za vsako matriko $A \in \mathbb{R}^{m \times n}$ obstaja singularni razcep:
\begin{equation}
    A = U \Sigma V^{T},
\end{equation}
kjer sta $U \in \mathbb{R}^{m\times m}$ in $V \in \mathbb{R}^{n\times n}$ ortogonalni
matriki in $\Sigma \in \mathbb{R}^{m \times n}$ diagonalna z~elementi $\alpha_1 \geq 
\alpha_2 \geq \dots \geq \alpha_n$ (singularne vrednosti). Velja tudi, da je minimum
(enačba~\ref{minimum}) dosežen pri vektorju:
\begin{equation}\label{singular}
    \vec{x} = \sum_{i=1}^n \frac{\vec{u}_i^T \vec{b}}{\alpha_i}\vec{v}_i,
\end{equation}

kjer sta vektorja $\vec{u}_i$ in $\vec{v}_i$ stolpca matrik $U$ in $V$.
Napake elementov tega vektorja pa so:

\begin{equation}\label{singularsigma}
    \sigma_i = \sum_{j=1}^n \left( \frac{V_{ij}}{\alpha_j} \right)^2
\end{equation}


\subsubsection*{Test $\chi^2$}
Delamo minimizacijo kumulativne $\chi^2$ funkcije:
\begin{equation}\label{hi2}
    \chi^2 = \sum_{i=1}^{N} \left( \frac{b_i - \vec{a}_i x_i}{\sigma_i} \right)^2,
\end{equation}
kjer je $\vec{a}_i$ vrstica v~matriki $A$.

%----------------------------------------------------------------------------------------
%	PROBLEM 1
%----------------------------------------------------------------------------------------

\section{Odziv tkiva}
Opazujemo odziv tkiva na različne reagente;
\begin{equation}
    \ce{Y} + \ce{X} \ce{<=>} \ce{Y}^*
\end{equation}
Če povišamo koncentracijo reagenta nad neko vrednost nasičenja, ga tkivo ne bo več
sprejelo. V~stacionarnem stanju lahko to zapišemo z~enačbo:
\begin{equation}\label{y}
    y  = \frac{y_0 x}{x+a},
\end{equation}
kjer je $y_0$ nasičeni odziv tkiva in $a$ koncentracija, potrebna za odziv, ki je enak
polovici nasičenega. V splošnem je ta zveza nelinearna, lahko pa jo preuredimo v~obliko:
\begin{equation}\label{liny}
    \frac{1}{y} = \frac{1}{y_0} + \frac{a}{y_0} \frac{1}{x}.
\end{equation}
Z~uvedbo $z = 1/y$, $z_0 = 1/y_0$, $t = 1/x$ ter $k = a/y_0$ pridemo do linearne 
enačbe oblike:
\begin{equation}\label{linz}
    z = kt + z_0
\end{equation}
Napake začetnega problema~\ref{y} so $\sigma_i$, ki se z~lineariziranjem enačbe tudi 
prepišejo:
\begin{equation}
    \Delta z = \left|{\frac{\mathrm{d}z}{\mathrm{d} y}} \right| \Delta y = \frac{1}{y^2} 
    \sigma.
\end{equation}

Napake linearizirane enačbe~\ref{liny} so: 
\begin{equation}\label{napaka}
    \widetilde{\sigma}_i = \sigma_i /y_i^2 = \sigma_i z_i^2. 
\end{equation}

 \begin{figure}    
    \resizebox{0.98\linewidth}{!}{\input{graf1.tex}}
    \caption{Na grafu je prikazana linearna regresija brez upoštevanja uteži napak na
     podatki $y$ in obratni vrednosti $z$.}
    \label{slika1}
\end{figure}
Na grafu~\ref{slika1} je prikazana linearna regresija v~primeru, ko ne upoštevamo napak
posameznih točk. Na levem grafu smo podatkom prilagajali enačbo~\ref{linz} in vidimo,
da precej odstopata zadnji dve točki (na desnem grafu prvi dve), ki imata v~skladu 
z~enačbo~\ref{napaka} tudi zelo veliko napako. Ker imajo vsi podatki enako utež, oddaljene
točke močno razpotegnejo premico in take oblike rešitev nima nobenega smisla.\\
Reševanja se lotimo tako, da minimiziramo $\chi^2$ -- poiščemo minimum enačbe~\ref{hi2} 
in pridemo do sistema enačb oblike $A\vec{a}=\vec{b}$, kjer definiramo elemente matrike $A$:
\begin{equation}
    A_{ij} = \sum_{k_1}^N \frac{\phi_i(x_k) \phi_j(x_k)}{\widetilde{\sigma_k}}.
\end{equation}
$\phi_i$ so testne funkcije. V~našem primeru imamo samo dve funkciji; $\phi_1=1$ in
$\phi_2 = t = 1/x$ in matrika $A$ ima tako obliko:
\begin{equation}
    A =  \sum_{k=1}^N \frac{1}{\widetilde{\sigma}^2_k}
    \begin{bmatrix}
	1 & \frac{1}{x_k} \\
	\frac{1}{x_k} & \frac{1}{x_k^2} \\
    \end{bmatrix}.
\end{equation}
Imamo še vektor $\vec{b}$ z~elementi
\begin{equation}
    b_i =  \sum_{k=1}^N \frac{z_k \phi_k(x_k)}{\widetilde{\sigma}^2_k},
\end{equation}
ki ima v~tem primeru samo dva člena
\begin{equation}
    \vec{b} = \sum_{k=1}^N \frac{z_k}{\widetilde{\sigma}^2_k}
    \begin{bmatrix}
	1 \\ \frac{1}{x_k}.
    \end{bmatrix}
\end{equation}
Matriki $A$ naredimo inverz, da dobimo vektor
\begin{equation}
    \vec{a} = A^{-1}\vec{b}
\end{equation}
in rešitev narišemo na graf~\ref{slika2}. Vidimo, da smo sedaj veliko bolj upoštevali 
prvih nekaj točk, ki imajo manjšo napako in tako je tudi krivulja na desnem grafu
veliko bolj pravilne oblike kot na prejšnji sliki. 
\begin{figure}    
    \resizebox{0.98\linewidth}{!}{\input{graf2.tex}}
    \caption{Na grafu je prikazana linearna regresija z~minimizacijo $\chi^2$ 
    porazdelitve.}
    \label{slika2}
\end{figure}
Obe rešitvi lahko tudi primerjamo s~pomočjo testa $\chi^2$ kjer za prvi primer, prikazan
na sliki~\ref{slika1} dobimo $\chi^2=3322,38$ za drugega, kjer smo upoštevali še napake,
pa $\chi^2=26.08$. Sicer niti druga vrednost ni še enaka idealni $\chi^2 = N-M=6$ (kjer
je $N$ število izmerkov, $M$ pa število parametrov), vendar je veliko bolje kot v~prvem
primeru.
\begin{figure}    
    \resizebox{0.98\linewidth}{!}{\input{graf3.tex}}
    \caption{Primerjava napak obeh metod.}
    \label{slika3}
\end{figure}
Na grafu~\ref{slika3} tudi vidimo primerjavo napak obeh metod. Sicer je na začetku 
drugi rezultat, ko smo gledali regresijo z~upoštevanjem uteži napak odstopanje od pravilne
vrednosti veliko večje, vendar že po drugem izmerku pade pod drugo funkcijo in dosežemo
natančnost na dve decimalni mesti.

%----------------------------------------------------------------------------------------
%	PROBLEM 2
%----------------------------------------------------------------------------------------
\section{Spekter na vzorcih lista}
Imamo zbrane štiri absorpcijske spektre kadmija na robu $L_3$ iz študije, kako
ta kovina vpliva na rastline. V~prvih dveh vzorcih so izolirane celične stene iz krovne 
plasti in iz sredice listov rastline C. \emph{Thlaspi}, ki je znan hiperakumulator težkih 
kovin.  Zadnja dva spektra sta dobljena na standardih, kompleksih \ce{Cd} sulfata 
z~glutationom (\ce{GSH}) in pektinom: v~prvem je \ce{Cd} vezan izključno na žveplo, 
v~drugem na kisik.  V listnih vzorcih dopuščamo obe vrsti vezave, vemo pa, da sta 
prispevka obeh v~spektru linearno sestavljena. Želimo dobiti odstotno razmerje vezi 
\ce{Cd-O} in \ce{Cd-S} v~obeh listnih vzorcih. Vsi štirje spektri so prikazani na 
grafu~\ref{slika4}.
\begin{figure}    
    \resizebox{0.98\linewidth}{!}{% GNUPLOT: LaTeX picture with Postscript
\begingroup
  \makeatletter
  \providecommand\color[2][]{%
    \GenericError{(gnuplot) \space\space\space\@spaces}{%
      Package color not loaded in conjunction with
      terminal option `colourtext'%
    }{See the gnuplot documentation for explanation.%
    }{Either use 'blacktext' in gnuplot or load the package
      color.sty in LaTeX.}%
    \renewcommand\color[2][]{}%
  }%
  \providecommand\includegraphics[2][]{%
    \GenericError{(gnuplot) \space\space\space\@spaces}{%
      Package graphicx or graphics not loaded%
    }{See the gnuplot documentation for explanation.%
    }{The gnuplot epslatex terminal needs graphicx.sty or graphics.sty.}%
    \renewcommand\includegraphics[2][]{}%
  }%
  \providecommand\rotatebox[2]{#2}%
  \@ifundefined{ifGPcolor}{%
    \newif\ifGPcolor
    \GPcolortrue
  }{}%
  \@ifundefined{ifGPblacktext}{%
    \newif\ifGPblacktext
    \GPblacktexttrue
  }{}%
  % define a \g@addto@macro without @ in the name:
  \let\gplgaddtomacro\g@addto@macro
  % define empty templates for all commands taking text:
  \gdef\gplbacktext{}%
  \gdef\gplfronttext{}%
  \makeatother
  \ifGPblacktext
    % no textcolor at all
    \def\colorrgb#1{}%
    \def\colorgray#1{}%
  \else
    % gray or color?
    \ifGPcolor
      \def\colorrgb#1{\color[rgb]{#1}}%
      \def\colorgray#1{\color[gray]{#1}}%
      \expandafter\def\csname LTw\endcsname{\color{white}}%
      \expandafter\def\csname LTb\endcsname{\color{black}}%
      \expandafter\def\csname LTa\endcsname{\color{black}}%
      \expandafter\def\csname LT0\endcsname{\color[rgb]{1,0,0}}%
      \expandafter\def\csname LT1\endcsname{\color[rgb]{0,1,0}}%
      \expandafter\def\csname LT2\endcsname{\color[rgb]{0,0,1}}%
      \expandafter\def\csname LT3\endcsname{\color[rgb]{1,0,1}}%
      \expandafter\def\csname LT4\endcsname{\color[rgb]{0,1,1}}%
      \expandafter\def\csname LT5\endcsname{\color[rgb]{1,1,0}}%
      \expandafter\def\csname LT6\endcsname{\color[rgb]{0,0,0}}%
      \expandafter\def\csname LT7\endcsname{\color[rgb]{1,0.3,0}}%
      \expandafter\def\csname LT8\endcsname{\color[rgb]{0.5,0.5,0.5}}%
    \else
      % gray
      \def\colorrgb#1{\color{black}}%
      \def\colorgray#1{\color[gray]{#1}}%
      \expandafter\def\csname LTw\endcsname{\color{white}}%
      \expandafter\def\csname LTb\endcsname{\color{black}}%
      \expandafter\def\csname LTa\endcsname{\color{black}}%
      \expandafter\def\csname LT0\endcsname{\color{black}}%
      \expandafter\def\csname LT1\endcsname{\color{black}}%
      \expandafter\def\csname LT2\endcsname{\color{black}}%
      \expandafter\def\csname LT3\endcsname{\color{black}}%
      \expandafter\def\csname LT4\endcsname{\color{black}}%
      \expandafter\def\csname LT5\endcsname{\color{black}}%
      \expandafter\def\csname LT6\endcsname{\color{black}}%
      \expandafter\def\csname LT7\endcsname{\color{black}}%
      \expandafter\def\csname LT8\endcsname{\color{black}}%
    \fi
  \fi
    \setlength{\unitlength}{0.0500bp}%
    \ifx\gptboxheight\undefined%
      \newlength{\gptboxheight}%
      \newlength{\gptboxwidth}%
      \newsavebox{\gptboxtext}%
    \fi%
    \setlength{\fboxrule}{0.5pt}%
    \setlength{\fboxsep}{1pt}%
\begin{picture}(7200.00,4320.00)%
    \gplgaddtomacro\gplbacktext{%
      \csname LTb\endcsname%%
      \put(708,652){\makebox(0,0)[r]{\strut{}$0$}}%
      \csname LTb\endcsname%%
      \put(708,996){\makebox(0,0)[r]{\strut{}$0.1$}}%
      \csname LTb\endcsname%%
      \put(708,1341){\makebox(0,0)[r]{\strut{}$0.2$}}%
      \csname LTb\endcsname%%
      \put(708,1685){\makebox(0,0)[r]{\strut{}$0.3$}}%
      \csname LTb\endcsname%%
      \put(708,2029){\makebox(0,0)[r]{\strut{}$0.4$}}%
      \csname LTb\endcsname%%
      \put(708,2374){\makebox(0,0)[r]{\strut{}$0.5$}}%
      \csname LTb\endcsname%%
      \put(708,2718){\makebox(0,0)[r]{\strut{}$0.6$}}%
      \csname LTb\endcsname%%
      \put(708,3062){\makebox(0,0)[r]{\strut{}$0.7$}}%
      \csname LTb\endcsname%%
      \put(708,3406){\makebox(0,0)[r]{\strut{}$0.8$}}%
      \csname LTb\endcsname%%
      \put(708,3751){\makebox(0,0)[r]{\strut{}$0.9$}}%
      \csname LTb\endcsname%%
      \put(708,4095){\makebox(0,0)[r]{\strut{}$1$}}%
      \csname LTb\endcsname%%
      \put(820,448){\makebox(0,0){\strut{}$0$}}%
      \csname LTb\endcsname%%
      \put(1996,448){\makebox(0,0){\strut{}$100$}}%
      \csname LTb\endcsname%%
      \put(3173,448){\makebox(0,0){\strut{}$200$}}%
      \csname LTb\endcsname%%
      \put(4349,448){\makebox(0,0){\strut{}$300$}}%
      \csname LTb\endcsname%%
      \put(5525,448){\makebox(0,0){\strut{}$400$}}%
      \csname LTb\endcsname%%
      \put(6702,448){\makebox(0,0){\strut{}$500$}}%
    }%
    \gplgaddtomacro\gplfronttext{%
      \csname LTb\endcsname%%
      \put(186,2373){\rotatebox{-270}{\makebox(0,0){\strut{}amplituda}}}%
      \csname LTb\endcsname%%
      \put(3831,142){\makebox(0,0){\strut{}$x$}}%
      \csname LTb\endcsname%%
      \put(1349,3280){\makebox(0,0)[l]{\strut{}Welch}}%
      \csname LTb\endcsname%%
      \put(1349,3484){\makebox(0,0)[l]{\strut{}Hann}}%
      \csname LTb\endcsname%%
      \put(1349,3688){\makebox(0,0)[l]{\strut{}Bartlett}}%
      \csname LTb\endcsname%%
      \put(1349,3892){\makebox(0,0)[l]{\strut{}eksponent}}%
    }%
    \gplbacktext
    \put(0,0){\includegraphics[width={360.00bp},height={216.00bp}]{graf4}}%
    \gplfronttext
  \end{picture}%
\endgroup
}
    \caption{Vsi štirje podani sprektri.}
    \label{slika4}
\end{figure}
Linearno sestavljenost vzorcev lahko zapišemo z~enačbama:
\begin{align}
    \mathrm{vzorec}_1(E) = a \cdot \mathrm{standard}_1 (E) 
    + b \cdot \mathrm{standard}_2 (E), \\
    \mathrm{vzorec}_2(E) = c \cdot \mathrm{standard}_1 (E) 
    + d \cdot \mathrm{standard}_2 (E).
\end{align}
Problem se lahko lotimo na enak način kot prejšnjega ali pa z~metodo singularnega razcepa
kjer za optimalni vektor uporabimo enačbo~\ref{singular} za njegovo napako 
pa~\ref{singularsigma}.
Z~grafa~\ref{slika5} vidimo, da je naše prilagajanje kar uspešno. Pri drugem vzorcu vidimo,
da je na nekaterih območjih precej bolj razgibana, kar tudi povzroči odstopanje od prave
vrednosti. To se vidi tudi z~grafa~\ref{slika6} kjer je napaka pri drugem spektru ponekod 
za tudi do več velikostnih redov večja, vendar sta obe še zmeraj zelo majhni, kar potrdi,
da smo našli optimalno oceno.\\
Dobili smo parametre $a,b,c,d$, ki predstavljajo razmerja vezi \ce{Cd-O} in \ce{Cd-S} 
v~obeh spektrih. Za prvi spekter je
\begin{enumerate}
    \item spekter: razmerje vezi \ce{Cd-O}/\ce{Cd-S} je $$\frac{a}{b} = \frac{0.27}{0.73}$$,
    \item spekter: razmerje vezi \ce{Cd-O}/\ce{Cd-S} je $$\frac{c}{d} = \frac{0.55}{0.45}.$$
\end{enumerate}

\begin{figure}    
    \resizebox{0.98\linewidth}{!}{% GNUPLOT: LaTeX picture with Postscript
\begingroup
  \makeatletter
  \providecommand\color[2][]{%
    \GenericError{(gnuplot) \space\space\space\@spaces}{%
      Package color not loaded in conjunction with
      terminal option `colourtext'%
    }{See the gnuplot documentation for explanation.%
    }{Either use 'blacktext' in gnuplot or load the package
      color.sty in LaTeX.}%
    \renewcommand\color[2][]{}%
  }%
  \providecommand\includegraphics[2][]{%
    \GenericError{(gnuplot) \space\space\space\@spaces}{%
      Package graphicx or graphics not loaded%
    }{See the gnuplot documentation for explanation.%
    }{The gnuplot epslatex terminal needs graphicx.sty or graphics.sty.}%
    \renewcommand\includegraphics[2][]{}%
  }%
  \providecommand\rotatebox[2]{#2}%
  \@ifundefined{ifGPcolor}{%
    \newif\ifGPcolor
    \GPcolortrue
  }{}%
  \@ifundefined{ifGPblacktext}{%
    \newif\ifGPblacktext
    \GPblacktexttrue
  }{}%
  % define a \g@addto@macro without @ in the name:
  \let\gplgaddtomacro\g@addto@macro
  % define empty templates for all commands taking text:
  \gdef\gplbacktext{}%
  \gdef\gplfronttext{}%
  \makeatother
  \ifGPblacktext
    % no textcolor at all
    \def\colorrgb#1{}%
    \def\colorgray#1{}%
  \else
    % gray or color?
    \ifGPcolor
      \def\colorrgb#1{\color[rgb]{#1}}%
      \def\colorgray#1{\color[gray]{#1}}%
      \expandafter\def\csname LTw\endcsname{\color{white}}%
      \expandafter\def\csname LTb\endcsname{\color{black}}%
      \expandafter\def\csname LTa\endcsname{\color{black}}%
      \expandafter\def\csname LT0\endcsname{\color[rgb]{1,0,0}}%
      \expandafter\def\csname LT1\endcsname{\color[rgb]{0,1,0}}%
      \expandafter\def\csname LT2\endcsname{\color[rgb]{0,0,1}}%
      \expandafter\def\csname LT3\endcsname{\color[rgb]{1,0,1}}%
      \expandafter\def\csname LT4\endcsname{\color[rgb]{0,1,1}}%
      \expandafter\def\csname LT5\endcsname{\color[rgb]{1,1,0}}%
      \expandafter\def\csname LT6\endcsname{\color[rgb]{0,0,0}}%
      \expandafter\def\csname LT7\endcsname{\color[rgb]{1,0.3,0}}%
      \expandafter\def\csname LT8\endcsname{\color[rgb]{0.5,0.5,0.5}}%
    \else
      % gray
      \def\colorrgb#1{\color{black}}%
      \def\colorgray#1{\color[gray]{#1}}%
      \expandafter\def\csname LTw\endcsname{\color{white}}%
      \expandafter\def\csname LTb\endcsname{\color{black}}%
      \expandafter\def\csname LTa\endcsname{\color{black}}%
      \expandafter\def\csname LT0\endcsname{\color{black}}%
      \expandafter\def\csname LT1\endcsname{\color{black}}%
      \expandafter\def\csname LT2\endcsname{\color{black}}%
      \expandafter\def\csname LT3\endcsname{\color{black}}%
      \expandafter\def\csname LT4\endcsname{\color{black}}%
      \expandafter\def\csname LT5\endcsname{\color{black}}%
      \expandafter\def\csname LT6\endcsname{\color{black}}%
      \expandafter\def\csname LT7\endcsname{\color{black}}%
      \expandafter\def\csname LT8\endcsname{\color{black}}%
    \fi
  \fi
    \setlength{\unitlength}{0.0500bp}%
    \ifx\gptboxheight\undefined%
      \newlength{\gptboxheight}%
      \newlength{\gptboxwidth}%
      \newsavebox{\gptboxtext}%
    \fi%
    \setlength{\fboxrule}{0.5pt}%
    \setlength{\fboxsep}{1pt}%
\begin{picture}(7180.00,4300.00)%
    \gplgaddtomacro\gplbacktext{%
      \csname LTb\endcsname%%
      \put(504,408){\makebox(0,0)[r]{\strut{}$0$}}%
      \csname LTb\endcsname%%
      \put(504,1023){\makebox(0,0)[r]{\strut{}$20$}}%
      \csname LTb\endcsname%%
      \put(504,1637){\makebox(0,0)[r]{\strut{}$40$}}%
      \csname LTb\endcsname%%
      \put(504,2252){\makebox(0,0)[r]{\strut{}$60$}}%
      \csname LTb\endcsname%%
      \put(504,2866){\makebox(0,0)[r]{\strut{}$80$}}%
      \csname LTb\endcsname%%
      \put(504,3481){\makebox(0,0)[r]{\strut{}$100$}}%
      \csname LTb\endcsname%%
      \put(504,4095){\makebox(0,0)[r]{\strut{}$120$}}%
      \csname LTb\endcsname%%
      \put(616,204){\makebox(0,0){\strut{}$0$}}%
      \csname LTb\endcsname%%
      \put(1861,204){\makebox(0,0){\strut{}$500$}}%
      \csname LTb\endcsname%%
      \put(3107,204){\makebox(0,0){\strut{}$1000$}}%
      \csname LTb\endcsname%%
      \put(4352,204){\makebox(0,0){\strut{}$1500$}}%
      \csname LTb\endcsname%%
      \put(5598,204){\makebox(0,0){\strut{}$2000$}}%
      \csname LTb\endcsname%%
      \put(6843,204){\makebox(0,0){\strut{}$2500$}}%
    }%
    \gplgaddtomacro\gplfronttext{%
      \csname LTb\endcsname%%
      \put(3931,3688){\makebox(0,0)[l]{\strut{}$B=$ vsaka $5./10.$ meritev}}%
      \csname LTb\endcsname%%
      \put(3931,3892){\makebox(0,0)[l]{\strut{}$A=$ vsaka meritev}}%
      \csname LTb\endcsname%%
      \put(3730,1637){\makebox(0,0)[l]{\strut{}$\mu_A$}}%
      \csname LTb\endcsname%%
      \put(3730,1483){\makebox(0,0)[l]{\strut{}$\mu_B$}}%
    }%
    \gplbacktext
    \put(0,0){\includegraphics{graf5}}%
    \gplfronttext
  \end{picture}%
\endgroup
}
    \caption{Narisana sta oba vzorca in dobljena kombinacija standardov.}
    \label{slika5}
\end{figure}
\begin{figure}    
    \resizebox{0.98\linewidth}{!}{% GNUPLOT: LaTeX picture with Postscript
\begingroup
  \makeatletter
  \providecommand\color[2][]{%
    \GenericError{(gnuplot) \space\space\space\@spaces}{%
      Package color not loaded in conjunction with
      terminal option `colourtext'%
    }{See the gnuplot documentation for explanation.%
    }{Either use 'blacktext' in gnuplot or load the package
      color.sty in LaTeX.}%
    \renewcommand\color[2][]{}%
  }%
  \providecommand\includegraphics[2][]{%
    \GenericError{(gnuplot) \space\space\space\@spaces}{%
      Package graphicx or graphics not loaded%
    }{See the gnuplot documentation for explanation.%
    }{The gnuplot epslatex terminal needs graphicx.sty or graphics.sty.}%
    \renewcommand\includegraphics[2][]{}%
  }%
  \providecommand\rotatebox[2]{#2}%
  \@ifundefined{ifGPcolor}{%
    \newif\ifGPcolor
    \GPcolortrue
  }{}%
  \@ifundefined{ifGPblacktext}{%
    \newif\ifGPblacktext
    \GPblacktexttrue
  }{}%
  % define a \g@addto@macro without @ in the name:
  \let\gplgaddtomacro\g@addto@macro
  % define empty templates for all commands taking text:
  \gdef\gplbacktext{}%
  \gdef\gplfronttext{}%
  \makeatother
  \ifGPblacktext
    % no textcolor at all
    \def\colorrgb#1{}%
    \def\colorgray#1{}%
  \else
    % gray or color?
    \ifGPcolor
      \def\colorrgb#1{\color[rgb]{#1}}%
      \def\colorgray#1{\color[gray]{#1}}%
      \expandafter\def\csname LTw\endcsname{\color{white}}%
      \expandafter\def\csname LTb\endcsname{\color{black}}%
      \expandafter\def\csname LTa\endcsname{\color{black}}%
      \expandafter\def\csname LT0\endcsname{\color[rgb]{1,0,0}}%
      \expandafter\def\csname LT1\endcsname{\color[rgb]{0,1,0}}%
      \expandafter\def\csname LT2\endcsname{\color[rgb]{0,0,1}}%
      \expandafter\def\csname LT3\endcsname{\color[rgb]{1,0,1}}%
      \expandafter\def\csname LT4\endcsname{\color[rgb]{0,1,1}}%
      \expandafter\def\csname LT5\endcsname{\color[rgb]{1,1,0}}%
      \expandafter\def\csname LT6\endcsname{\color[rgb]{0,0,0}}%
      \expandafter\def\csname LT7\endcsname{\color[rgb]{1,0.3,0}}%
      \expandafter\def\csname LT8\endcsname{\color[rgb]{0.5,0.5,0.5}}%
    \else
      % gray
      \def\colorrgb#1{\color{black}}%
      \def\colorgray#1{\color[gray]{#1}}%
      \expandafter\def\csname LTw\endcsname{\color{white}}%
      \expandafter\def\csname LTb\endcsname{\color{black}}%
      \expandafter\def\csname LTa\endcsname{\color{black}}%
      \expandafter\def\csname LT0\endcsname{\color{black}}%
      \expandafter\def\csname LT1\endcsname{\color{black}}%
      \expandafter\def\csname LT2\endcsname{\color{black}}%
      \expandafter\def\csname LT3\endcsname{\color{black}}%
      \expandafter\def\csname LT4\endcsname{\color{black}}%
      \expandafter\def\csname LT5\endcsname{\color{black}}%
      \expandafter\def\csname LT6\endcsname{\color{black}}%
      \expandafter\def\csname LT7\endcsname{\color{black}}%
      \expandafter\def\csname LT8\endcsname{\color{black}}%
    \fi
  \fi
    \setlength{\unitlength}{0.0500bp}%
    \ifx\gptboxheight\undefined%
      \newlength{\gptboxheight}%
      \newlength{\gptboxwidth}%
      \newsavebox{\gptboxtext}%
    \fi%
    \setlength{\fboxrule}{0.5pt}%
    \setlength{\fboxsep}{1pt}%
\begin{picture}(7180.00,4300.00)%
    \gplgaddtomacro\gplbacktext{%
      \csname LTb\endcsname%%
      \put(728,408){\makebox(0,0)[r]{\strut{}$-4000$}}%
      \csname LTb\endcsname%%
      \put(728,1023){\makebox(0,0)[r]{\strut{}$-2000$}}%
      \csname LTb\endcsname%%
      \put(728,1637){\makebox(0,0)[r]{\strut{}$0$}}%
      \csname LTb\endcsname%%
      \put(728,2252){\makebox(0,0)[r]{\strut{}$2000$}}%
      \csname LTb\endcsname%%
      \put(728,2866){\makebox(0,0)[r]{\strut{}$4000$}}%
      \csname LTb\endcsname%%
      \put(728,3481){\makebox(0,0)[r]{\strut{}$6000$}}%
      \csname LTb\endcsname%%
      \put(728,4095){\makebox(0,0)[r]{\strut{}$8000$}}%
      \csname LTb\endcsname%%
      \put(840,204){\makebox(0,0){\strut{}$-5000$}}%
      \csname LTb\endcsname%%
      \put(1698,204){\makebox(0,0){\strut{}$0$}}%
      \csname LTb\endcsname%%
      \put(2555,204){\makebox(0,0){\strut{}$5000$}}%
      \csname LTb\endcsname%%
      \put(3413,204){\makebox(0,0){\strut{}$10000$}}%
      \csname LTb\endcsname%%
      \put(4270,204){\makebox(0,0){\strut{}$15000$}}%
      \csname LTb\endcsname%%
      \put(5128,204){\makebox(0,0){\strut{}$20000$}}%
      \csname LTb\endcsname%%
      \put(5985,204){\makebox(0,0){\strut{}$25000$}}%
      \csname LTb\endcsname%%
      \put(6843,204){\makebox(0,0){\strut{}$30000$}}%
    }%
    \gplgaddtomacro\gplfronttext{%
      \csname LTb\endcsname%%
      \put(1369,3484){\makebox(0,0)[l]{\strut{}brez hitrosti}}%
      \csname LTb\endcsname%%
      \put(1369,3688){\makebox(0,0)[l]{\strut{}brez lokacija}}%
      \csname LTb\endcsname%%
      \put(1369,3892){\makebox(0,0)[l]{\strut{}kontrola}}%
    }%
    \gplbacktext
    \put(0,0){\includegraphics{graf6}}%
    \gplfronttext
  \end{picture}%
\endgroup
}
    \caption{Prikazan je logaritem odmikov dobljenih vrednosti od izmerjenih.}
    \label{slika6}
\end{figure}


%----------------------------------------------------------------------------------------
%	PROBLEM 3
%----------------------------------------------------------------------------------------
\section{Magnetni spektrometer}
Z~dobro rekonstrukcijo poti delcev v~spektrometrih lahko potem izračunamo poljubne 
kinematične količine. Naša naloga bo tako poiskati to rekonstrukcijo. Podane imamo podatke 
o~začetnih položajih $x_{\mathrm{fp}}$ v~goriščni ravnini, odmik $\vartheta_{\mathrm{fp}}$
od goriščne ravnine in disperzijski kot $\vartheta_{\mathrm{t}}$ na tarči. Želimo
poiskati preslikavo 
\begin{equation}
    (x_{\mathrm{fp}}, \vartheta_{\mathrm{fp}} ) \rightarrow \vartheta_{\mathrm{t}}.
\end{equation}
Za testni model vzamemo obliko:
\begin{equation}\label{polinom}
    f(\vartheta, x) = \left( \sum_{p=0}^N b_p \vartheta^p \right)\left(
    \sum_{q=0}^Mc_q x^q \right).
\end{equation}
Spet lahko rešujemo preko metode singularnega razcepa in po $\chi^2$ testu preverimo
kako dobro se prilagajajo podatkom. Grafi podatkov in \emph{fita} niso najbolje pregledni,
je pa toliko bolj zanimivo kako se $\chi^2$ spreminja s~spreminjanjem 
polinoma~\ref{polinom}. S~slike~\ref{slika7} vidimo, da smo najbolj ugoden moden preslikave
dobili, ko smo vzeli polinom s~dvajsetimi členi, kar bo takrat, ko bomo vzeli stopnjo
$N=4$ in $M=5$ in upoštevali le produkte skupne stopnje $p+q>5$. Točko tik nad tem 
minimumom dobimo na enak način, le da je tam stopnja $N=5$ in $N=4$. Pri vsakem številu
členov bi dobili več točk, če bi upoštevali tudi višje potence pri vsakem računanju, 
vendar pri večji razliki med maksimalnima stopnjama polinomov $M$ in $N$
dobivamo večja odstopanja kot tukaj kjer smo gledali le $|N-M|=1$. \\

\begin{figure}    
    \resizebox{0.98\linewidth}{!}{\input{graf7.tex}}
    \caption{Primerjava odstopanja od prave vrednosti za različno število členov 
    v~polinomu.}
    \label{slika7}
\end{figure}
\subsubsection*{Negativne potence polinoma}
Podobno lahko naredimo tudi, če vzamemo, da lahko $p$ in $q$ zavzameta tudi negativna 
števila in zopet narišemo graf~\ref{slika8} $\chi^2$ v~odvisnosti od števila členov 
polinoma. Tokratni minimum imamo pri $35$ členih -- narisan z~oranžno, z~modro (pri
$36$ členih) pa je narisan optimalni polinom s~prejšnjega grafa, le da mu tokrat
dovolimo da zavzame še člene na stopnjo $p=q=-1$. Vrednost minimuma je primerljiva 
s~prejšnjim ($\chi^2_1 = 44.98$, $\chi^2_2 = 49.33$) vendar smo morali porabiti
veliko več računanja, da smo dobili (nekoliko) slabše prilagajanje.
\begin{figure}    
    \resizebox{0.98\linewidth}{!}{\input{graf8.tex}}
    \caption{Primerjava odstopanja od prave vrednosti za različno število členov 
    v~polinomu, ki vsebuje tudi negativne potence.}
    \label{slika8}
\end{figure}



\end{document}
