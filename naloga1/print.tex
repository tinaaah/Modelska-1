%%%%%%%%%%%%%%%%%%%%%%%%%%%%%%%%%%%%%%%%%
% Short Sectioned Assignment
% LaTeX Template
% Version 1.0 (5/5/12)
%
% This template has been downloaded from:
% http://www.LaTeXTemplates.com
%
% Original author:
% Frits Wenneker (http://www.howtotex.com)
%
% License:
% CC BY-NC-SA 3.0 (http://creativecommons.org/licenses/by-nc-sa/3.0/)
%
%%%%%%%%%%%%%%%%%%%%%%%%%%%%%%%%%%%%%%%%%

%----------------------------------------------------------------------------------------
%	PACKAGES AND OTHER DOCUMENT CONFIGURATIONS
%----------------------------------------------------------------------------------------

\documentclass[a4paper,pdftex,12pt]{article} % A4 paper and 11pt font size

\usepackage[T1]{fontenc} % Use 8-bit encoding that has 256 glyphs
\usepackage[english,slovene]{babel} % English language/hyphenation

\usepackage{amsmath,amsfonts,amsthm,amssymb,mathrsfs} % Math packages
\usepackage{dsfont}

\usepackage[pdftex]{graphicx}

\usepackage{lipsum} % Used for inserting dummy 'Lorem ipsum' text into the template

\usepackage{sectsty} % Allows customizing section commands
%\allsectionsfont{\centering \normalfont\scshape} % Make all sections centered, the default font and small caps

\usepackage{fancyhdr} % Custom headers and footers
\pagestyle{fancyplain} % Makes all pages in the document conform to the custom headers and footers
\fancyhead{} % No page header - if you want one, create it in the same way as the footers below
\fancyfoot[L]{} % Empty left footer
\fancyfoot[C]{} % Empty center footer
\fancyfoot[R]{\thepage} % Page numbering for right footer
\renewcommand{\headrulewidth}{0pt} % Remove header underlines
\renewcommand{\footrulewidth}{0pt} % Remove footer underlines
\setlength{\headheight}{13.6pt} % Customize the height of the header

\numberwithin{equation}{section} % Number equations within sections (i.e. 1.1, 1.2, 2.1, 2.2 instead of 1, 2, 3, 4)
\numberwithin{figure}{section} % Number figures within sections (i.e. 1.1, 1.2, 2.1, 2.2 instead of 1, 2, 3, 4)
\numberwithin{table}{section} % Number tables within sections (i.e. 1.1, 1.2, 2.1, 2.2 instead of 1, 2, 3, 4)

\setlength\parindent{0pt} % Removes all indentation from paragraphs - comment this line for an assignment with lots of text

%----------------------------------------------------------------------------------------
%	TITLE SECTION
%----------------------------------------------------------------------------------------

\newcommand{\horrule}[1]{\rule{\linewidth}{#1}} % Create horizontal rule command with 1 argument of height

\title{	
\normalfont \normalsize 
\textsc{Modelska analiza 1} \\ [25pt] % Your university, school and/or department name(s)
\horrule{0.2pt} \\[0.4cm] % Thin top horizontal rule
\huge 1. naloga\\ % The assignment title
\horrule{0.2pt} \\[0.5cm] % Thick bottom horizontal rule
}

\author{Tina Klobas} % Your name

\date{\normalsize\today} % Today's date or a custom date

\begin{document}

\maketitle % Print the title

\section{Opis problema}
Pri tej nalogi nas zanima kakšne oblika mora biti funkcija hitrosti vožnje, da je vožnja
čimbolj udobna hkrati pa želimo prevoziti razdaljo do semaforja ravno v trenutku, ko se
prižge zelena luč. Za pogoj udobne vožnje minimiziramo integral kvadrata pospeška:
\begin{equation} \label{eq1}
    \int_0^{t_0} \! \dot v^2 \, \mathrm{d}t,
\end{equation}
ob pogoju, da prevozimo razdaljo v~določenem časovnem intervalu:
\begin{equation} \label{eq2}
    \int_0^{t_0} \! v \,  \mathrm{d}t \leq l.
\end{equation}

%----------------------------------------------------------------------------------------
%	PROBLEM 1
%----------------------------------------------------------------------------------------

\section{Brezdimenzijska oblika problema}
Da problem spravimo v~brezdimenzijsko obliko, zapišemo novi količini:
\begin{equation*}
    \tau = \frac{t}{t_0} \quad \mathrm{in} \quad u=v\frac{t_0}{l}.
\end{equation*}
Tako iz enačb~\ref{eq1} ter~\ref{eq2} dobimo:
\begin{equation} \label{eq3}
    \int_0^1 \! \dot u^2 \, \mathrm{d}\tau = \mathrm{min}
\end{equation}
in
\begin{equation}\label{vez}
    \int_0^1 \! u \, \mathrm{d}\tau \leq 1.
\end{equation}

%----------------------------------------------------------------------------------------
%	PROBLEM 2
%----------------------------------------------------------------------------------------

\section{Optimalna vožnja}

Da minimiziramo funkcijo, ki nastopa v~enačbi~\ref{eq3} sestavimo Lagrangejevo funkcijo: 
$\mathcal{L} = \dot u^2 - \lambda u$ iz česar po uporabi Euler-Lagrangejevih enačb
\begin{equation}\label{EL}
    \frac{\mathrm{d}}{\mathrm{d}t} \left( \frac{\partial \mathcal{L} }{\partial \dot q} 
    \right) - \frac{\partial \mathcal{L}}{\partial q} = 0
\end{equation}
dobimo diferencialno enačbo
\begin{equation}
    2 \ddot u + \lambda = 0.
\end{equation}
Ob upoštevanju robnih pogojev
\begin{equation}
    1. \; u(0) = u_0 \qquad 2. \; \dot u(1) = 0,
\end{equation}
kjer je $u_0$ neka poljubna začetna hitrost, in vezi~\ref{vez} dobimo rešitev za 
brezdimenzijsko hitrost:
\begin{equation}\label{osnovna}
    u(\tau) = - \frac{3}{2} \left( 1- u_0 \right) \tau^2 + 3 (1-u_0) \tau + u_0.
\end{equation}
Rešitve se razlikujejo zaradi prostega parametra začetne hitrosti $u_0$, vendar jim je 
vsem skupno to, da je njihov integral enak ena kar je na grafu~\ref{slika1} razvidno v~tem
da imajo vsi enako ploščino pod grafom.
\begin{figure}    
    \input{temp.tex}
    \caption{Hitrost $u$ ob različnih začetnih hitrostih $u_0$}
    \label{slika1}
\end{figure}
\subsection{Radar tik za semaforjem}
Če omejimo še končno hitrost dobimo $2.$ robni pogoj:
\begin{equation}
    u(1)= u_1
\end{equation}
s čimer dobimo dodaten prosti parameter katerega lahko spreminjamo. Na sliki~\ref{slika2}
smo si izbrali različne končne hitrosti pri enakih začetnih, funkcija $u$ pa je oblike:
\begin{equation}\label{robni}
    u = -6 \left(1- \frac{u1+u0}{2} \right) \tau^2 + 2 (3-u_1-2u_0) \tau + u_0.
\end{equation}

\begin{figure}    
    \input{graf3.tex}
    \caption{Hitrost $u$ ob različnih končnih hitrostih $u_1$ in pri začetni $u_0=0,5$.}
    \label{slika2}
\end{figure}

%----------------------------------------------------------------------------------------
%	PROBLEM 3
%----------------------------------------------------------------------------------------

\section{Višje potence pospeška}
Posplošimo problem udobne vožnje na sode potence pospeška;
\begin{equation} 
    \int_0^1 \! \dot u^{2p} \, \mathrm{d}\tau = \mathrm{min},
\end{equation}
kjer naj bo $p \in \mathds{Z} $.
Na enak način kot prej z~uporabo~\ref{EL} dobimo enačbo za hitrost:
\begin{equation}
    u = \frac{4p-1}{2p} \left( 1-u_0 \right) \left[ 1 - \left( 1- \tau 
    \right)^{\frac{2p}{2p-1}} \right] + u_0.
\end{equation}
Z~grafa~\ref{slika2}, kjer so prikazane funkcije hitrosti pri štirih različnih izbirah $p$,
vidimo, da sta limiti, ko gre $p$ proti zelo velikim vrednostim ter proti zelo majhnim,
isti. To tudi računsko potrdimo, saj je: $$ \lim_{p \to \pm \infty}\! u = 2 \tau + u_0.$$
\begin{figure}    
    \input{graf2.tex}
    \caption{Hitrost $u$ ob začetnem pogoju $u_0 = 0,5$ in pri različnih izbirah p.}
    \label{slika3}
\end{figure}

%----------------------------------------------------------------------------------------
%	PROBLEM 3
%----------------------------------------------------------------------------------------

\section{Višje potence hitrosti}
Zanimiv problem je razširitev problema, kjer omejimo tudi kvadrat hitrosti v~funkcionalu:
$\mathcal{L} = \dot u^2 + c u^2 - \lambda u$. Spet delamo na podoben način kot prej in
pridelamo enačbo:
\begin{equation}
    u = (u_0-u_c) \left[ \operatorname{cosh}(\sqrt{c}t) -\operatorname{tanh} (\sqrt{c})
    \operatorname{sinh}(\sqrt{c}t) \right] + u_c ,
\end{equation}
kjer je $u_c = (1-u_0) / (1 - \operatorname{tanh}(\sqrt{c})/\sqrt{c})$. Z~grafa~\ref{slika4}
vidimo, da funkcija s~padanjem $c$ proti $0$ pričakovano limitira k~rešitvi začetnega
problema~\ref{osnovna}.
\begin{figure}    
    % GNUPLOT: LaTeX picture with Postscript
\begingroup
  \makeatletter
  \providecommand\color[2][]{%
    \GenericError{(gnuplot) \space\space\space\@spaces}{%
      Package color not loaded in conjunction with
      terminal option `colourtext'%
    }{See the gnuplot documentation for explanation.%
    }{Either use 'blacktext' in gnuplot or load the package
      color.sty in LaTeX.}%
    \renewcommand\color[2][]{}%
  }%
  \providecommand\includegraphics[2][]{%
    \GenericError{(gnuplot) \space\space\space\@spaces}{%
      Package graphicx or graphics not loaded%
    }{See the gnuplot documentation for explanation.%
    }{The gnuplot epslatex terminal needs graphicx.sty or graphics.sty.}%
    \renewcommand\includegraphics[2][]{}%
  }%
  \providecommand\rotatebox[2]{#2}%
  \@ifundefined{ifGPcolor}{%
    \newif\ifGPcolor
    \GPcolortrue
  }{}%
  \@ifundefined{ifGPblacktext}{%
    \newif\ifGPblacktext
    \GPblacktexttrue
  }{}%
  % define a \g@addto@macro without @ in the name:
  \let\gplgaddtomacro\g@addto@macro
  % define empty templates for all commands taking text:
  \gdef\gplbacktext{}%
  \gdef\gplfronttext{}%
  \makeatother
  \ifGPblacktext
    % no textcolor at all
    \def\colorrgb#1{}%
    \def\colorgray#1{}%
  \else
    % gray or color?
    \ifGPcolor
      \def\colorrgb#1{\color[rgb]{#1}}%
      \def\colorgray#1{\color[gray]{#1}}%
      \expandafter\def\csname LTw\endcsname{\color{white}}%
      \expandafter\def\csname LTb\endcsname{\color{black}}%
      \expandafter\def\csname LTa\endcsname{\color{black}}%
      \expandafter\def\csname LT0\endcsname{\color[rgb]{1,0,0}}%
      \expandafter\def\csname LT1\endcsname{\color[rgb]{0,1,0}}%
      \expandafter\def\csname LT2\endcsname{\color[rgb]{0,0,1}}%
      \expandafter\def\csname LT3\endcsname{\color[rgb]{1,0,1}}%
      \expandafter\def\csname LT4\endcsname{\color[rgb]{0,1,1}}%
      \expandafter\def\csname LT5\endcsname{\color[rgb]{1,1,0}}%
      \expandafter\def\csname LT6\endcsname{\color[rgb]{0,0,0}}%
      \expandafter\def\csname LT7\endcsname{\color[rgb]{1,0.3,0}}%
      \expandafter\def\csname LT8\endcsname{\color[rgb]{0.5,0.5,0.5}}%
    \else
      % gray
      \def\colorrgb#1{\color{black}}%
      \def\colorgray#1{\color[gray]{#1}}%
      \expandafter\def\csname LTw\endcsname{\color{white}}%
      \expandafter\def\csname LTb\endcsname{\color{black}}%
      \expandafter\def\csname LTa\endcsname{\color{black}}%
      \expandafter\def\csname LT0\endcsname{\color{black}}%
      \expandafter\def\csname LT1\endcsname{\color{black}}%
      \expandafter\def\csname LT2\endcsname{\color{black}}%
      \expandafter\def\csname LT3\endcsname{\color{black}}%
      \expandafter\def\csname LT4\endcsname{\color{black}}%
      \expandafter\def\csname LT5\endcsname{\color{black}}%
      \expandafter\def\csname LT6\endcsname{\color{black}}%
      \expandafter\def\csname LT7\endcsname{\color{black}}%
      \expandafter\def\csname LT8\endcsname{\color{black}}%
    \fi
  \fi
    \setlength{\unitlength}{0.0500bp}%
    \ifx\gptboxheight\undefined%
      \newlength{\gptboxheight}%
      \newlength{\gptboxwidth}%
      \newsavebox{\gptboxtext}%
    \fi%
    \setlength{\fboxrule}{0.5pt}%
    \setlength{\fboxsep}{1pt}%
\begin{picture}(7200.00,4320.00)%
    \gplgaddtomacro\gplbacktext{%
      \csname LTb\endcsname%%
      \put(708,652){\makebox(0,0)[r]{\strut{}$0$}}%
      \csname LTb\endcsname%%
      \put(708,996){\makebox(0,0)[r]{\strut{}$0.1$}}%
      \csname LTb\endcsname%%
      \put(708,1341){\makebox(0,0)[r]{\strut{}$0.2$}}%
      \csname LTb\endcsname%%
      \put(708,1685){\makebox(0,0)[r]{\strut{}$0.3$}}%
      \csname LTb\endcsname%%
      \put(708,2029){\makebox(0,0)[r]{\strut{}$0.4$}}%
      \csname LTb\endcsname%%
      \put(708,2374){\makebox(0,0)[r]{\strut{}$0.5$}}%
      \csname LTb\endcsname%%
      \put(708,2718){\makebox(0,0)[r]{\strut{}$0.6$}}%
      \csname LTb\endcsname%%
      \put(708,3062){\makebox(0,0)[r]{\strut{}$0.7$}}%
      \csname LTb\endcsname%%
      \put(708,3406){\makebox(0,0)[r]{\strut{}$0.8$}}%
      \csname LTb\endcsname%%
      \put(708,3751){\makebox(0,0)[r]{\strut{}$0.9$}}%
      \csname LTb\endcsname%%
      \put(708,4095){\makebox(0,0)[r]{\strut{}$1$}}%
      \csname LTb\endcsname%%
      \put(820,448){\makebox(0,0){\strut{}$0$}}%
      \csname LTb\endcsname%%
      \put(1996,448){\makebox(0,0){\strut{}$100$}}%
      \csname LTb\endcsname%%
      \put(3173,448){\makebox(0,0){\strut{}$200$}}%
      \csname LTb\endcsname%%
      \put(4349,448){\makebox(0,0){\strut{}$300$}}%
      \csname LTb\endcsname%%
      \put(5525,448){\makebox(0,0){\strut{}$400$}}%
      \csname LTb\endcsname%%
      \put(6702,448){\makebox(0,0){\strut{}$500$}}%
    }%
    \gplgaddtomacro\gplfronttext{%
      \csname LTb\endcsname%%
      \put(186,2373){\rotatebox{-270}{\makebox(0,0){\strut{}amplituda}}}%
      \csname LTb\endcsname%%
      \put(3831,142){\makebox(0,0){\strut{}$x$}}%
      \csname LTb\endcsname%%
      \put(1349,3280){\makebox(0,0)[l]{\strut{}Welch}}%
      \csname LTb\endcsname%%
      \put(1349,3484){\makebox(0,0)[l]{\strut{}Hann}}%
      \csname LTb\endcsname%%
      \put(1349,3688){\makebox(0,0)[l]{\strut{}Bartlett}}%
      \csname LTb\endcsname%%
      \put(1349,3892){\makebox(0,0)[l]{\strut{}eksponent}}%
    }%
    \gplbacktext
    \put(0,0){\includegraphics[width={360.00bp},height={216.00bp}]{graf4}}%
    \gplfronttext
  \end{picture}%
\endgroup

    \caption{Hitrost $u$ ob različnih vrednostih konstante $c$ in pri začetni hitrosti
    $u_0=0,5$.}
    \label{slika4}
\end{figure}

%----------------------------------------------------------------------------------------
%	PROBLEM 3
%----------------------------------------------------------------------------------------

\section{Zaporedni semaforji}
Dodatna razširitev osnovnega problema je primer zaporednih semaforjev. V splošnem imamo
na vsakem intervalu med zaporednima semaforjema funkcijo hitrosti oblike:
\begin{equation}
    u^{(i)} = - \frac{\lambda_i}{4} \tau^2 + A_i \tau + B_i.
\end{equation}
Koeficiente $A_i$ in $B_i$ določimo iz robnega pogoja za prvi semafor $u^{(0)}(0)=u_0$ ter
za zadnjega $\dot u^{(n)}(t_0^{(i)}/t_0^{(0)}) = 0$, kjer smo na podoben način kot na 
začetku vse normirali na konstante iz začetnega problema $t_0^{(0)}$ in $l_0^{(0)}$. Na 
mejah med zaporednimi semaforji morata veljati še pogoja:
\begin{equation}
    1. \, u^{(i)}(\mathrm{zg.})=u^{(i+1)}(\mathrm{sp.}) \quad \mathrm{in} \quad
    2. \, \dot u^{(i)}(\mathrm{zg.})=\dot u^{(i+1)}(\mathrm{sp.})
\end{equation}
kjer smo z~\emph{zg.} označili zgornji rob s~\emph{sp.} pa spodnjega. 
\subsection{Primer za 2 semaforja}
Imamo dveh sistem enačb
\begin{align}
    u_1 = \frac{\lambda_1}{4}\tau^2 + A_1\tau + B_1  \\
    u_2 = \frac{\lambda_2}{4}\tau^2 + A_2\tau + B_2
\end{align}
s štirimi robnimi pogoji:
\begin{align}
    u_1 (0)  &= u_0, \\
    u_2 \left( \frac{t_1}{t_1} \right) &= 0, \\
    u_1 (1) &= u_2 (1) \, \mathrm{in}\\
    \dot u_1 (1) &= \dot u_2 (1) 
\end{align}
(velja $t_1=t_0$ in $l_1=l$ iz začetnega problema). Vpeljemo še nove oznake 
$T= (t_2)/(t_1)$ in $L=(l_2)/l_1$. Vključno z~obemi vezmi (oblike~\ref{vez}) imamo
tako $6$ enačb za $6$ neznank in (s~pomočjo priljubljenega računala) pridemo do rešitve,
prikazane na sliki~\ref{slika5}. Vidimo, da se z~izbrano Lagrangejevo funkcijo ne izognemo
negativnim hitrostim in bi jo morali prilagoditi. 
\begin{figure}    
    % GNUPLOT: LaTeX picture with Postscript
\begingroup
  \makeatletter
  \providecommand\color[2][]{%
    \GenericError{(gnuplot) \space\space\space\@spaces}{%
      Package color not loaded in conjunction with
      terminal option `colourtext'%
    }{See the gnuplot documentation for explanation.%
    }{Either use 'blacktext' in gnuplot or load the package
      color.sty in LaTeX.}%
    \renewcommand\color[2][]{}%
  }%
  \providecommand\includegraphics[2][]{%
    \GenericError{(gnuplot) \space\space\space\@spaces}{%
      Package graphicx or graphics not loaded%
    }{See the gnuplot documentation for explanation.%
    }{The gnuplot epslatex terminal needs graphicx.sty or graphics.sty.}%
    \renewcommand\includegraphics[2][]{}%
  }%
  \providecommand\rotatebox[2]{#2}%
  \@ifundefined{ifGPcolor}{%
    \newif\ifGPcolor
    \GPcolortrue
  }{}%
  \@ifundefined{ifGPblacktext}{%
    \newif\ifGPblacktext
    \GPblacktexttrue
  }{}%
  % define a \g@addto@macro without @ in the name:
  \let\gplgaddtomacro\g@addto@macro
  % define empty templates for all commands taking text:
  \gdef\gplbacktext{}%
  \gdef\gplfronttext{}%
  \makeatother
  \ifGPblacktext
    % no textcolor at all
    \def\colorrgb#1{}%
    \def\colorgray#1{}%
  \else
    % gray or color?
    \ifGPcolor
      \def\colorrgb#1{\color[rgb]{#1}}%
      \def\colorgray#1{\color[gray]{#1}}%
      \expandafter\def\csname LTw\endcsname{\color{white}}%
      \expandafter\def\csname LTb\endcsname{\color{black}}%
      \expandafter\def\csname LTa\endcsname{\color{black}}%
      \expandafter\def\csname LT0\endcsname{\color[rgb]{1,0,0}}%
      \expandafter\def\csname LT1\endcsname{\color[rgb]{0,1,0}}%
      \expandafter\def\csname LT2\endcsname{\color[rgb]{0,0,1}}%
      \expandafter\def\csname LT3\endcsname{\color[rgb]{1,0,1}}%
      \expandafter\def\csname LT4\endcsname{\color[rgb]{0,1,1}}%
      \expandafter\def\csname LT5\endcsname{\color[rgb]{1,1,0}}%
      \expandafter\def\csname LT6\endcsname{\color[rgb]{0,0,0}}%
      \expandafter\def\csname LT7\endcsname{\color[rgb]{1,0.3,0}}%
      \expandafter\def\csname LT8\endcsname{\color[rgb]{0.5,0.5,0.5}}%
    \else
      % gray
      \def\colorrgb#1{\color{black}}%
      \def\colorgray#1{\color[gray]{#1}}%
      \expandafter\def\csname LTw\endcsname{\color{white}}%
      \expandafter\def\csname LTb\endcsname{\color{black}}%
      \expandafter\def\csname LTa\endcsname{\color{black}}%
      \expandafter\def\csname LT0\endcsname{\color{black}}%
      \expandafter\def\csname LT1\endcsname{\color{black}}%
      \expandafter\def\csname LT2\endcsname{\color{black}}%
      \expandafter\def\csname LT3\endcsname{\color{black}}%
      \expandafter\def\csname LT4\endcsname{\color{black}}%
      \expandafter\def\csname LT5\endcsname{\color{black}}%
      \expandafter\def\csname LT6\endcsname{\color{black}}%
      \expandafter\def\csname LT7\endcsname{\color{black}}%
      \expandafter\def\csname LT8\endcsname{\color{black}}%
    \fi
  \fi
    \setlength{\unitlength}{0.0500bp}%
    \ifx\gptboxheight\undefined%
      \newlength{\gptboxheight}%
      \newlength{\gptboxwidth}%
      \newsavebox{\gptboxtext}%
    \fi%
    \setlength{\fboxrule}{0.5pt}%
    \setlength{\fboxsep}{1pt}%
\begin{picture}(7180.00,4300.00)%
    \gplgaddtomacro\gplbacktext{%
      \csname LTb\endcsname%%
      \put(728,408){\makebox(0,0)[r]{\strut{}$-4000$}}%
      \csname LTb\endcsname%%
      \put(728,1023){\makebox(0,0)[r]{\strut{}$-2000$}}%
      \csname LTb\endcsname%%
      \put(728,1637){\makebox(0,0)[r]{\strut{}$0$}}%
      \csname LTb\endcsname%%
      \put(728,2252){\makebox(0,0)[r]{\strut{}$2000$}}%
      \csname LTb\endcsname%%
      \put(728,2866){\makebox(0,0)[r]{\strut{}$4000$}}%
      \csname LTb\endcsname%%
      \put(728,3481){\makebox(0,0)[r]{\strut{}$6000$}}%
      \csname LTb\endcsname%%
      \put(728,4095){\makebox(0,0)[r]{\strut{}$8000$}}%
      \csname LTb\endcsname%%
      \put(840,204){\makebox(0,0){\strut{}$-5000$}}%
      \csname LTb\endcsname%%
      \put(1698,204){\makebox(0,0){\strut{}$0$}}%
      \csname LTb\endcsname%%
      \put(2555,204){\makebox(0,0){\strut{}$5000$}}%
      \csname LTb\endcsname%%
      \put(3413,204){\makebox(0,0){\strut{}$10000$}}%
      \csname LTb\endcsname%%
      \put(4270,204){\makebox(0,0){\strut{}$15000$}}%
      \csname LTb\endcsname%%
      \put(5128,204){\makebox(0,0){\strut{}$20000$}}%
      \csname LTb\endcsname%%
      \put(5985,204){\makebox(0,0){\strut{}$25000$}}%
      \csname LTb\endcsname%%
      \put(6843,204){\makebox(0,0){\strut{}$30000$}}%
    }%
    \gplgaddtomacro\gplfronttext{%
      \csname LTb\endcsname%%
      \put(1369,3484){\makebox(0,0)[l]{\strut{}brez hitrosti}}%
      \csname LTb\endcsname%%
      \put(1369,3688){\makebox(0,0)[l]{\strut{}brez lokacija}}%
      \csname LTb\endcsname%%
      \put(1369,3892){\makebox(0,0)[l]{\strut{}kontrola}}%
    }%
    \gplbacktext
    \put(0,0){\includegraphics{graf6}}%
    \gplfronttext
  \end{picture}%
\endgroup

    \caption{Na grafu so z modrim prikazane rešitve za $u_1$ ter z~oranžnim za $u_2$ 
    ob začetnem pogoju $u_0=0,5$ za različne parametre $T$ ter $L$. V zgornji vrsti je 
    $L=1$, $T$ pa $2$ in $4$, v spodnji pa $L=4$ ter isto kot zgoraj zaporedoma $T=2$ in 
    $T=4$.}
    \label{slika5}
\end{figure}


\end{document}
