\documentclass[a4paper,pdftex,12pt]{article}
\usepackage[utf8]{inputenc}
\usepackage[T1]{fontenc} 
\usepackage[english,slovene]{babel} 
\usepackage{amsmath,amsfonts,amsthm,amssymb,mathrsfs,empheq} % Math packages
\usepackage{mathtools}
\usepackage{dsfont}
\usepackage[pdftex]{graphicx}
%\usepackage{makeidx}

\usepackage[version=3]{mhchem} %kemija


\renewcommand{\vec}[1]{\boldsymbol{\mathbf{#1}}}                                        
\newcommand{\ihat}[0]{\boldsymbol{\mathbf{\oldhat{\textbf{\i}}}}} % pokončna j in i (j i n i
\newcommand{\iu}{{i\mkern1mu}}	    %imaginarno število

\usepackage{fancyhdr} % Custom headers and footers
\pagestyle{fancyplain} % Makes all pages in the document conform to the custom headers and footers
\fancyhead{} % No page header - if you want one, create it in the same way as the footers below
\fancyfoot[L]{} % Empty left footer
\fancyfoot[C]{} % Empty center footer
\fancyfoot[R]{\thepage} % Page numbering for right footer
\renewcommand{\headrulewidth}{0pt} % Remove header underlines
\renewcommand{\footrulewidth}{0pt} % Remove footer underlines
\setlength{\headheight}{13.6pt} % Customize the height of the header

%\numberwithin{equation}{section} % Number equations within sections (i.e. 1.1, 1.2, 2.1, 2.2 instead of 1, 2, 3, 4)
\numberwithin{figure}{section} % Number figures within sections (i.e. 1.1, 1.2, 2.1, 2.2 instead of 1, 2, 3, 4)
%\numberwithin{table}{section} % Number tables within sections (i.e. 1.1, 1.2, 2.1, 2.2 instead of 1, 2, 3, 4)

\setlength\parindent{0pt} % Removes all indentation from paragraphs - comment this line for an assignment with lots of text

%----------------------------------------------------------------------------------------
%	TITLE SECTION
%----------------------------------------------------------------------------------------

\newcommand{\horrule}[1]{\rule{\linewidth}{#1}} % Create horizontal rule command with 1 argument of height

\title{	
\normalfont \normalsize 
\textsc{Modelska analiza 1} \\ [25pt] % Your university, school and/or department name(s)
%\horrule{0.2pt} \\[0.4cm] % Thin top horizontal rule
\huge 5. naloga\\ % The assignment title
%\horrule{0.2pt} \\[0.5cm] % Thick bottom horizontal rule
}

\author{Tina Klobas} % Your name

\date{\normalsize\today} % Today's date or a custom date

\begin{document}

\maketitle % Print the title

%----------------------------------------------------------------------------------------
%	PROBLEM 1
%----------------------------------------------------------------------------------------

\section{Binarna reakcija}
Opazujemo reakcijo elementa \ce{A}, pri čemer nastane radikal v~vzbujenem stanju, ki potem
razpade na dva druga elementa. To poteka preko para binarnih enačb:
\begin{align}\label{reakcija1}
    \ce{A + A &<=>[\ce{$p$}][\ce{$q$}] A + A^*}, \\
    \ce{A^* &->[\ce{$r$}] B + C},
\end{align}
\subsection{Eksaktna rešitev}
Za enačbo~\ref{reakcija1} lahko za koncentracijo prisotnih snovi zapišemo gibalne enačbe:
\begin{align}\label{binarna1}
    \dot{\ce{[A]}} &= -p \ce{[A]}^2 + q \ce{[A]} \ce{[A^*]}, \\
    \ce{[\dot{\mathrm{A}}^*]} &= p \ce{[A]}^2 - q \ce{[A]} \ce{[A^*]} -r \ce{[A^*]}, \\
    \dot{\ce{[B]}} &= r \ce{[A^*]}, \\
    \dot{\ce{[C]}} &= r \ce{[A^*]}.
\end{align}
S~parametri $p,q$ in $r$ označimo verjetnosti za prehode. Sistem lahko spravimo 
v~brezdimenzijsko obliko tako, da uvedemo konstanto $A_0 = A(0)+A^*(0)+B(0)+C(0)$ s~katero 
delimo cel sistem~\ref{binarna1}. Z~$\lambda=\frac{r}{qA_0}$, $\mu = \frac{q}{p}$ in
brezdimenzijskim časom $\tau = \frac{t}{A_0 p}$ dobimo sistem:
\begin{align}\label{binarna2}
    \dot{a} &= -a^2 + \mu a a^*, \\
    \dot{a} &= a^2 - \mu a a^* - \lambda \mu a^*, \\
    \dot{b} = \dot{c} &= \lambda \mu a^*,
\end{align}
kjer so konstante $a, a^*, c, d$ oznake za koncentracije snovi deljene s~konstanto $A_0$.
Na grafu~\ref{slika1} smo za primer vzeli začetne vrednosti $a_0=1, a^*=b=c=0$ in konstanti
$\lambda = 1, \mu =1000$. Vidimo, da pri tako velikem razmerju $\mu$ vmesnega stanja
z~vzbujenim elementom A sploh ni in se tako praktično vsa začetna koncentracija \ce{[A]} 
pretvori v~\ce{[B]} in~\ce{[C]}. Na grafu~\ref{slika7} je prikazano obnašanje koncentracij
snovi za različne konstante $\mu$ pri enaki $\lambda=1$. Konstanti $\mu$ ter $\lambda$ tako
predstavljata hitrosti reakcij~\ref{binarna1} in pri njunih velikih vrednostih 
omejimo življenjski čas in vrh koncentracije reaktanta \ce{A^*} ter pospešimo izginjanje 
(nastajanje) \ce{A} (\ce{B} oz. \ce{C}).\\
Na slikah~\ref{slika2} in~\ref{slika3} smo za začetne vrednosti količin vzeli $a_0=1$ in
$a^*=b=c=0$ in pogledali rešitve za različne vrednosti $\lambda=\{0.1, 1, 10\}$ pri
vrednosti $\mu = 1000$. Na desnem grafu~\ref{slika3} je bližje prikazan začetni skok,
ki je pri izbrani konstanti $\mu=1000$ zelo hiter in nizek (pri manjših vrednostih
pa počasnejši in nižji~\ref{slika7}).

\begin{figure}    
    \resizebox{0.98\linewidth}{!}{\input{graf1.tex}}
    \caption{Spreminjanje koncentracij vseh reaktantov in produktov za $\lambda=1$ in $\mu =
    1000$.}
    \label{slika1}
\end{figure}
\begin{figure}    
    \resizebox{0.98\linewidth}{!}{\input{graf7.tex}}
    \caption{Na grafih je prikazano spreminjanje koncentracije (z~leve proti desni) $a$, 
    $a^*$ ter $b$, za različne hitrosti $\mu$.}
    \label{slika7}
\end{figure}

\begin{figure}    
    \resizebox{0.98\linewidth}{!}{\input{graf2.tex}}
    \caption{Na levi je graf spreminjanja količine $a$, na desni pa $b$ oz. $c$ pri 
    različnih vrednostih konstante $\lambda$.}
    \label{slika2}
\end{figure}
\begin{figure}    
    \resizebox{0.98\linewidth}{!}{\input{graf3.tex}}
    \caption{Na grafih je prikazano spreminjanje $a^*$ v~dveh različnih časovnih skalah.}
    \label{slika3}
\end{figure}
\subsection{Stacionarno stanje}
Ker je vrh koncentracije \ce{A^*} pri $\mu=1000$ zelo majhen, njegov življenjski čas pa zelo
kratek, lahko naredimo približek, da je koncentracija $a^*$ ves čas v~stacionarnem stanju;
$\dot{a}^* = 0$,
\begin{align}\label{binarna3}
    a^* &= \frac{a^2}{\mu \left( a + \lambda \right) }, \\
    \dot{a} &= - \frac{\lambda a^2}{\mu \left( a + \lambda \right)}, \\
    \dot{b} = \dot{c} &= \frac{\lambda a^2}{\left( a + \lambda \right) }.
\end{align}
Graf~\ref{slika4} zgleda zelo podoben grafu eksaktne rešitve in graf~\ref{slika5} prikaže,
da je razlika med vrednostmi šele na 4. decimalki. Z~grafa~\ref{slika6} vidimo, da pri
obeh koncentracijah neka točka najmanjše razlike. Na grafu razlik vzbujenega stanja
vidimo, da s~približkom izgubimo začetni skok koncentracije.
neka točka 
\begin{figure}    
    \resizebox{0.9\linewidth}{!}{% GNUPLOT: LaTeX picture with Postscript
\begingroup
  \makeatletter
  \providecommand\color[2][]{%
    \GenericError{(gnuplot) \space\space\space\@spaces}{%
      Package color not loaded in conjunction with
      terminal option `colourtext'%
    }{See the gnuplot documentation for explanation.%
    }{Either use 'blacktext' in gnuplot or load the package
      color.sty in LaTeX.}%
    \renewcommand\color[2][]{}%
  }%
  \providecommand\includegraphics[2][]{%
    \GenericError{(gnuplot) \space\space\space\@spaces}{%
      Package graphicx or graphics not loaded%
    }{See the gnuplot documentation for explanation.%
    }{The gnuplot epslatex terminal needs graphicx.sty or graphics.sty.}%
    \renewcommand\includegraphics[2][]{}%
  }%
  \providecommand\rotatebox[2]{#2}%
  \@ifundefined{ifGPcolor}{%
    \newif\ifGPcolor
    \GPcolortrue
  }{}%
  \@ifundefined{ifGPblacktext}{%
    \newif\ifGPblacktext
    \GPblacktexttrue
  }{}%
  % define a \g@addto@macro without @ in the name:
  \let\gplgaddtomacro\g@addto@macro
  % define empty templates for all commands taking text:
  \gdef\gplbacktext{}%
  \gdef\gplfronttext{}%
  \makeatother
  \ifGPblacktext
    % no textcolor at all
    \def\colorrgb#1{}%
    \def\colorgray#1{}%
  \else
    % gray or color?
    \ifGPcolor
      \def\colorrgb#1{\color[rgb]{#1}}%
      \def\colorgray#1{\color[gray]{#1}}%
      \expandafter\def\csname LTw\endcsname{\color{white}}%
      \expandafter\def\csname LTb\endcsname{\color{black}}%
      \expandafter\def\csname LTa\endcsname{\color{black}}%
      \expandafter\def\csname LT0\endcsname{\color[rgb]{1,0,0}}%
      \expandafter\def\csname LT1\endcsname{\color[rgb]{0,1,0}}%
      \expandafter\def\csname LT2\endcsname{\color[rgb]{0,0,1}}%
      \expandafter\def\csname LT3\endcsname{\color[rgb]{1,0,1}}%
      \expandafter\def\csname LT4\endcsname{\color[rgb]{0,1,1}}%
      \expandafter\def\csname LT5\endcsname{\color[rgb]{1,1,0}}%
      \expandafter\def\csname LT6\endcsname{\color[rgb]{0,0,0}}%
      \expandafter\def\csname LT7\endcsname{\color[rgb]{1,0.3,0}}%
      \expandafter\def\csname LT8\endcsname{\color[rgb]{0.5,0.5,0.5}}%
    \else
      % gray
      \def\colorrgb#1{\color{black}}%
      \def\colorgray#1{\color[gray]{#1}}%
      \expandafter\def\csname LTw\endcsname{\color{white}}%
      \expandafter\def\csname LTb\endcsname{\color{black}}%
      \expandafter\def\csname LTa\endcsname{\color{black}}%
      \expandafter\def\csname LT0\endcsname{\color{black}}%
      \expandafter\def\csname LT1\endcsname{\color{black}}%
      \expandafter\def\csname LT2\endcsname{\color{black}}%
      \expandafter\def\csname LT3\endcsname{\color{black}}%
      \expandafter\def\csname LT4\endcsname{\color{black}}%
      \expandafter\def\csname LT5\endcsname{\color{black}}%
      \expandafter\def\csname LT6\endcsname{\color{black}}%
      \expandafter\def\csname LT7\endcsname{\color{black}}%
      \expandafter\def\csname LT8\endcsname{\color{black}}%
    \fi
  \fi
    \setlength{\unitlength}{0.0500bp}%
    \ifx\gptboxheight\undefined%
      \newlength{\gptboxheight}%
      \newlength{\gptboxwidth}%
      \newsavebox{\gptboxtext}%
    \fi%
    \setlength{\fboxrule}{0.5pt}%
    \setlength{\fboxsep}{1pt}%
\begin{picture}(7200.00,4320.00)%
    \gplgaddtomacro\gplbacktext{%
      \csname LTb\endcsname%%
      \put(708,652){\makebox(0,0)[r]{\strut{}$0$}}%
      \csname LTb\endcsname%%
      \put(708,996){\makebox(0,0)[r]{\strut{}$0.1$}}%
      \csname LTb\endcsname%%
      \put(708,1341){\makebox(0,0)[r]{\strut{}$0.2$}}%
      \csname LTb\endcsname%%
      \put(708,1685){\makebox(0,0)[r]{\strut{}$0.3$}}%
      \csname LTb\endcsname%%
      \put(708,2029){\makebox(0,0)[r]{\strut{}$0.4$}}%
      \csname LTb\endcsname%%
      \put(708,2374){\makebox(0,0)[r]{\strut{}$0.5$}}%
      \csname LTb\endcsname%%
      \put(708,2718){\makebox(0,0)[r]{\strut{}$0.6$}}%
      \csname LTb\endcsname%%
      \put(708,3062){\makebox(0,0)[r]{\strut{}$0.7$}}%
      \csname LTb\endcsname%%
      \put(708,3406){\makebox(0,0)[r]{\strut{}$0.8$}}%
      \csname LTb\endcsname%%
      \put(708,3751){\makebox(0,0)[r]{\strut{}$0.9$}}%
      \csname LTb\endcsname%%
      \put(708,4095){\makebox(0,0)[r]{\strut{}$1$}}%
      \csname LTb\endcsname%%
      \put(820,448){\makebox(0,0){\strut{}$0$}}%
      \csname LTb\endcsname%%
      \put(1996,448){\makebox(0,0){\strut{}$100$}}%
      \csname LTb\endcsname%%
      \put(3173,448){\makebox(0,0){\strut{}$200$}}%
      \csname LTb\endcsname%%
      \put(4349,448){\makebox(0,0){\strut{}$300$}}%
      \csname LTb\endcsname%%
      \put(5525,448){\makebox(0,0){\strut{}$400$}}%
      \csname LTb\endcsname%%
      \put(6702,448){\makebox(0,0){\strut{}$500$}}%
    }%
    \gplgaddtomacro\gplfronttext{%
      \csname LTb\endcsname%%
      \put(186,2373){\rotatebox{-270}{\makebox(0,0){\strut{}amplituda}}}%
      \csname LTb\endcsname%%
      \put(3831,142){\makebox(0,0){\strut{}$x$}}%
      \csname LTb\endcsname%%
      \put(1349,3280){\makebox(0,0)[l]{\strut{}Welch}}%
      \csname LTb\endcsname%%
      \put(1349,3484){\makebox(0,0)[l]{\strut{}Hann}}%
      \csname LTb\endcsname%%
      \put(1349,3688){\makebox(0,0)[l]{\strut{}Bartlett}}%
      \csname LTb\endcsname%%
      \put(1349,3892){\makebox(0,0)[l]{\strut{}eksponent}}%
    }%
    \gplbacktext
    \put(0,0){\includegraphics[width={360.00bp},height={216.00bp}]{graf4}}%
    \gplfronttext
  \end{picture}%
\endgroup
}
    \caption{Prikazano je spreminjanje koncentracij snovi s~časom pri približku 
    stacinoarnega stanja \ce{A^*}, za konstanti $\lambda = 1$ in $\mu = 1000$.}
    \label{slika4}
\end{figure}
\begin{figure}    
    \hspace{1cm}\resizebox{0.9\linewidth}{!}{% GNUPLOT: LaTeX picture with Postscript
\begingroup
  \makeatletter
  \providecommand\color[2][]{%
    \GenericError{(gnuplot) \space\space\space\@spaces}{%
      Package color not loaded in conjunction with
      terminal option `colourtext'%
    }{See the gnuplot documentation for explanation.%
    }{Either use 'blacktext' in gnuplot or load the package
      color.sty in LaTeX.}%
    \renewcommand\color[2][]{}%
  }%
  \providecommand\includegraphics[2][]{%
    \GenericError{(gnuplot) \space\space\space\@spaces}{%
      Package graphicx or graphics not loaded%
    }{See the gnuplot documentation for explanation.%
    }{The gnuplot epslatex terminal needs graphicx.sty or graphics.sty.}%
    \renewcommand\includegraphics[2][]{}%
  }%
  \providecommand\rotatebox[2]{#2}%
  \@ifundefined{ifGPcolor}{%
    \newif\ifGPcolor
    \GPcolortrue
  }{}%
  \@ifundefined{ifGPblacktext}{%
    \newif\ifGPblacktext
    \GPblacktexttrue
  }{}%
  % define a \g@addto@macro without @ in the name:
  \let\gplgaddtomacro\g@addto@macro
  % define empty templates for all commands taking text:
  \gdef\gplbacktext{}%
  \gdef\gplfronttext{}%
  \makeatother
  \ifGPblacktext
    % no textcolor at all
    \def\colorrgb#1{}%
    \def\colorgray#1{}%
  \else
    % gray or color?
    \ifGPcolor
      \def\colorrgb#1{\color[rgb]{#1}}%
      \def\colorgray#1{\color[gray]{#1}}%
      \expandafter\def\csname LTw\endcsname{\color{white}}%
      \expandafter\def\csname LTb\endcsname{\color{black}}%
      \expandafter\def\csname LTa\endcsname{\color{black}}%
      \expandafter\def\csname LT0\endcsname{\color[rgb]{1,0,0}}%
      \expandafter\def\csname LT1\endcsname{\color[rgb]{0,1,0}}%
      \expandafter\def\csname LT2\endcsname{\color[rgb]{0,0,1}}%
      \expandafter\def\csname LT3\endcsname{\color[rgb]{1,0,1}}%
      \expandafter\def\csname LT4\endcsname{\color[rgb]{0,1,1}}%
      \expandafter\def\csname LT5\endcsname{\color[rgb]{1,1,0}}%
      \expandafter\def\csname LT6\endcsname{\color[rgb]{0,0,0}}%
      \expandafter\def\csname LT7\endcsname{\color[rgb]{1,0.3,0}}%
      \expandafter\def\csname LT8\endcsname{\color[rgb]{0.5,0.5,0.5}}%
    \else
      % gray
      \def\colorrgb#1{\color{black}}%
      \def\colorgray#1{\color[gray]{#1}}%
      \expandafter\def\csname LTw\endcsname{\color{white}}%
      \expandafter\def\csname LTb\endcsname{\color{black}}%
      \expandafter\def\csname LTa\endcsname{\color{black}}%
      \expandafter\def\csname LT0\endcsname{\color{black}}%
      \expandafter\def\csname LT1\endcsname{\color{black}}%
      \expandafter\def\csname LT2\endcsname{\color{black}}%
      \expandafter\def\csname LT3\endcsname{\color{black}}%
      \expandafter\def\csname LT4\endcsname{\color{black}}%
      \expandafter\def\csname LT5\endcsname{\color{black}}%
      \expandafter\def\csname LT6\endcsname{\color{black}}%
      \expandafter\def\csname LT7\endcsname{\color{black}}%
      \expandafter\def\csname LT8\endcsname{\color{black}}%
    \fi
  \fi
    \setlength{\unitlength}{0.0500bp}%
    \ifx\gptboxheight\undefined%
      \newlength{\gptboxheight}%
      \newlength{\gptboxwidth}%
      \newsavebox{\gptboxtext}%
    \fi%
    \setlength{\fboxrule}{0.5pt}%
    \setlength{\fboxsep}{1pt}%
\begin{picture}(7180.00,4300.00)%
    \gplgaddtomacro\gplbacktext{%
      \csname LTb\endcsname%%
      \put(504,408){\makebox(0,0)[r]{\strut{}$0$}}%
      \csname LTb\endcsname%%
      \put(504,1023){\makebox(0,0)[r]{\strut{}$20$}}%
      \csname LTb\endcsname%%
      \put(504,1637){\makebox(0,0)[r]{\strut{}$40$}}%
      \csname LTb\endcsname%%
      \put(504,2252){\makebox(0,0)[r]{\strut{}$60$}}%
      \csname LTb\endcsname%%
      \put(504,2866){\makebox(0,0)[r]{\strut{}$80$}}%
      \csname LTb\endcsname%%
      \put(504,3481){\makebox(0,0)[r]{\strut{}$100$}}%
      \csname LTb\endcsname%%
      \put(504,4095){\makebox(0,0)[r]{\strut{}$120$}}%
      \csname LTb\endcsname%%
      \put(616,204){\makebox(0,0){\strut{}$0$}}%
      \csname LTb\endcsname%%
      \put(1861,204){\makebox(0,0){\strut{}$500$}}%
      \csname LTb\endcsname%%
      \put(3107,204){\makebox(0,0){\strut{}$1000$}}%
      \csname LTb\endcsname%%
      \put(4352,204){\makebox(0,0){\strut{}$1500$}}%
      \csname LTb\endcsname%%
      \put(5598,204){\makebox(0,0){\strut{}$2000$}}%
      \csname LTb\endcsname%%
      \put(6843,204){\makebox(0,0){\strut{}$2500$}}%
    }%
    \gplgaddtomacro\gplfronttext{%
      \csname LTb\endcsname%%
      \put(3931,3688){\makebox(0,0)[l]{\strut{}$B=$ vsaka $5./10.$ meritev}}%
      \csname LTb\endcsname%%
      \put(3931,3892){\makebox(0,0)[l]{\strut{}$A=$ vsaka meritev}}%
      \csname LTb\endcsname%%
      \put(3730,1637){\makebox(0,0)[l]{\strut{}$\mu_A$}}%
      \csname LTb\endcsname%%
      \put(3730,1483){\makebox(0,0)[l]{\strut{}$\mu_B$}}%
    }%
    \gplbacktext
    \put(0,0){\includegraphics{graf5}}%
    \gplfronttext
  \end{picture}%
\endgroup
}
    \caption{Logaritem razlike med eksaktno rešitvijo in približkom stacionarnega stanja,
    v~različnih časovnih skalah.}
    \label{slika5}
\end{figure}
\begin{figure}    
    \hspace{1cm}\resizebox{0.9\linewidth}{!}{% GNUPLOT: LaTeX picture with Postscript
\begingroup
  \makeatletter
  \providecommand\color[2][]{%
    \GenericError{(gnuplot) \space\space\space\@spaces}{%
      Package color not loaded in conjunction with
      terminal option `colourtext'%
    }{See the gnuplot documentation for explanation.%
    }{Either use 'blacktext' in gnuplot or load the package
      color.sty in LaTeX.}%
    \renewcommand\color[2][]{}%
  }%
  \providecommand\includegraphics[2][]{%
    \GenericError{(gnuplot) \space\space\space\@spaces}{%
      Package graphicx or graphics not loaded%
    }{See the gnuplot documentation for explanation.%
    }{The gnuplot epslatex terminal needs graphicx.sty or graphics.sty.}%
    \renewcommand\includegraphics[2][]{}%
  }%
  \providecommand\rotatebox[2]{#2}%
  \@ifundefined{ifGPcolor}{%
    \newif\ifGPcolor
    \GPcolortrue
  }{}%
  \@ifundefined{ifGPblacktext}{%
    \newif\ifGPblacktext
    \GPblacktexttrue
  }{}%
  % define a \g@addto@macro without @ in the name:
  \let\gplgaddtomacro\g@addto@macro
  % define empty templates for all commands taking text:
  \gdef\gplbacktext{}%
  \gdef\gplfronttext{}%
  \makeatother
  \ifGPblacktext
    % no textcolor at all
    \def\colorrgb#1{}%
    \def\colorgray#1{}%
  \else
    % gray or color?
    \ifGPcolor
      \def\colorrgb#1{\color[rgb]{#1}}%
      \def\colorgray#1{\color[gray]{#1}}%
      \expandafter\def\csname LTw\endcsname{\color{white}}%
      \expandafter\def\csname LTb\endcsname{\color{black}}%
      \expandafter\def\csname LTa\endcsname{\color{black}}%
      \expandafter\def\csname LT0\endcsname{\color[rgb]{1,0,0}}%
      \expandafter\def\csname LT1\endcsname{\color[rgb]{0,1,0}}%
      \expandafter\def\csname LT2\endcsname{\color[rgb]{0,0,1}}%
      \expandafter\def\csname LT3\endcsname{\color[rgb]{1,0,1}}%
      \expandafter\def\csname LT4\endcsname{\color[rgb]{0,1,1}}%
      \expandafter\def\csname LT5\endcsname{\color[rgb]{1,1,0}}%
      \expandafter\def\csname LT6\endcsname{\color[rgb]{0,0,0}}%
      \expandafter\def\csname LT7\endcsname{\color[rgb]{1,0.3,0}}%
      \expandafter\def\csname LT8\endcsname{\color[rgb]{0.5,0.5,0.5}}%
    \else
      % gray
      \def\colorrgb#1{\color{black}}%
      \def\colorgray#1{\color[gray]{#1}}%
      \expandafter\def\csname LTw\endcsname{\color{white}}%
      \expandafter\def\csname LTb\endcsname{\color{black}}%
      \expandafter\def\csname LTa\endcsname{\color{black}}%
      \expandafter\def\csname LT0\endcsname{\color{black}}%
      \expandafter\def\csname LT1\endcsname{\color{black}}%
      \expandafter\def\csname LT2\endcsname{\color{black}}%
      \expandafter\def\csname LT3\endcsname{\color{black}}%
      \expandafter\def\csname LT4\endcsname{\color{black}}%
      \expandafter\def\csname LT5\endcsname{\color{black}}%
      \expandafter\def\csname LT6\endcsname{\color{black}}%
      \expandafter\def\csname LT7\endcsname{\color{black}}%
      \expandafter\def\csname LT8\endcsname{\color{black}}%
    \fi
  \fi
    \setlength{\unitlength}{0.0500bp}%
    \ifx\gptboxheight\undefined%
      \newlength{\gptboxheight}%
      \newlength{\gptboxwidth}%
      \newsavebox{\gptboxtext}%
    \fi%
    \setlength{\fboxrule}{0.5pt}%
    \setlength{\fboxsep}{1pt}%
\begin{picture}(7180.00,4300.00)%
    \gplgaddtomacro\gplbacktext{%
      \csname LTb\endcsname%%
      \put(728,408){\makebox(0,0)[r]{\strut{}$-4000$}}%
      \csname LTb\endcsname%%
      \put(728,1023){\makebox(0,0)[r]{\strut{}$-2000$}}%
      \csname LTb\endcsname%%
      \put(728,1637){\makebox(0,0)[r]{\strut{}$0$}}%
      \csname LTb\endcsname%%
      \put(728,2252){\makebox(0,0)[r]{\strut{}$2000$}}%
      \csname LTb\endcsname%%
      \put(728,2866){\makebox(0,0)[r]{\strut{}$4000$}}%
      \csname LTb\endcsname%%
      \put(728,3481){\makebox(0,0)[r]{\strut{}$6000$}}%
      \csname LTb\endcsname%%
      \put(728,4095){\makebox(0,0)[r]{\strut{}$8000$}}%
      \csname LTb\endcsname%%
      \put(840,204){\makebox(0,0){\strut{}$-5000$}}%
      \csname LTb\endcsname%%
      \put(1698,204){\makebox(0,0){\strut{}$0$}}%
      \csname LTb\endcsname%%
      \put(2555,204){\makebox(0,0){\strut{}$5000$}}%
      \csname LTb\endcsname%%
      \put(3413,204){\makebox(0,0){\strut{}$10000$}}%
      \csname LTb\endcsname%%
      \put(4270,204){\makebox(0,0){\strut{}$15000$}}%
      \csname LTb\endcsname%%
      \put(5128,204){\makebox(0,0){\strut{}$20000$}}%
      \csname LTb\endcsname%%
      \put(5985,204){\makebox(0,0){\strut{}$25000$}}%
      \csname LTb\endcsname%%
      \put(6843,204){\makebox(0,0){\strut{}$30000$}}%
    }%
    \gplgaddtomacro\gplfronttext{%
      \csname LTb\endcsname%%
      \put(1369,3484){\makebox(0,0)[l]{\strut{}brez hitrosti}}%
      \csname LTb\endcsname%%
      \put(1369,3688){\makebox(0,0)[l]{\strut{}brez lokacija}}%
      \csname LTb\endcsname%%
      \put(1369,3892){\makebox(0,0)[l]{\strut{}kontrola}}%
    }%
    \gplbacktext
    \put(0,0){\includegraphics{graf6}}%
    \gplfronttext
  \end{picture}%
\endgroup
}
    \caption{Logaritem razlike med eksaktno rešitvijo in približkom stacionarnega stanja,
    v~različnih odvisnosti od logaritma časa. Na levi je graf $a$, na desni pa $a^*$.}
    \label{slika6}
\end{figure}

%----------------------------------------------------------------------------------------
%	PROBLEM 2
%----------------------------------------------------------------------------------------
\section{Sinteza bromovodikove kisline}
Opazujemo reakcijo
$$ \ce{H2 + Br2 <=> 2HBr},$$
ki jo lahko zapišemo kot sistem binarnih reakcij:
\begin{align}\label{sinteza}
    \ce{Br2 &<=>[\ce{$p_1$}][\ce{$q_1$}] 2Br}, \\
    \ce{Br + H2 &<=>[\ce{$p_2$}][\ce{$q_2$}] HBr + H}, \\
    \ce{H + Br2 &->[\ce{$p_3$}] HBr + Br}.
\end{align}
Zopet lahko zapišemo enačbe za kinematiko sistema, kjer zapišemo koncentracije \ce{H}, 
\ce{H2}, \ce{Br}, \ce{Br2} in \ce{HBr} z~oznakami $H, h, B, b$ in $\chi$. Že zdaj
upoštevamo, da so vmesne reakcije v~stacionarnem stanju  tako da dobimo 
sistem enačb:
\begin{align*}
    \dot{B} = \dot{H} &= 0, \\
    \dot{b} &= -p_1 b + q_1 B^2 - p_3 H b, \\
    \dot{h} &= -p_2 B h + q_2 \chi H, \\
    \dot{\chi} &= p_2 B h -q_2 \chi H + p_3 H b.
\end{align*}
Iz sistema sledi
\begin{align*}
    B &= \sqrt{\frac{p_1}{q_1}b}, \\
    H &= p_2 h \frac{\sqrt{\frac{p_1}{q_1}b}}{p_3b + q_2\chi}, \\
    \dot{\chi} &= \frac{k h \sqrt{b}}{m + \frac{\chi}{b}}, \\
    \dot{b} = \dot{h} &= - \frac{1}{2}\frac{k h \sqrt{b}}{m + \frac{\chi}{b}},
\end{align*}
kjer vpeljemo konstanti $m=\frac{p_3}{q_2}$ ter $k=\frac{1p_2p_3}{q_2}\sqrt{\frac{p_1}{q_1}}
$. Obe konstanti bi v~praksi lahko izmerili tako, da bi merili koncentracijo \ce{[Hbr]} in
tako dobljen potek \emph{fittali} na empirično enačbo.
Vse enačbe koncentracije normiramo na začetno vrednost $A_0 = h(0) + b(0) + \chi (0)$
tako da je $\chi/A_0 = x$, $h/A_0 = y$ in $b/A_0 = z$. Vpeljemo še brezdimenzijski čas 
$\tau = k \sqrt{A_0}$ in tako dobimo brezdimenzijski sistem
\begin{align}\label{brezdimHBr}
    \dot{x} &= \frac{y \sqrt{z}}{m + \frac{x}{z}}, \\
    \dot{y} = \dot{z} &= - \frac{1}{2} \frac{y \sqrt{z}}{m + \frac{x}{z}}.
\end{align}
Iz enačbe~\ref{brezdimHBr} vidimo, da je vsota $\dot{x} + \dot{y} + \dot{z} =1$ in je potem
celotna koncentracija $A(t)$ ves čas konstantna $A=A_0$.\\
Na grafu~\ref{slika8} je prikazan primer poteka reakcije. Vidimo, da koncentracija $x$ raste
dvakrat hitreje kot drugi dve padata, kar očitno tudi iz enačb~\ref{brezdimHBr}. Vidimo
tudi, da je celotna koncentracija $A$, res konstantna. Na sliki~\ref{slika9} je prikazana
odvisnost poteka reakcije glede na začetno razmerje $\ce{H}/\ce{Br}$. Vidimo, da razmerje
ne vpliva na hitrost reakcije (kar je vidno tudi po enačbah). Opazimo lahko tudi, da ko je
razmerje manjše od ena (oziroma $y$ večja od $z$), končna koncentracija $x$ ne doseže
začetne vrednosti $y$, saj prehitro zmanjka reaktanta $z$.\\
Če v~začetno stanje primešamo veliko koncentracije produkta $x$, se njegov časovni potek 
začne počasi gladiti in izgubljati zadnji zavoj, namreč ves čas mora veljati, da je
vsota vseh treh koncentracij konstantna. Hkrati začne koncentracija reaktantov vedno
manj \emph{zaobljeno} padati in postaja hitrost spremembe vedno večja in postaja prehod
vedno ožji, kar je tudi prikazano na slikah~\ref{slika10}.
\begin{figure}    
    \hspace{1cm}\resizebox{0.9\linewidth}{!}{\input{graf8.tex}}
    \caption{Primer poteka koncetracij \ce{H2}, \ce{Br2} in \ce{HBr} za $m=2,5$ in začetno 
    razmerje koncentracij $\ce{H}/\ce{Br}=0,5$.}
    \label{slika8}
\end{figure}
\begin{figure}    
    \hspace{1cm}\resizebox{0.9\linewidth}{!}{\input{graf9.tex}}
    \caption{Primer poteka koncetracij \ce{H2}, \ce{Br2} in \ce{HBr} za $m=2,5$ in začetno 
    razmerje koncentracij $\ce{H}/\ce{Br}=0,5$.}
    \label{slika9}
\end{figure}
\begin{figure}    
    \hspace{1cm}\resizebox{0.9\linewidth}{!}{% GNUPLOT: LaTeX picture with Postscript
\begingroup
  \makeatletter
  \providecommand\color[2][]{%
    \GenericError{(gnuplot) \space\space\space\@spaces}{%
      Package color not loaded in conjunction with
      terminal option `colourtext'%
    }{See the gnuplot documentation for explanation.%
    }{Either use 'blacktext' in gnuplot or load the package
      color.sty in LaTeX.}%
    \renewcommand\color[2][]{}%
  }%
  \providecommand\includegraphics[2][]{%
    \GenericError{(gnuplot) \space\space\space\@spaces}{%
      Package graphicx or graphics not loaded%
    }{See the gnuplot documentation for explanation.%
    }{The gnuplot epslatex terminal needs graphicx.sty or graphics.sty.}%
    \renewcommand\includegraphics[2][]{}%
  }%
  \providecommand\rotatebox[2]{#2}%
  \@ifundefined{ifGPcolor}{%
    \newif\ifGPcolor
    \GPcolortrue
  }{}%
  \@ifundefined{ifGPblacktext}{%
    \newif\ifGPblacktext
    \GPblacktexttrue
  }{}%
  % define a \g@addto@macro without @ in the name:
  \let\gplgaddtomacro\g@addto@macro
  % define empty templates for all commands taking text:
  \gdef\gplbacktext{}%
  \gdef\gplfronttext{}%
  \makeatother
  \ifGPblacktext
    % no textcolor at all
    \def\colorrgb#1{}%
    \def\colorgray#1{}%
  \else
    % gray or color?
    \ifGPcolor
      \def\colorrgb#1{\color[rgb]{#1}}%
      \def\colorgray#1{\color[gray]{#1}}%
      \expandafter\def\csname LTw\endcsname{\color{white}}%
      \expandafter\def\csname LTb\endcsname{\color{black}}%
      \expandafter\def\csname LTa\endcsname{\color{black}}%
      \expandafter\def\csname LT0\endcsname{\color[rgb]{1,0,0}}%
      \expandafter\def\csname LT1\endcsname{\color[rgb]{0,1,0}}%
      \expandafter\def\csname LT2\endcsname{\color[rgb]{0,0,1}}%
      \expandafter\def\csname LT3\endcsname{\color[rgb]{1,0,1}}%
      \expandafter\def\csname LT4\endcsname{\color[rgb]{0,1,1}}%
      \expandafter\def\csname LT5\endcsname{\color[rgb]{1,1,0}}%
      \expandafter\def\csname LT6\endcsname{\color[rgb]{0,0,0}}%
      \expandafter\def\csname LT7\endcsname{\color[rgb]{1,0.3,0}}%
      \expandafter\def\csname LT8\endcsname{\color[rgb]{0.5,0.5,0.5}}%
    \else
      % gray
      \def\colorrgb#1{\color{black}}%
      \def\colorgray#1{\color[gray]{#1}}%
      \expandafter\def\csname LTw\endcsname{\color{white}}%
      \expandafter\def\csname LTb\endcsname{\color{black}}%
      \expandafter\def\csname LTa\endcsname{\color{black}}%
      \expandafter\def\csname LT0\endcsname{\color{black}}%
      \expandafter\def\csname LT1\endcsname{\color{black}}%
      \expandafter\def\csname LT2\endcsname{\color{black}}%
      \expandafter\def\csname LT3\endcsname{\color{black}}%
      \expandafter\def\csname LT4\endcsname{\color{black}}%
      \expandafter\def\csname LT5\endcsname{\color{black}}%
      \expandafter\def\csname LT6\endcsname{\color{black}}%
      \expandafter\def\csname LT7\endcsname{\color{black}}%
      \expandafter\def\csname LT8\endcsname{\color{black}}%
    \fi
  \fi
    \setlength{\unitlength}{0.0500bp}%
    \ifx\gptboxheight\undefined%
      \newlength{\gptboxheight}%
      \newlength{\gptboxwidth}%
      \newsavebox{\gptboxtext}%
    \fi%
    \setlength{\fboxrule}{0.5pt}%
    \setlength{\fboxsep}{1pt}%
\begin{picture}(7200.00,4320.00)%
    \gplgaddtomacro\gplbacktext{%
      \csname LTb\endcsname%%
      \put(504,408){\makebox(0,0)[r]{\strut{}$0$}}%
      \csname LTb\endcsname%%
      \put(504,954){\makebox(0,0)[r]{\strut{}$0.1$}}%
      \csname LTb\endcsname%%
      \put(504,1501){\makebox(0,0)[r]{\strut{}$0.2$}}%
      \csname LTb\endcsname%%
      \put(504,2048){\makebox(0,0)[r]{\strut{}$0.3$}}%
      \csname LTb\endcsname%%
      \put(504,2594){\makebox(0,0)[r]{\strut{}$0.4$}}%
      \csname LTb\endcsname%%
      \put(504,3140){\makebox(0,0)[r]{\strut{}$0.5$}}%
      \csname LTb\endcsname%%
      \put(504,3687){\makebox(0,0)[r]{\strut{}$0.6$}}%
      \csname LTb\endcsname%%
      \put(616,204){\makebox(0,0){\strut{}$0$}}%
      \csname LTb\endcsname%%
      \put(1239,204){\makebox(0,0){\strut{}$1$}}%
      \csname LTb\endcsname%%
      \put(1861,204){\makebox(0,0){\strut{}$2$}}%
      \csname LTb\endcsname%%
      \put(2484,204){\makebox(0,0){\strut{}$3$}}%
      \csname LTb\endcsname%%
      \put(3107,204){\makebox(0,0){\strut{}$4$}}%
      \csname LTb\endcsname%%
      \put(3730,204){\makebox(0,0){\strut{}$5$}}%
      \csname LTb\endcsname%%
      \put(4352,204){\makebox(0,0){\strut{}$6$}}%
      \csname LTb\endcsname%%
      \put(4975,204){\makebox(0,0){\strut{}$7$}}%
      \csname LTb\endcsname%%
      \put(5598,204){\makebox(0,0){\strut{}$8$}}%
      \csname LTb\endcsname%%
      \put(6220,204){\makebox(0,0){\strut{}$9$}}%
      \csname LTb\endcsname%%
      \put(6843,204){\makebox(0,0){\strut{}$10$}}%
    }%
    \gplgaddtomacro\gplfronttext{%
      \csname LTb\endcsname%%
      \put(5163,2382){\makebox(0,0)[l]{\strut{}$ \Delta t = 0.001$}}%
      \csname LTb\endcsname%%
      \put(5163,2586){\makebox(0,0)[l]{\strut{}$ \Delta t = 0.01$}}%
      \csname LTb\endcsname%%
      \put(5163,2790){\makebox(0,0)[l]{\strut{}$ \Delta t = 0.1$}}%
      \csname LTb\endcsname%%
      \put(5163,2994){\makebox(0,0)[l]{\strut{}$ \Delta t = 0.25$}}%
      \csname LTb\endcsname%%
      \put(5163,3198){\makebox(0,0)[l]{\strut{}$ \Delta t = 0.5$}}%
      \csname LTb\endcsname%%
      \put(5163,3402){\makebox(0,0)[l]{\strut{}$ \Delta t = 1$}}%
      \csname LTb\endcsname%%
      \put(3729,3993){\makebox(0,0){\strut{}$N_0=25$}}%
    }%
    \gplbacktext
    \put(0,0){\includegraphics[width={360.00bp},height={216.00bp}]{graf10}}%
    \gplfronttext
  \end{picture}%
\endgroup
}
    \caption{Primer, če primešamo v~začetno stanje tudi nekaj produkta.}
    \label{slika10}
\end{figure}


%----------------------------------------------------------------------------------------
%	PROBLEM 3
%----------------------------------------------------------------------------------------
\section{Kemijske ure}
Opazujemo ravnovesje para reakcij
\begin{align*}
    \ce{S2O8^2- + 2I- &-> I2 + 2SO4^2-}, \\
    \ce{2S2O3^2- + I2 &-> 2I^- + S4O6^2-}.
\end{align*}
Poenostavimo obe reakciji na dve stopnji od katerih je ena hitra ($q$) druga pa počasna 
($s$)
\begin{align*}
    \ce{S2O8^2- + I^- &->[\ce{$q_1$}] IS2O3- + I^-}, \\
    \ce{IS2O8^3- + I^- &->[\ce{$s_1$}] I2 + 2SO4^2-} \mathrm{in} \\
    \ce{S2O3^2- + I2 &->[\ce{$q_2$}] IS2O3- + I^-}, \\
    \ce{IS2O3^3- + S2O3^2- &->[\ce{$s_2$}] I^- + S4O6^2-}.
\end{align*}
Opazujemo koncentracije \ce{[I^-]}, \ce{[I2]}, \ce{[S2O3^2-]} in \ce{[S2O8^2-]} in zapišemo
enačbe dinamike zanje. Sistema ne bomo pisali, vendar če privzamemo še, da so koncentracije
vmesnih reaktantov \ce{[IS2O8^3-]} in \ce{[SO4^2-]} v~stacionarnem stanju, če vse
konstante delimo z~vsoto začetnih koncentracij $A$ in vpeljemo brezdimenzijski čas $\tau =
s_1 t$ dobimo končni sistem
\begin{align}\label{ura}
    \dot{x} = -2 x w + \lambda2 y z, \\
    \dot{y} = x w - \lambda y z, \\
    \dot{z} = - 2 \lambda y z, \\
    \dot{w} = - x w, 
\end{align}
kjer so brezdimenzijske koncentracije $x,w,y,z$ po vrsti normirane \ce{[I^-]}, \ce{[I2]}, 
\ce{[S2O3^2-]} in \ce{[S2O8^2-]}, $\lambda$ pa razmerje $q_2/q_1$. Na grafu~\ref{slika10}
pri zelo veliki $\lambda=5000$ opazujemo spreminjanje koncentracij. Če imamo na začetku
enaki koncentraciji joda \ce{I-} in tiosulfata, bo prvi konstanten drugi pa se bo porabljal
in spreminjal v~produkte, če pa zmanjšamo koncentracijo začetnega \ce{S2O8^{2-}} bo
koncentracija \ce{I-} konstantna dokler se tiosulfat ne bo porabil in takrat se bo začela
spreminjat še koncentracija \ce{I}. V~tem trenutku ko se tioslfat porabi se porabljanje
joda pozna v~obarvanju zmesi. Na grafu~\ref{slika11} vidimo, da manjšanje konstante 
$\lambda$ povzroči to, da vse koncentracije prej preidejo iz ravnovesja in se začnejo
porabljati ali pa nastajati. To tudi vidimo po enačbah~\ref{ura} saj pri velikih
$\lambda$ prevladajo le eni členi ($\lambda z y$) medtem ko nižamo to razmerje pridemo
v~območje kjer obe komponenti enako doprineseta, če bi pa razmerje še manjšali
bi člen $xw$ prevladal nad drugim. Z~večanjem $\lambda$ postaja prehod tudi veliko hitrejši
in ostrejši.
\begin{figure}    
    \hspace{1cm}\resizebox{0.9\linewidth}{!}{% GNUPLOT: LaTeX picture with Postscript
\begingroup
  \makeatletter
  \providecommand\color[2][]{%
    \GenericError{(gnuplot) \space\space\space\@spaces}{%
      Package color not loaded in conjunction with
      terminal option `colourtext'%
    }{See the gnuplot documentation for explanation.%
    }{Either use 'blacktext' in gnuplot or load the package
      color.sty in LaTeX.}%
    \renewcommand\color[2][]{}%
  }%
  \providecommand\includegraphics[2][]{%
    \GenericError{(gnuplot) \space\space\space\@spaces}{%
      Package graphicx or graphics not loaded%
    }{See the gnuplot documentation for explanation.%
    }{The gnuplot epslatex terminal needs graphicx.sty or graphics.sty.}%
    \renewcommand\includegraphics[2][]{}%
  }%
  \providecommand\rotatebox[2]{#2}%
  \@ifundefined{ifGPcolor}{%
    \newif\ifGPcolor
    \GPcolortrue
  }{}%
  \@ifundefined{ifGPblacktext}{%
    \newif\ifGPblacktext
    \GPblacktexttrue
  }{}%
  % define a \g@addto@macro without @ in the name:
  \let\gplgaddtomacro\g@addto@macro
  % define empty templates for all commands taking text:
  \gdef\gplbacktext{}%
  \gdef\gplfronttext{}%
  \makeatother
  \ifGPblacktext
    % no textcolor at all
    \def\colorrgb#1{}%
    \def\colorgray#1{}%
  \else
    % gray or color?
    \ifGPcolor
      \def\colorrgb#1{\color[rgb]{#1}}%
      \def\colorgray#1{\color[gray]{#1}}%
      \expandafter\def\csname LTw\endcsname{\color{white}}%
      \expandafter\def\csname LTb\endcsname{\color{black}}%
      \expandafter\def\csname LTa\endcsname{\color{black}}%
      \expandafter\def\csname LT0\endcsname{\color[rgb]{1,0,0}}%
      \expandafter\def\csname LT1\endcsname{\color[rgb]{0,1,0}}%
      \expandafter\def\csname LT2\endcsname{\color[rgb]{0,0,1}}%
      \expandafter\def\csname LT3\endcsname{\color[rgb]{1,0,1}}%
      \expandafter\def\csname LT4\endcsname{\color[rgb]{0,1,1}}%
      \expandafter\def\csname LT5\endcsname{\color[rgb]{1,1,0}}%
      \expandafter\def\csname LT6\endcsname{\color[rgb]{0,0,0}}%
      \expandafter\def\csname LT7\endcsname{\color[rgb]{1,0.3,0}}%
      \expandafter\def\csname LT8\endcsname{\color[rgb]{0.5,0.5,0.5}}%
    \else
      % gray
      \def\colorrgb#1{\color{black}}%
      \def\colorgray#1{\color[gray]{#1}}%
      \expandafter\def\csname LTw\endcsname{\color{white}}%
      \expandafter\def\csname LTb\endcsname{\color{black}}%
      \expandafter\def\csname LTa\endcsname{\color{black}}%
      \expandafter\def\csname LT0\endcsname{\color{black}}%
      \expandafter\def\csname LT1\endcsname{\color{black}}%
      \expandafter\def\csname LT2\endcsname{\color{black}}%
      \expandafter\def\csname LT3\endcsname{\color{black}}%
      \expandafter\def\csname LT4\endcsname{\color{black}}%
      \expandafter\def\csname LT5\endcsname{\color{black}}%
      \expandafter\def\csname LT6\endcsname{\color{black}}%
      \expandafter\def\csname LT7\endcsname{\color{black}}%
      \expandafter\def\csname LT8\endcsname{\color{black}}%
    \fi
  \fi
    \setlength{\unitlength}{0.0500bp}%
    \ifx\gptboxheight\undefined%
      \newlength{\gptboxheight}%
      \newlength{\gptboxwidth}%
      \newsavebox{\gptboxtext}%
    \fi%
    \setlength{\fboxrule}{0.5pt}%
    \setlength{\fboxsep}{1pt}%
\begin{picture}(7200.00,4320.00)%
    \gplgaddtomacro\gplbacktext{%
      \csname LTb\endcsname%%
      \put(504,408){\makebox(0,0)[r]{\strut{}$0$}}%
      \csname LTb\endcsname%%
      \put(504,954){\makebox(0,0)[r]{\strut{}$0.1$}}%
      \csname LTb\endcsname%%
      \put(504,1501){\makebox(0,0)[r]{\strut{}$0.2$}}%
      \csname LTb\endcsname%%
      \put(504,2048){\makebox(0,0)[r]{\strut{}$0.3$}}%
      \csname LTb\endcsname%%
      \put(504,2594){\makebox(0,0)[r]{\strut{}$0.4$}}%
      \csname LTb\endcsname%%
      \put(504,3140){\makebox(0,0)[r]{\strut{}$0.5$}}%
      \csname LTb\endcsname%%
      \put(504,3687){\makebox(0,0)[r]{\strut{}$0.6$}}%
      \csname LTb\endcsname%%
      \put(616,204){\makebox(0,0){\strut{}$0$}}%
      \csname LTb\endcsname%%
      \put(1239,204){\makebox(0,0){\strut{}$1$}}%
      \csname LTb\endcsname%%
      \put(1861,204){\makebox(0,0){\strut{}$2$}}%
      \csname LTb\endcsname%%
      \put(2484,204){\makebox(0,0){\strut{}$3$}}%
      \csname LTb\endcsname%%
      \put(3107,204){\makebox(0,0){\strut{}$4$}}%
      \csname LTb\endcsname%%
      \put(3730,204){\makebox(0,0){\strut{}$5$}}%
      \csname LTb\endcsname%%
      \put(4352,204){\makebox(0,0){\strut{}$6$}}%
      \csname LTb\endcsname%%
      \put(4975,204){\makebox(0,0){\strut{}$7$}}%
      \csname LTb\endcsname%%
      \put(5598,204){\makebox(0,0){\strut{}$8$}}%
      \csname LTb\endcsname%%
      \put(6220,204){\makebox(0,0){\strut{}$9$}}%
      \csname LTb\endcsname%%
      \put(6843,204){\makebox(0,0){\strut{}$10$}}%
    }%
    \gplgaddtomacro\gplfronttext{%
      \csname LTb\endcsname%%
      \put(5163,2382){\makebox(0,0)[l]{\strut{}$ \Delta t = 0.001$}}%
      \csname LTb\endcsname%%
      \put(5163,2586){\makebox(0,0)[l]{\strut{}$ \Delta t = 0.01$}}%
      \csname LTb\endcsname%%
      \put(5163,2790){\makebox(0,0)[l]{\strut{}$ \Delta t = 0.1$}}%
      \csname LTb\endcsname%%
      \put(5163,2994){\makebox(0,0)[l]{\strut{}$ \Delta t = 0.25$}}%
      \csname LTb\endcsname%%
      \put(5163,3198){\makebox(0,0)[l]{\strut{}$ \Delta t = 0.5$}}%
      \csname LTb\endcsname%%
      \put(5163,3402){\makebox(0,0)[l]{\strut{}$ \Delta t = 1$}}%
      \csname LTb\endcsname%%
      \put(3729,3993){\makebox(0,0){\strut{}$N_0=25$}}%
    }%
    \gplbacktext
    \put(0,0){\includegraphics[width={360.00bp},height={216.00bp}]{graf10}}%
    \gplfronttext
  \end{picture}%
\endgroup
}
    \caption{Primer poteka za začetne vrednosti $w=0.5, y=0, x=1$ in na levi $y=x$
    na desni pa $a=0.8$. $\lambda=5000$.}
    \label{slika10}
\end{figure}
\begin{figure}    
    \hspace{1cm}\resizebox{0.9\linewidth}{!}{\input{graf11.tex}}
    \caption{Potek koncentracije z~leve proti desni $x,z$ in $y$. Za začetne vrednosti smo
    vzeli $w=0.5, y=0, x=1, y=0.8$.}
    \label{slika11}
\end{figure}
Pri presežku začetne koncentracije $w$ (persulfata) vidimo da reakcija še veliko hitreje
poteče~\ref{slika12} (verjetno še hitreje kar vidimo s~tega da je funkcija odsekoma zvezna
zaradi premalo gostega računanja). Graf~\ref{slika13} prikazuje spreminjanje te hitrosti
od začetne koncentracije $w$. Na grafu~\ref{slika14} je prikazan potek koncentracij $x$ in
$y$ za različne vrednosti začetnega tiosulfata. Na desnem grafu vidimo, da z~zmanjšanjem 
koncentracije $z_0$ zmanjšamo tudi čas, ko se začne $y$ porabljati, kar je tudi smiselno
saj smo že na grafu~\ref{slika11} videli, da ko se porabi ves $w$ se začne porabljati
še jod.
\begin{figure}    
    \hspace{1cm}\resizebox{0.9\linewidth}{!}{\input{graf12.tex}}
    \caption{V~logaritemski časovni skali je prikazano spreminjanje koncentracij $x,z,y$
    z~velikim prebitkom koncentracije persulfata. Razmerje med $w$ in $x$ je bilo $w/x=20$.}
    \label{slika12}
\end{figure}
\begin{figure}    
    \hspace{1cm}\resizebox{0.9\linewidth}{!}{% GNUPLOT: LaTeX picture with Postscript
\begingroup
  \makeatletter
  \providecommand\color[2][]{%
    \GenericError{(gnuplot) \space\space\space\@spaces}{%
      Package color not loaded in conjunction with
      terminal option `colourtext'%
    }{See the gnuplot documentation for explanation.%
    }{Either use 'blacktext' in gnuplot or load the package
      color.sty in LaTeX.}%
    \renewcommand\color[2][]{}%
  }%
  \providecommand\includegraphics[2][]{%
    \GenericError{(gnuplot) \space\space\space\@spaces}{%
      Package graphicx or graphics not loaded%
    }{See the gnuplot documentation for explanation.%
    }{The gnuplot epslatex terminal needs graphicx.sty or graphics.sty.}%
    \renewcommand\includegraphics[2][]{}%
  }%
  \providecommand\rotatebox[2]{#2}%
  \@ifundefined{ifGPcolor}{%
    \newif\ifGPcolor
    \GPcolortrue
  }{}%
  \@ifundefined{ifGPblacktext}{%
    \newif\ifGPblacktext
    \GPblacktexttrue
  }{}%
  % define a \g@addto@macro without @ in the name:
  \let\gplgaddtomacro\g@addto@macro
  % define empty templates for all commands taking text:
  \gdef\gplbacktext{}%
  \gdef\gplfronttext{}%
  \makeatother
  \ifGPblacktext
    % no textcolor at all
    \def\colorrgb#1{}%
    \def\colorgray#1{}%
  \else
    % gray or color?
    \ifGPcolor
      \def\colorrgb#1{\color[rgb]{#1}}%
      \def\colorgray#1{\color[gray]{#1}}%
      \expandafter\def\csname LTw\endcsname{\color{white}}%
      \expandafter\def\csname LTb\endcsname{\color{black}}%
      \expandafter\def\csname LTa\endcsname{\color{black}}%
      \expandafter\def\csname LT0\endcsname{\color[rgb]{1,0,0}}%
      \expandafter\def\csname LT1\endcsname{\color[rgb]{0,1,0}}%
      \expandafter\def\csname LT2\endcsname{\color[rgb]{0,0,1}}%
      \expandafter\def\csname LT3\endcsname{\color[rgb]{1,0,1}}%
      \expandafter\def\csname LT4\endcsname{\color[rgb]{0,1,1}}%
      \expandafter\def\csname LT5\endcsname{\color[rgb]{1,1,0}}%
      \expandafter\def\csname LT6\endcsname{\color[rgb]{0,0,0}}%
      \expandafter\def\csname LT7\endcsname{\color[rgb]{1,0.3,0}}%
      \expandafter\def\csname LT8\endcsname{\color[rgb]{0.5,0.5,0.5}}%
    \else
      % gray
      \def\colorrgb#1{\color{black}}%
      \def\colorgray#1{\color[gray]{#1}}%
      \expandafter\def\csname LTw\endcsname{\color{white}}%
      \expandafter\def\csname LTb\endcsname{\color{black}}%
      \expandafter\def\csname LTa\endcsname{\color{black}}%
      \expandafter\def\csname LT0\endcsname{\color{black}}%
      \expandafter\def\csname LT1\endcsname{\color{black}}%
      \expandafter\def\csname LT2\endcsname{\color{black}}%
      \expandafter\def\csname LT3\endcsname{\color{black}}%
      \expandafter\def\csname LT4\endcsname{\color{black}}%
      \expandafter\def\csname LT5\endcsname{\color{black}}%
      \expandafter\def\csname LT6\endcsname{\color{black}}%
      \expandafter\def\csname LT7\endcsname{\color{black}}%
      \expandafter\def\csname LT8\endcsname{\color{black}}%
    \fi
  \fi
    \setlength{\unitlength}{0.0500bp}%
    \ifx\gptboxheight\undefined%
      \newlength{\gptboxheight}%
      \newlength{\gptboxwidth}%
      \newsavebox{\gptboxtext}%
    \fi%
    \setlength{\fboxrule}{0.5pt}%
    \setlength{\fboxsep}{1pt}%
\begin{picture}(7200.00,4320.00)%
    \gplgaddtomacro\gplbacktext{%
      \csname LTb\endcsname%%
      \put(616,408){\makebox(0,0)[r]{\strut{}$0$}}%
      \csname LTb\endcsname%%
      \put(616,1064){\makebox(0,0)[r]{\strut{}$0.05$}}%
      \csname LTb\endcsname%%
      \put(616,1720){\makebox(0,0)[r]{\strut{}$0.1$}}%
      \csname LTb\endcsname%%
      \put(616,2375){\makebox(0,0)[r]{\strut{}$0.15$}}%
      \csname LTb\endcsname%%
      \put(616,3031){\makebox(0,0)[r]{\strut{}$0.2$}}%
      \csname LTb\endcsname%%
      \put(616,3687){\makebox(0,0)[r]{\strut{}$0.25$}}%
      \csname LTb\endcsname%%
      \put(728,204){\makebox(0,0){\strut{}$2$}}%
      \csname LTb\endcsname%%
      \put(1492,204){\makebox(0,0){\strut{}$3$}}%
      \csname LTb\endcsname%%
      \put(2257,204){\makebox(0,0){\strut{}$4$}}%
      \csname LTb\endcsname%%
      \put(3021,204){\makebox(0,0){\strut{}$5$}}%
      \csname LTb\endcsname%%
      \put(3786,204){\makebox(0,0){\strut{}$6$}}%
      \csname LTb\endcsname%%
      \put(4550,204){\makebox(0,0){\strut{}$7$}}%
      \csname LTb\endcsname%%
      \put(5314,204){\makebox(0,0){\strut{}$8$}}%
      \csname LTb\endcsname%%
      \put(6079,204){\makebox(0,0){\strut{}$9$}}%
      \csname LTb\endcsname%%
      \put(6843,204){\makebox(0,0){\strut{}$10$}}%
    }%
    \gplgaddtomacro\gplfronttext{%
      \csname LTb\endcsname%%
      \put(4603,3198){\makebox(0,0)[l]{\strut{}$\beta =1$}}%
      \csname LTb\endcsname%%
      \put(4603,3402){\makebox(0,0)[l]{\strut{}$\beta_s =5 \beta, \, \beta_r=4\beta$}}%
      \csname LTb\endcsname%%
      \put(3785,3993){\makebox(0,0){\strut{}$N_0=25$}}%
    }%
    \gplbacktext
    \put(0,0){\includegraphics[width={360.00bp},height={216.00bp}]{graf13}}%
    \gplfronttext
  \end{picture}%
\endgroup
}
    \caption{V~logaritemski časovni skali je prikazano spreminjanje koncentracij $x,z,y$
    z~razmerjem $\lambda=10$ za različne začetne koncentracije $w$.}
    \label{slika13}
\end{figure}
\begin{figure}    
    \hspace{1cm}\resizebox{0.9\linewidth}{!}{\input{graf14.tex}}
    \caption{Spreminjanje koncentracij $x,y$ z~razmerjem $\lambda=10$ za različne začetne 
    vrednosti $z$.}
    \label{slika14}
\end{figure}


\end{document}
